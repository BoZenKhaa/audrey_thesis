\chapter*[Summary]{Summary}
\label{summary}
\addcontentsline{toc}{chapter}{Summary}

\quad\\

\quad\\

\newpage

\noindent Cancers of the colon and the rectum are the third most common type of cancer in the world, affecting over one million people annually. Many clinical, genetic, and nutritional risk factors that modify risk of colorectal cancer have been identified. A history of colorectal cancer in one and two or more first-degree family members increases personal risk of developing colorectal cancer two- and four-fold, respectively. A personal history of colorectal adenomas, benign established precursors of colorectal carcinomas, also increases risk of developing colorectal carcinomas, particularly if the adenomas are of the villous variety, at least 1 cm, or develop in multiples. Inheritance of cancer-associated genes also strongly increases personal risk for developing colorectal cancer. Lynch syndrome is a well-known inherited cancer syndrome, in which individuals with Lynch syndrome inherit a pathogenic germline mutation in one of the mismatch repair (MMR) genes, which results in microsatellite instability in tumours. Compared to the general population, where the lifetime risk of developing colorectal cancer is 5-6\%, Lynch syndrome individuals have a 70-85\% risk of developing colorectal cancer to age 70. Dietary and lifestyle factors also influence the risk of developing colorectal cancer, by possibly interacting with genetic susceptibility to these exposures due to specific variants in related metabolising genes. The World Cancer Research Fund/American Institute for Cancer Research estimates that roughly one-third, and even up to one-half in some countries, of all colorectal cancers could be prevented by regular physical activity, an avoidance of body fatness within the normal range of body weight, and proper food and nutrition. Micronutrients such as folate and other B vitamins have also been found to decrease risk of developing colorectal cancer.

\noindent Aberrant DNA methylation occurs during colorectal carcinogenesis. Global hypomethylation as well as gene-specific hypermethylation of promoter regions have been observed in colorectal tumours. Both these processes depend on the efficiency of one-carbon metabolism, where folate and other B vitamins play a key role. Folate is a mediator of one-carbon groups in one-carbon metabolism, while vitamins B2, B6, and B12 are cofactors for biological reactions. Methionine is an essential amino acid that is converted to S-adenosylmethionine, the universal donor of methyl groups for a multitude of reactions including DNA methylation. The studies described in this thesis aim to improve our knowledge about the impact of B vitamin status on global DNA methylation in persons at differential risk for colorectal cancer.

\noindent \textbf{Chapter 1} is a general introduction to B vitamins, DNA methylation, and colorectal carcinogenesis. A description of the studies conducted as part of this thesis is also included.

\noindent \textbf{Chapter 2} is a literature review that summarises the impact of nutrition on global and gene-specific DNA methylation in different cancer sites as explored in epidemiological studies.

\noindent \textbf{Chapter 3} is based on data  collected from persons at low risk for developing colorectal cancer. Here, a cross-sectional study was conducted to describe the impact of plasma B vitamins on long interspersed nuclear element-1 (LINE-1) methylation in low risk persons in order to improve our understanding of cancer processes. We used data from 1,142 age-and sex-stratified randomly selected individuals who had participated in a population-based survey. Those in the extremes (10th (2.69-5.62 nmol/L) and 90th (37.24-95.94 nmol/L) percentiles) of plasma folate concentrations were selected. In a total of 275 healthy, cancer-free persons with plasma folate in the 10th (n=138) and 90th (n=137) percentiles, we describe leukocyte LINE-1 DNA methylation in relation to circulating concentrations of B vitamins (folate, riboflavin, vitamin B6 species (pyridoxal 5'-phosphate, pyridoxal, and 4-pyridoxic acid), cobalamin, methylmalonic acid (MMA)) and methionine, and dietary and lifestyle characteristics. Age, sex, and \emph{MTHFR} C677T genotype were also examined as possible predictors of LINE-1 methylation in the extremes separately using multivariable linear regression adjusted for each other. Our results suggest no difference in leukocyte LINE-1 DNA methylation between those in the 10th and in the 90th percentile for plasma folate. Mean percentage of LINE-1 DNA methylation (IQR) for those in the 10th percentile was 73.7 (72.6-74.6) and 73.6 (72.5-74.7) for those in the 90th percentile. Age at blood draw was inversely related to LINE-1 methylation in both extremes ($\beta$ estimate (95\%CI) of 0.07 (-0.13, -0.01) for the 10th percentile and -0.08 (-0.14, -0.02) for the 90th percentile), and LINE-1 DNA methylation was lower in females compared with males in the 10th percentile of plasma folate ($\beta$ estimate (95\%CI) of -0.68 (-1.37, 0.01)), while there was no association with \emph{MTHFR} C677T genotype.

\noindent In a population at medium risk for colorectal cancer, the effect of folic acid supplementation on global DNA methylation was investigated (\textbf{Chapter 4}). We examined whether long-term daily supplementation with 0.8 mg of folic acid increases global DNA methylation compared with placebo in individuals with elevated plasma homocysteine. We also investigated if these effects are modified by \emph{MTHFR} C677T genotype. Two hundred sixteen participants out of 818 subjects who had participated in a randomised double-blind placebo-controlled trial were selected, pre-stratified on \emph{MTHFR} C677T genotype and matched on age and smoking status. They were allocated to receive either folic acid (0.8 mg/d; n = 105) or placebo treatment (n = 111) for three years. Peripheral blood leukocyte DNA methylation and serum and erythrocyte folate were assessed. Global DNA methylation was measured using liquid chromatography-tandem mass spectrometry and expressed as a percentage of 5-methylcytosines versus the total number of cytosine. Analysis of covariance (ANCOVA) was used to estimate differences between groups at baseline and after the intervention and also to compare the effect of supplementation on global DNA methylation, adjusted for baseline DNA methylation and the variables on which subjects were matched (i.e. age and current smoking status). Following three years of folic acid supplementation, there was no difference in global DNA methylation between those randomised to folic acid and those in the placebo group (difference = 0.008, 95\%CI =20.05,0.07, P = 0.79). There was also no difference between treatment groups when stratified for MTHFR C677T genotype (CC, n = 76; CT, n = 70; TT, n = 70), baseline erythrocyte folate status or baseline DNA methylation levels. Our results indicate that in moderately hyperhomocysteinemic men and women, long-term folic acid supplementation does not increase global DNA methylation in peripheral blood leukocytes.

\noindent Moving further along the continuum for colorectal cancer risk, \textbf{Chapter 5} illustrates the association between plasma B vitamins and LINE-1 DNA methylation in leukocytes of persons at high risk for developing colorectal cancer (i.e. patients with at least one histologically confirmed colorectal adenoma ever in their life). We used LINE-1 bisulfite pyrosequencing to measure global DNA methylation levels in leukocytes of 281 colorectal adenoma patients. Multivariable least squares linear regression was used to assess associations between plasma B vitamin concentrations and LINE-1 methylation levels. The final linear regression models included the covariates age, sex, BMI, alcohol intake, smoking status, family history of colorectal cancer, and other measured analytes (i.e. mutual B vitamins and methionine). Analyses were also stratified by number of lifetime adenomas (one vs at least two), family history of colorectal cancer in a first-degree relative, and \emph{MTHFR} C677T genotype. Plasma folate was inversely associated with LINE-1 methylation in the entire population of colorectal adenoma patients (adjusted $\beta$ estimate (95\%CI) of -1.46 (-2.65, -0.27)). For those with one adenoma, plasma methionine was positively associated with LINE-1 methylation (adjusted $\beta$ estimate of 3.47 (-0.40, 7.33)). Based on our results, there is no evidence to suggest that in colorectal adenoma patients (persons at high risk for colorectal cancer), plasma folate concentrations are positively related to LINE-1 methylation in leukocytes.

\noindent \textbf{Chapter 6} presents a prospective cohort study that was conducted to explore the impact of folate, vitamin B2, vitamin B6, vitamin B12, and methionine intake on colorectal carcinogenesis in 470 mismatch repair gene mutation carriers (i.e. persons with Lynch Syndrome), who have a very high lifetime risk for developing colorectal cancer. Dietary intakes were assessed by food frequency questionnaire. Cox regression models with robust sandwich covariance estimation, adjusted for age, sex, physical activity, number of colonoscopies during person-time, NSAID use, and mutual vitamins were used to calculate hazard ratios (HR) and 95\% confidence intervals (95\% CI). Analyses were also stratified by \emph{MTHFR} C677T genotype. Compared to the lowest tertile of intake, adjusted HRs (95\%CI)s for CRT development in the highest tertile are 1.06 (0.59-1.91) for folate, 0.77 (0.39-1.51) for vitamin B2, 0.98 (0.59-1.62) for vitamin B6, 1.24 (0.77-2.00) for vitamin B12, and 1.36 (0.83-2.20) for methionine. Low vitamin B2 and low methionine intake were statistically significantly associated with an increased risk of CRT in \emph{MTHFR} 677TT individuals compared to a combined reference of persons with low intake and CC genotype, but there was no significant interaction. In conclusion, there is no suggestion that intake of any dietary B vitamin or methionine is associated with CRT development among those with Lynch Syndrome.

\noindent The final chapter (\textbf{Chapter 7}) contains the general discussion, where we consider the results from all our studies as presented in the previous chapters as a whole. Our results in comparison to studies by others are also considered to help us make a more thoughtful and thorough insight into the scientific consequences of our work. The overall findings of this thesis suggest that status of folate and other B vitamins may not be associated with global DNA methylation nor LINE-1 DNA methylation in leukocytes except possibly in persons with a history of colorectal adenomas. Because neither our results nor those of others provide evidence to suggest that a low or high folate status could influence leukocyte DNA methylation in individuals at differential risk for colorectal cancer, these results taken together would not endorse revisions to the current recommendations. Because DNA methylation patterns are not static, but change over the course of a lifetime and in response to different environmental factors, it would be informative to perform an epigenome-wide association study every 10 years starting in early life and follow these persons over a lifetime. Furthermore, mathematical modelling and in vivo studies have shown that DNA methylation competes with DNA synthesis pathways for folate species, so a long-term randomised controlled trial of folic acid supplementation in healthy younger persons would help improve our knowledge about the impact of folic acid on DNA synthesis and methylation from a younger age onwards.
