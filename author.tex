\chapter*{About the author}
\chaptermark{About the author}
\label{About the author}
\addcontentsline{toc}{chapter}{About the author}

\newpage

\noindent \textbf{Curriculum vitae}

\noindent Audrey Ying-Chee Jung was born in Kamloops, a town in south central British Columbia, Canada. After completing her undergraduate degree in Biological Sciences, she moved to the lovely little historic town of Wageningen, the Netherlands, made possible by an award from the Wageningen University Scholars Programme. She graduated with a Master's in Human Nutrition with a specialisation in Epidemiology. For her Master's thesis, she spent 4 months at the Jiangsu Provincial Center for Disease Control and Prevention in the even more historical, but large and bustling city of Nanjing, China. From there, she moved to Seattle, Washington, USA for 4 months to perform an internship within the Cancer Prevention Program at the Fred Hutchinson Cancer Research Center. In mid-November 2007, she started her doctorate at Radboud University Medical Center in Nijmegen, the Netherlands, where in a small park by her house, she broke her right foot. Although burdened by a fear of public speaking and cats, Audrey finds comfort in reading, winter sports, dim sum, and zoete dropjes.


\noindent \textbf{List of publications}

\noindent \textbf{Refereed journal publications}

\noindent Heine-Br\"oring RC, Winkels RM, Botma A, van Duijnhoven FJ, \textbf{Jung AY}, Kleibeuker JH, Nagengast FM, Vasen HF, Kampman E. Dietary supplement use and colorectal adenoma risk in individuals with Lynch Syndrome: the GEOLynch Cohort Study. PLoS One. 2013 Jun 18;8(6):e66819.

\noindent \textbf{Jung AY}, Botma A, Lute C, Blom HJ, Ueland PM, Kvalheim G, Midttun O, Nagengast F, Steegenga W, Kampman E. Plasma B vitamins and LINE-1 DNA methylation in leukocytes of patients with a history of colorectal adenomas. Molecular Nutrition and Food Research. 2012 Nov 7. doi: 10.1002/mnfr.201200069.

\noindent \textbf{Jung AY}, Smulders Y, Verhoef P, Kok FJ, Blom H, Kok RM, Kampman E, Durga. No Effect of Folic Acid Supplementation on Global DNA Methylation in Men and Women with Moderately Elevated Homocysteine. PLoS ONE 2011; 6(9): e24976. doi:10.1371/journal.pone.0024976.

\noindent \textbf{Jung AY}, Poole EM, Bigler J, Whitton J, Potter JD, Ulrich CM. DNA methyltransferase and alcohol dehydrogenase: gene-nutrient interactions in relation to risk of colorectal polyps. Cancer Epidemiology Biomarkers \& Prevention 2008 Feb; 17(2):330-338. doi: 10.1158/1055-9965.EPI-07-2608.

\noindent \textbf{Jung AY}, van Duijnhoven FJB, Nagengast FM, Botma A, Heine-Broring RC, Kleibeuker JH, Vasen HFA, Harryvan JL, Winkels RM, Kampman E. Dietary B vitamin and methionine intake and MTHFR C677T genotype on risk of colorectal tumours in patients with Lynch Syndrome: the GEOLynch Cohort Study. Submitted.

\noindent \textbf{Jung AY}, den Heijer M, Nagengast FM, Blom H, Lute C, Kiemeney LALM, Ueland PM, Midttun O, Kvalheim G, Steegenga W, Kampman E. Age and sex but not plasma folate nor MTHFR C677T genotype are related to LINE-1 methylation in a healthy population. Submitted.

\noindent \textbf{Book chapters}
\noindent Audrey Jung and Ellen Kampman. (2011). Nutrition, Epigenetics, and Cancer: An Epidemiological Perspective. In Mihai D. Niculescu and Paul Haggarty, editors, Nutrition in Epigenetics (pp.329-343). UK: Blackwell Publishing Ltd. ISBN 978-0-8138-1605-0.
