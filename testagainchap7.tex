\chapter{General discussion}
\label{chap7_generaldiscussion}
\AddEverypageHook{%
	\ifthenelse{\equal{\Chaptername}{General discussion}}{
		\ifthenelse{\isodd{\thepage}}%
		{\TileWallPaper{17.4cm}{24.4cm}{thumbCh7.pdf}}%
		{\ClearWallPaper}
	} {\ClearWallPaper}
} 


\newpage

\noindent The aim of this thesis was to investigate the role of plasma B vitamins as determinants of global DNA methylation, an intermediate biomarker in colorectal carcinogenesis, in low (Nijmegen Biomedical Study (NBS)), medium (Folic Acid and Carotid Intima-media Thickness (FACIT)), and high risk (POLIEP-follow-up) populations for colorectal cancer. B vitamin intake and risk of developing colorectal tumours was also investigated in a population with an inherited mutation conferring very high risk for developing colorectal cancer (GEOLynch). The purpose of this chapter is to summarise the main findings of the studies in this research project as well as compare and contrast these results to those known in literature. Strengths and limitations of our studies will also be discussed. The chapter concludes with final remarks and based on our results, we offer implications for public health and suggestions for future research that will further our understanding of one-carbon metabolism and DNA methylation in colorectal carcinogenesis.

\section[]{Main findings for folate/folic acid} % level 1
\noindent Table \ref{table7_1} summarises the main findings of folate/folic acid by study population. In a low risk population (NBS), healthy persons between 50-70 years of age without a personal history of cancer participating in a population-based survey were ranked based on their plasma folate status and those in the 10th percentile (n=138) and the 90th (n=137) percentile were selected. We did not observe a difference in leukocyte LINE-1 DNA methylation levels between the two groups. We did observe, however, that age was inversely associated with LINE-1 methylation independent of plasma folate concentrations. LINE-1 DNA methylation was also lower in females compared with males, and \emph{MTHFR} C677T genotype not associated with LINE-1 methylation.

\noindent In a medium risk population, participants in the randomised-controlled trial (n=218) were cancer-free men and post-menopausal women between 50-70 years of age with moderately elevated homocysteine concentrations. There was no difference in global DNA methylation, measured by liquid chromatography-tandem mass spectrometry, at baseline and following 800 $\mu$g daily supplementation with folic acid for 3 years between the treatment and placebo groups. There was no difference in global DNA methylation between those in the folate group and those in the placebo group after stratification by \emph{MTHFR} C677T genotype, baseline erythrocyte folate concentrations, baseline DNA methylation levels, age, current smoking status, alcohol intake, body mass index, or sex.

\noindent In a high risk population consisting of sporadic colorectal adenoma patients (n=281), plasma folate was inversely related to LINE-1 DNA methylation. After stratification by number of lifetime adenomas (1 \emph{vs}. $\geq$2), a persistent inverse association between plasma folate and LINE-1 methylation was observed in both strata. Plasma folate was inversely related to LINE-1 methylation in \emph{MTHFR} 677CC individuals, but not in CT nor TT persons. Plasma folate was borderline inversely associated with LINE-1 methylation for those with a family history of colorectal cancer, but not in those without a family history.

\noindent In a prospective cohort study of a very high risk population for developing colorectal cancer -- germline MMR gene mutation carriers -- 131 out of 470 persons developed a CRT during a median person-time of 28.0 months. There was no association between dietary folate intake and colorectal tumour risk. The MMR gene mutation carriers in our study are scattered throughout the Netherlands making blood collection additionally challenging, but we were able to collect blood samples from a subset of participants, namely those in the Nijmegen cohort (n=88). Therefore, we were able to investigate preliminary cross-sectional relationships between plasma B vitamins and global DNA methylation in these individuals. Following exclusion of those for which we had no information on plasma folate or DNA methylation, 78 persons were included in the analyses. DNA methylation levels, measured using the luminometric methylation assay (LUMA), were between 26.6-75.8\%. Multivariable linear regression was used to evaluate associations between plasma folate and global DNA methylation adjusting for age, sex, and mutual analytes. There was no association between plasma folate and global DNA methylation ($\beta$-estimate 0.01, 95\%CI -0.08, 0.07). The results stemming from these preliminary analyses will not be discussed further in this thesis.

\noindent In conclusion, there was no association between plasma folate and leukocyte LINE-1 methylation in a low risk population, between folic acid supplementation and leukocyte global DNA methylation in a medium risk population, between plasma folate and leukocyte DNA methylation in a very high risk population (preliminary findings), nor between dietary folate intake and CRT development in a very high risk population. Only in a high risk population -- persons with a history of sporadic CRA -- was plasma folate inversely related to leukocyte LINE-1 methylation.

\section[]{Main findings for other B vitamins and methionine} % level 1
\noindent Table \ref{table7_2} summarises findings for other B vitamins and methionine by study population. Based on the study design for our low and medium risk populations, other B vitamins and methionine could not be analysed as a main exposure.

\noindent In patients with a history of colorectal adenomas, there was a positive association between plasma methionine and LINE-1 methylation in those without a family history of colorectal cancer, but not in those with a family history of colorectal cancer. We additionally observed an inverse association between plasma riboflavin and LINE-1 methylation in \emph{MTHFR} 677TT individuals but not in CC nor CT persons.

\noindent In the prospective cohort study of germline MMR gene mutation carriers, no association between dietary intake of any B vitamin nor methionine and CRT risk was observed, and stratified analyses with \emph{MTHFR} C677T genotype did not reveal significant nutrient-gene interactions for any of the B vitamins or methionine.

\section[]{Main findings -- a comparison and contrast with others} % level 1
\noindent Our results seem to be fairly unexpected when compared to the bulk of current literature. In order to improve our understanding of our results as a contribution to the literature, here we describe and give possible explanations for the differences between our results and earlier reports from others. The research described in this thesis focuses predominantly on the relationships between plasma B vitamins and leukocyte LINE-1 DNA methylation across various populations at differential risk for colorectal cancer. In this section, we compare (the similarities) and contrast (the differences) the results of our studies with those in the literature in order to give a better overview of how our studies complement those of others and how together these studies contribute to our overall understanding so far of the relationships between plasma B vitamins and LINE-1 methylation in different populations.

\noindent We first examine all randomised-controlled trials of folic acid supplementation on leukocyte global DNA methylation in low and medium risk individuals (Table \ref{table7_3}). Then we recapitulate investigations of controlled intake of dietary folate and folic acid from enriched foods on leukocyte global DNA methylation in low risk populations (Table \ref{table7_4}) followed by reports that have investigated associations between folate status (intake and plasma concentrations) and leukocyte global DNA methylation also in low risk populations (Table \ref{table7_5}). This section concludes with studies that have looked at folate status and leukocyte global DNA methylation in high risk populations (i.e. persons with a history of colorectal adenoma or colorectal carcinoma) (Table \ref{table7_6}).

\subsection{Low and medium risk populations} % level 2
\noindent At the time of submission of this thesis, to the best of our knowledge, there were two other randomised placebo-controlled trials that investigated the effects of folic acid supplementation on leukocyte global DNA methylation in low and medium risk individuals (Table \ref{table7_3}). There was no effect of folic acid supplementation on leukocyte global DNA methylation in all three trials. The methyl acceptance assay was used in two of the trials (not in our own study); this semi-quantitative method reflects the capacity of DNA to accept [\textsuperscript{3}H]-labelled methyl groups in the presence of \emph{SssI} prokaryotic methylase enzyme. [\textsuperscript{3}H]methyl group incorporation into DNA is inversely related to DNA methylation, so high [\textsuperscript{3}H]-methyl group incorporation suggests low global DNA methylation and \emph{vice versa} \cite{c71}.

\noindent As far as we know, there are currently four studies that have investigated controlled intake of dietary folate and folic acid from enriched foods on global DNA methylation in leukocytes of low risk persons (Table \ref{table7_4}). All of these studies were carried out in the United States, where mandatory enrichment of flour, rice, pasta, and other grain products with 140 $\mu$g of folic acid per 100g began in 1998 \cite{c72}. There were no differences in DNA methylation between the groups at baseline in all studies. In two studies, folate depletion was followed by an increase in [\textsuperscript{3}H]methyl incorporation implying a decrease in global DNA methylation \cite{c73,c74}. Following folate repletion, a decrease in [\textsuperscript{3}H]methyl incorporation (increased DNA methylation) was reported \cite{c73}; using liquid chromatography tandem mass spectrometry to measure DNA methylation, however, seemed to result in an increase in the mCyt/tCyt ratio (increased DNA methylation) only in persons with the \emph{MTHFR} 677TT genotype \cite{c75}. There was no change in DNA methylation after folate depletion nor repletion when the methyl acceptance assay was used to measure DNA methylation in the same study \cite{c75}. 
 
\noindent A total of seven studies, including one of our own, have explored the association between folate status (biomarker or intake) and global DNA methylation in leukocytes of low risk individuals (Table \ref{table7_5}). Plasma folate was explored in three studies \cite{c76,c77,c78}, dietary folate intake in three others \cite{c79,c710,c711}, and red blood cell folate in one study \cite{c712}. There was no association in the majority of studies. In the three studies with plasma folate, there was no association with global DNA methylation in two studies \cite{c77,c78}. There was a positive association between red blood cell folate \cite{c712} in one study and plasma folate \cite{c76} in another study with global DNA methylation, but only in those with the \emph{MTHFR} 677TT genotype and not in persons with the CC genotype.

\subsection{High risk population} % level 2 
\noindent As can be seen in table \ref{table7_6}, there have been two studies that have investigated either the effect of folic acid supplementation on leukocyte DNA methylation \cite{c713} or associations between plasma folate and leukocyte DNA methylation in persons at high risk for colorectal cancer (i.e. colorectal adenoma patients) \cite{c714}. The effect of folic acid supplementation on leukocyte DNA methylation has been explored in a British population of CRA patients. Results from this RCT conducted by Pufulete \emph{et al}. show that DNA methylation in leukocytes of CRA patients increased after 10 weeks 400 $\mu$g folic acid supplementation daily compared with the placebo group \cite{c713}. Conversely, our own study has indicated an inverse association between plasma folate and LINE-1 methylation \cite{c714}. Global DNA methylation was measured differently between the two studies (global DNA methylation \cite{c713} \emph{vs} LINE-1 methylation \cite{c714}), and the methods used to quantify DNA methylation also differed (methyl acceptance assay \cite{c713} \emph{vs} pyrosequencing \cite{c714}), which could partially contribute to the discrepancies in methylation seen between the studies.

\subsection{Very high risk population} % level 2 
\noindent To the very best of our knowledge, associations between B vitamins and DNA methylation in very high risk populations, such as a Lynch syndrome, are still unknown. Neither dietary information nor DNA methylation measurements were assessed in reports describing a later age of onset of colorectal cancer for Lynch syndrome individuals with the \emph{MTHFR} 677CT or TT genotype compared to those with the CC genotype \cite{c715,c716}. In chapter 6 of this thesis, our results seem to indicate null associations between B vitamin and methionine intake and colorectal tumour risk. Additional preliminary cross-sectional analyses in a small population also did not suggest significant relationships between plasma B vitamins and methionine with leukocyte global DNA methylation, but stratified analyses and replication in larger studies are needed to better understand these relationships.

\section{Challenges in estimating B vitamin status} % level 1
\subsection{Using food frequency questionnaires} % level 2
\noindent In our very high risk population, the GEOLynch study, we have used the same self-administered food frequency questionnaire (FFQ) as we used in our other studies to assess habitual dietary intake. This questionnaire has been previously validated for folate and vitamin B12 \cite{c717} in addition to fats and cholesterol \cite{c718}. Because blood samples were only provided by 88 persons in the Nijmegen cohort of the GEOLynch study, we elected to use dietary intake information, which was available from all 470 participants.

\noindent Dietary folate intake was expressed as micrograms ($\mu$g) per day of dietary folate equivalents (DFEs), a unit which accounts for the differences in absorption of naturally-occurring food folates and the more bioavailable synthetic folic acid form of folate. The number of DFEs in natural foods is equivalent to the micrograms of folate in that food, and the number of DFEs in fortified foods is the micrograms of food folate in that food plus 1.7 times the micrograms of added folic acid \cite{c719}. Folic acid from supplements is not commonly used in the Netherlands \cite{c720} and our FFQ was also not validated for folic acid from supplements -- only folate from foods was assessed \cite{c717}. Often validations of FFQs can show better correlations if supplement use was included in the analyses \cite{c721,c722,c723}. Including supplement use in our linear regression models and Cox regression models did not change our results appreciably.

\noindent FFQs in assessing habitual dietary intake are regularly used in epidemiological studies because they are relatively inexpensive, measure long-term dietary intake, and straightforward for participants to complete \cite{c724} although there is more measurement error involved compared with recalls and dietary records and may not be entirely accurate \cite{c725,c726}. Difficulties in collecting information about all possible foods, frequency of foods eaten and serving size add to the measurement error involved with FFQs \cite{c727}. Recall bias is less likely to happen in cohort studies than in case-control studies because diets are reported before the occurrence of disease; however, other measurement errors may exist. The ranking of subjects can be influenced when persons with high intake under-report their intake and when persons with low intake over-report their intake \cite{c728}. The associated random variation would furthermore attenuate risk estimates \cite{c729}.

\noindent As mentioned, the FFQ used in all our studies including the GEOLynch study has previously been validated for folate and vitamin B12 \cite{c717}. For 81 Lynch Syndrome participants of the Nijmegen cohort in our study, Spearman correlation coefficients between B vitamin intake and related biomarkers were 0.11 for folate, 0.13 for vitamin B2, 0.13 for vitamin B6, -0.05 for vitamin B12, and 0.24 for methionine, demonstrating poor correlation between dietary intake and plasma vitamers. This is not surprising, as dietary folate intake and plasma folate are usually poorly correlated \cite{c722,c730,c731}. More specifically, for the population in which our FFQ was validated, the correlation coefficient between serum folate and the FFQ was 0.20, similar to blood folate-FFQ folate correlations from other studies \cite{c722,c730,c731}. The minor differences between this correlation coefficient and the one from our Lynch population could be due to differences in blood collection protocols, methods to measure plasma/serum folate, time between blood collection and analysis, seasonal variation, and sample size.

\subsection{Using biomarkers} % level 2
\noindent In the NBS and POLIEP-follow up studies, vitamers in plasma were measured using microbiological assays and liquid chromatography-tandem mass spectrometry, and the biomarkers have shown good intraindividual reproducibility and are regarded as reliable measures in epidemiological studies \cite{c732}. For plasma folate, we have measured the vitaminer 5-methyltetrahydrofolate using a well-known microbiological assay \cite{c733}. There are many forms of folate in plasma with 5-methyltetrahydrofolate being the predominant form as well as the form that donates methyl groups for DNA methylation \cite{c734}.

\noindent There is disagreement whether B vitamin concentrations in blood reflect those in colorectal mucosa. Three studies have reported direct correlations between serum and red blood cell folate with colonic mucosal folate \cite{c735,c736,c737} while other studies did not observe significant correlations \cite{c734,c738,c739}. Tissue folate extraction, tissue folate determination (i.e. two different microbiological assays \cite{c736,c739} \emph{vs} liquid chromatography \cite{c734}), and source of folate (i.e. colonocytes \cite{c738,c739} \emph{vs}. colonic mucosa \cite{c734,c735,c736}) could partially account for the disparate results between studies. Furthermore, tissue storage time could also contribute to these differences (i.e. 6 years 34 \emph{vs} 2 months \cite{c736}, as folate loss in blood stored at -20\textsuperscript{o}C can already be detected after a few years \cite{c740,c741} and in certain cases even as early as 6 days after blood draw \cite{c742}. As well, these studies all tended to be small, which could also contribute to the observed differences in results.

\noindent Results from a 1-year randomised controlled trial with 5 mg folic acid daily in 17 persons with a history of colorectal adenomas showed serum and red blood cell folate concentrations directly correlating with colonic mucosal folate concentrations at baseline, 6 months, and at 1 year \cite{c735}. Further inspection of the data revealed that at the 6 month and 1 year time points, serum and red blood cell folate were directly related to colonic folate in persons in the placebo group (at 6 months, $\rho$ of 0.79 for serum folate and $\rho$ of 0.67 for red blood cell folate; at 1 year, $\rho$ of 0.79 for serum folate and $\rho$ of 0.68 for red blood cell folate) but not in persons in the folic acid group (at 6 months, $\rho$ of 0.25 for serum folate and $\rho$ of 0.03 for red blood cell folate; at 1 year, $\rho$ of 0.49 for serum folate and $\rho$ of 0.12 for red blood cell folate), demonstrating that perhaps serum and red blood cell folate are correlated with colonic mucosal folate under conventional physiological ranges of folic acid concentrations but not following supraphysiological doses \cite{c735}. There is no national food fortification programme in the Netherlands, where dietary supplement use, while increasing, is not yet commonplace (18-41\% of the adult Dutch population between spring and autumn and 29-54\% in the winter \cite{c720}), so it is unlikely that persons in our study populations (including those in our randomised controlled trial where participants were allocated to either placebo or 0.8 mg/day folic acid) were exposed to supraphysiological doses of folic acid, which may settle possible reservations concerning correlations between concentrations of folate in plasma and colonic mucosal in our studies.

\noindent Measurement of biomarkers and other exposures in our studies is similar to those performed in other studies. And like other studies, measurement error is unavoidable. Random measurement error in plasma metabolites (exposure) will bias regression coefficients towards the null \cite{c743,c744}.
 
\section{Assessing global DNA methylation} % level 1 
\noindent We have used several methods to assess global DNA methylation. As mentioned, we used liquid chromatography-tandem mass spectrometry 8 to assess global DNA methylation in the randomised controlled trial. We have used bisulfite PCR and pyrosequencing of LINE-1 to estimate global DNA methylation in the NBS and POLIEP-follow-up studies. In the GEOLynch study, we used the LUMA method to measure global DNA methylation. We selected to use LUMA and LINE-1 pyrosequencing because of their frequent use in numerous epidemiological studies \cite{c745,c746}. Prior to using these analytical techniques, DNA must be isolated from blood. We have collected blood in EDTA Vacutainers, which were then spun, and plasma and buffy coat were removed and stored at -80\textsuperscript{o}C in separate vials. DNA was extracted and isolated from buffy coat using two different methods -- for the NBS and POLIEP-follow-up studies, a Hamilton STAR robot was used, and for the GEOLynch study, DNA was isolated manually using a kit (Qiagen Puregene Kit B). There are reports that different DNA isolation techniques could yield variable global DNA methylation levels using LUMA, and this should be taken into account when interpreting results from the LUMA method \cite{c747}.

\noindent The LUMA method was developed by Karimi and colleagues 48 and employs the restriction endonucleases \emph{Hpa}II and \emph{Msp}I, respectively. \emph{Hpa}II is sensitive to 5-methyldeoxycytosine (5-mdC) methylation at all CCGG sequences in the genome while \emph{Msp}I is insensitive. A third enzyme called \emph{Eco}RI digests the DNA and creates an overhang with a sequence devoid of any cytosine and guanine nucleotides and serves as a control of DNA in the assay. The percentage of DNA methylation is expressed as $$[1-(\text{\emph{Hpa}II}  / \text{\emph{Eco}RI}  \sum \text{G}/  \sum \text{T})/( \text{\emph{Msp}I} / \text{\emph{Eco}RI}  \sum \text{G}/  \sum \text{T})]*100$$ In addition to the LUMA, a number of other methods have been developed to measure global DNA methylation (reviewed in \cite{c749}). Because repetitive elements represent between 30 to 50\% of the human genome 50 and contain CpG methylation in healthy cells \cite{c751}, measuring DNA methylation in repetitive elements such as LINE-1 and Alu sequences, centromere-adjacent satellite 2 (Sat2), and centromeric satellite alpha (Sat$\alpha$) DNA sequences has been used as a substitute for global DNA methylation \cite{c752}. The ``best'' method to use in a particular study will depend on the number of samples, quality and quantity of available DNA, desired coverage, and required resolution \cite{c749}.

\noindent As noted above, we have used LINE-1 DNA methylation as a surrogate for global DNA methylation in the NBS and POLIEP-follow up studies. Despite the prolific use of DNA methylation in repeat elements as a proxy for global DNA methylation, several recent reports have suggested that neither leukocyte methylation in LINE-1 nor Alu are good surrogates for global DNA methylation, as methylation levels in LINE-1 and Alu do not accurately represent global methylation levels \cite{c734,c753}. Global DNA methylation is the total genomic 5-mdC level and reflects DNA methylation not only in LINE-1 or other repeat elements but also in CpG islands and CpG sites located outside repeat elements. This could partially account for the differences in methylation between the two methods. 
 
\noindent In our study of persons at very high risk for developing colorectal cancer (GEOLynch cohort study), we have also collected sputum samples. There are various methods available to isolate and extract high quality DNA from sputum, but there is not yet evidence that global DNA methylation in sputum assessed using any method is related to colorectal carcinogenesis, whereas several studies have observed an association between global DNA methylation in leukocytes and colorectal development \cite{c754}. Furthermore, DNA methylation is likely to be tissue-specific \cite{c734,c755}, and because we had measured leukocyte global DNA methylation in our three other populations, measuring DNA methylation in leukocytes in the GEOLynch study would allow for better comparisons between our studies. 
 
\noindent We have measured DNA methylation using methods also widely employed in other studies, but as in all observational studies, measurement error is unavoidable. Random measurement error in LINE-1 methylation (outcome) will increase standard error of the estimates and consequently widen the corresponding confidence intervals making us less sure of our estimate \cite{c743,c744}. 
 
\section{Global \emph{vs}. gene-specific DNA methylation} % level 1 
\noindent As mentioned in Chapter 1, in addition to global DNA methylation in colorectal carcinogenesis, specific genes such as tumour-suppressor genes, can be silenced through DNA methylation at promoter regions of those genes \cite{c756,c757}. Hypermethylation of CpG islands in gene promoters results in inactivation of transcriptional activity of the gene \cite{c758}. Approximately 5\% of all genes are hypermethylated in colorectal cancer \cite{c759}. Using LINE-1 pyrosequencing or the LUMA method to measure global DNA methylation does not provide us with information about gene-specific DNA methylation. Based on our results, then, we cannot discount the influence of plasma folate, other B vitamins or methionine on gene-specific methylation as B vitamins have previously been associated with gene-specific methylation in colorectal adenomas \cite{c760} and carcinomas \cite{c761}. 
 
\noindent With this in mind, we have collected colorectal adenoma and carcinoma tissues of participants in our POLIEP-follow up study, which is comprised of persons with a history of colorectal adenomas and are therefore considered at high risk for colorectal cancer. The national network and registry of histopathology and cytopathology reports (Pathologisch Anatomisch Landelijk Geautomatiseerd Archief (PALGA)) \cite{c762} was used to identify and locate tumour tissues from the patients in our study. DNA was extracted from formalin-fixed, paraffin-embedded colorectal adenoma and carcinoma tissue using the Puregene\texttrademark~DNA isolation kit, and microdissection was performed, and only areas containing > 70\% tumour cells were used. We have data on gene-specific methylation of \emph{CACNA1G}, \emph{IGF2}, \emph{NEUROG1}, \emph{RUNX3}, and \emph{SOCS1} quantified using MethyLight \cite{c763}, and statistical analyses will be performed shortly. 
 
\section{Study populations: from low-risk to very high-risk} % level 1 
\noindent In this PhD project, we have utilized four different study populations in order to collectively investigate the role of plasma B vitamins as determinants of global DNA methylation along a continuum of colorectal cancer risk. Each of these four studies is unique in representing a specific risk population for colorectal cancer. The NBS included low-risk individuals in a population-based survey, the FACIT trial was comprised of medium-risk persons in a randomised-controlled trial, the POLIEP-follow-up of high-risk patients in a cohort study, and the GEOLynch cohort study consisted of those at very high risk for developing colorectal cancer. Methodological issues present in observational epidemiological studies merit further discussion. 
 
\section{Study design} % level 1 
\noindent We have used a cross-sectional analysis in two of our studies (chapters 3 and 5), a randomised-controlled trial in another (chapter 4), and our fourth study employed a cohort design (chapter 6). A well-known evidence-based pyramid describing the best research evidence places randomised-controlled trials above cohort studies and cohort studies above cross-sectional studies. Participants in our cohort study of MMR mutation carriers were identified through a national hereditary cancer registry, which registers families based on their family history of cancer in order to facilitate screening in these high risk families. However, not all patients and their families may be registered, particularly smaller families, and therefore would not represent all MMR mutation carriers in the Netherlands. It is unlikely, though, that identification and selection of registered MMR mutation carriers is related to B vitamin status, and would affect our results. While informative, the nature of cross-sectional studies cannot reveal whether LINE-1 methylation is a result of plasma folate concentrations, as the timing of events or causality cannot be inferred from this type of study. 
 
\noindent In all types of observational studies and small randomised trials, \textbf{confounding} may influence the observed associations. Confounding refers to the mixing of effects. When the effect of the exposure is mixed with the effect of another variable, this causes confounding. In order to be a confounder, the variable must be a risk factor for the outcome (in this thesis: global DNA methylation or risk of CRT) and also be associated with the exposure (in this thesis: usually plasma folate concentrations). Possible confounders can be identified \emph{a priori} based on literature and knowledge of mechanisms. Possible confounders can then be confirmed during statistical analyses by noting a change in risk estimate or regression coefficients after inclusion of the potential confounder into the model. In all our studies, we have included potential confounders if they changed the hazards ratio or regression coefficients by at least 10\% using manual backwards selection. Still, the presence of residual confounding due to improperly measured (measurement errors) or unmeasured variables cannot be excluded. 
 
\section[]{Effect modification} % level 1 
\noindent We have also explored possible effect modification by the \emph{MTHFR} C677T genotype, family history of colorectal cancer, alcohol intake, and smoking status in the associations between plasma B vitamins and methionine with global DNA methylation and between B vitamin and methionine intake and risk of colorectal tumours. 
 
\subsection{\emph{MTHFR} C677T genotype (rs1801133)} % level 2 
\noindent The common C to T transition in the \emph{MTHFR} gene at nucleotide 677 results in an alanine to valine substitution, subsequently producing a thermolabile variant and reduced enzyme activity \cite{c76}. This functional single nucleotide polymorphism (SNP) is possibly the most well-studied of all SNPs in one-carbon metabolism (OCM)-related genes. The TT genotype has been implicated with modifying associations between plasma folate or folate intake and global DNA methylation \cite{c75,c76,c712,c764,c765}. 
 
\noindent In our studies, the \emph{MTHFR} C677T genotype was not an effect modifier in the associations between folic acid supplementation and global DNA methylation in the medium risk group. In the high risk group, plasma folate was inversely associated with LINE-1 methylation in persons with the CC genotype while plasma riboflavin was inversely associated with LINE-1 methylation in persons with the TT genotype. In high risk groups, the effect of \emph{MTHFR} C667T genotype was unexpected. There was an inverse association between plasma folate and LINE-1 methylation for those with the CC genotype, and for those with the TT genotype, there was an inverse association between plasma riboflavin and LINE-1 methylation, although no interaction terms reached statistical significance. 
 
\subsection{Family history of colorectal cancer} % level 2 
\noindent The associations between B vitamin status and global DNA methylation could also be different for those with a family history of colorectal cancer compared to those without a family history. A history of colorectal cancer in a first degree relative has been associated with increased personal risk for sporadic colorectal cancer \cite{c766,c767}. The mechanisms involved in bringing about this increased risk are unknown, but a recent study did reveal that those with a positive colorectal cancer family history were more likely to develop LINE-1 methylation-low colorectal tumours compared to those without a family history of colorectal cancer \cite{c768}. Furthermore, in the pre-folic acid fortification era in the United States, Fuchs \emph{et al}. have observed a stronger inverse association between folate intake and colorectal cancer risk in women with a family history of colorectal cancer compared with those without a family history \cite{c769}. Known syndromes of familial clustering of colorectal cancer such as Lynch Syndrome, caused by a germline mutation in one of the DNA mismatch repair genes (\emph{MLH1}, \emph{MSH2}, \emph{MSH6}, \emph{PMS2}) \cite{c770}, constitute less than 5\% of all colorectal cancer \cite{c771}. The mechanisms associated with familial clustering of colorectal cancer in nonfamilial colorectal cancer (i.e. not the known familial colorectal cancer syndromes) are unknown, but genetic factors in combination with environmental factors likely play a role \cite{c772}. 
 
\noindent We have investigated family history of colorectal cancer as a possible effect modifier in the association between plasma B vitamins and leukocyte LINE-1 methylation in those previously diagnosed with colorectal adenomas. Family history of colorectal cancer in biological first-degree family members (i.e. parents, siblings, children) was assessed using a general questionnaire. Participants were asked for the number of siblings, and children they have and the age at which their biological parents, siblings, and children were first diagnosed with cancer and the type of cancer. For each participant, it was possible to include information about family history for up to and including 4 children and 4 siblings. 
 
\noindent Family history of cancers was self-reported using a questionnaire, a method often used in epidemiological studies \cite{c773,c774}, but whose precision and accuracy has been widely debated against medical records and cancer registries \cite{c775,c776,c777}. Patient-reported family cancer histories for first-degree relatives, however, seem to be precise and accurate for colorectal cancer according to results of an evidence-based analysis \cite{c778}. Our high risk cohort consisted exclusively of persons with a history of colorectal adenomas, and reporting of family history of colorectal cancer was similar between those with one adenoma and those with at least two adenomas. We have observed a positive association between plasma methionine and LINE-1 methylation but only in those with no family history of colorectal cancer. This relationship was not observed in those with a family history of colorectal cancer. 
 
\subsection{Alcohol and smoking} % level 2 
\noindent Alcohol interferes with folate absorption and folate-metabolising enzymes \cite{c779}, which could disturb one-carbon metabolism and limit methyl group availability for DNA methylation. Cigarette smoking has been shown to reduce systemic circulating concentrations of B vitamins by possibly increasing the activity of antioxidant defense enzymes in tissues, which require folate, B6 species, and riboflavin \cite{c780}, but smoking was not an effect modifier in the association between folate intake and global DNA methylation in a study of low risk for colorectal cancer older adults \cite{c79}. 
 
\noindent Smoking \cite{c781}, similar to other unfavourable lifestyle habits, as well as alcohol intake \cite{c782} are likely to be underreported. This underreporting of smoking could result in overreporting of former or never smokers in our studies. We have studied smoking using the categories never smoker, former smoker, and current smoker, and did not find smoking to be an effect modifier in the associations between plasma B vitamin concentrations and LINE-1 DNA methylation. In the randomised-controlled trial, participants were matched on smoking status. 
 
\noindent Underreporting of alcohol intake can also affect the ranking of participants \cite{c782}, but in our studies, the food frequency questionnaire used to estimate alcohol intake was validated against a 24-hour recall and showed high correlation for alcohol intake \cite{c783} as well for ranking subjects on alcoholic drinks \cite{c784}. We have stratified using tertiles of alcohol intake, but no effect modification by alcohol intake in the associations between plasma B vitamins and LINE-1 methylation and folic acid supplementation and global DNA methylation were observed. 
 
\subsection{Other potential effect modifiers} % level 2 
\noindent Although other genetic variants in OCM genes do exist (such as \emph{MTR} A2756G, \emph{MTRR} A66G, and another functional SNP \emph{MTHFR} A1298C in addition to others), few studies have investigated the relationships between these variants with B vitamin intake or biomarkers in healthy individuals and also in colorectal neoplasia (reviewed in \cite{c785}). This review concludes that the evidence on polymorphisms other than \emph{MTHFR} C677T in combination with folate and/or related nutrients and their relationship to colorectal cancer are tentative at best \cite{c785}. Therefore, while genetic variation in additional OCM-associated SNPs could theoretically in part affect our results, based on the current evidence, it does not seem likely that these SNPs would change our conclusions. 
 
\section[]{Public health implications} % level 1 
\noindent The overall findings of this thesis suggest that status of folate and other B vitamins may not be associated with global  DNA methylation nor LINE-1 DNA methylation in leukocytes except possibly in persons with a history of colorectal adenomas. 
 
\noindent The current recommendations for cancer prevention outlined by the World Cancer Research Fund/American Institute for Cancer Research in their 2007 Second Expert Report is based on systematic literature reviews of literature on food, nutrition, and physical activity \cite{c786}. The eight recommendations are to be as lean as possible, be physically active, limit consumption of energy-dense foods, eat plant-based foods, limit intake of red meat and avoid processed meat, limit alcohol intake, limit salt consumption and avoid mouldy grains and legumes, and aim to meet nutritional needs through diet alone (i.e. without the use of dietary supplements). Although research on food, nutrition, physical activity, and cancer survival is still in its infancy, these recommendations are suggested for cancer survivors as well \cite{c786}, and could help cancer survivors improve their quality of life \cite{c787}. 
 
\noindent The results from our studies support the current evidence-based recommendations set out by the WCRF/AICR, as they pertain to the status of folate and other B vitamins and colorectal cancer prevention. Because neither our results nor those of others provide evidence to suggest that a low or high folate status could influence leukocyte DNA methylation in individuals at differential risk for colorectal cancer, these results taken together would not support revisions to the current recommendations. It may be important to reiterate that based on our findings and in line with the WCRF/AICR recommendations, individuals should seek to meet nutritional needs through diet and not through dietary supplementation, for example, through the use of folic acid supplements. Given the age range of the populations we have studied, our studies do not provide data on the effect of folate status early or very late in life, and because epigenetic patterns are established \emph{in utero}, and maternal folate status and early life exposures could affect DNA methylation levels in offspring, as some studies have demonstrated 88, 89, B vitamins could similarly impact DNA methylation in early or later life differently. 
 
\section[]{Future research} % level 1 
\noindent The influence of one-carbon metabolism in connection with DNA methylation in the aetiology of colorectal cancer is complicated and intricate due to the complexity of these processes. Consequently, the task of unravelling this puzzle is daunting. There are, however, further steps that we can take to help improve our understanding of folate and DNA methylation in colorectal carcinogenesis. 
 
\noindent One-carbon metabolism is very complex involving interactions between a multitude of genetic and nutritional/environmental factors, so the use of pathway-based approaches that can simultaneously model many variables \cite{c790,c791,c792} would be beneficial. DNA methylation patterns in specific loci as well as in the whole genome in response to B vitamin intake and over time as well in different risk populations should be explored as well. An epigenome-wide association study (EWAS) could help us identify colorectal cancer related epigenome profiles. However, because DNA methylation patterns are not static but change over time and in response to environmental factors, it would be informative to perform an EWAS every 10 years starting in early life and follow these persons over a lifetime. Furthermore, in light of the continuous lively discussions about folic acid fortification in the Netherlands and the rest of Europe \cite{c793}, a long-term randomised controlled trial in healthy younger individuals could be conducted to investigate effects of folic acid fortification on DNA methylation and also DNA synthesis, as mathematical modelling and in vivo studies show DNA methylation competing with DNA synthesis pathways for folate species \cite{c794,c795}. Future research into this field is necessary to increase our understanding into the roles of B vitamins and DNA methylation in colorectal carcinogenesis and will be the foundation in developing additional effective colorectal cancer prevention strategies.

%--------------------------------------------------- TABLES GO HERE

% TABLE 7.1 HERE
\begin{sidewaystable}
\footnotesize
\caption{Summary of the studies presented in this thesis with folate or folic acid as an exposure.} 
\label{table7_1}
%\renewcommand{\arraystretch}{1.4}
%\begin{adjustbox}{width=18cm}
\begin{tabular}{L{1.3cm}C{1.5cm}C{2cm}C{2.1cm}C{2.1cm}C{2.1cm}C{4cm}}
\hline
\bfseries Colorectal cancer risk & \bfseries Name of study and chapter & \bfseries Study design & \bfseries Population & \bfseries Exposure & \bfseries Outcome & \bfseries Main findings\\
\hline
\parbox[t][5.2cm]{1.3cm}{Low} &
\parbox[t][5.2cm]{1.5cm}{\centering NBS (3)} &
\parbox[t][5.2cm]{2cm}{\centering Cross-sectional analysis of a population-based survey} &
\parbox[t][5.2cm]{2.1cm}{\centering Healthy men and women without a personal history of cancer aged 50-70 (n=275)} &
\parbox[t][5.2cm]{2.1cm}{\centering Extremes of plasma folate (10\textsuperscript{th} and 90\textsuperscript{th} percentiles)} &
\parbox[t][5.2cm]{2.1cm}{\centering LINE-1 methylation in peripheral blood leukocytes} & 
\parbox[t][5.2cm]{4cm}{ \centering - No difference in LINE-1 methylation between the 10\textsuperscript{th} percentile and the 90\textsuperscript{th} percentile of plasma folate\\
- LINE-1 DNA methylation lower in females than in males\\
- \emph{MTHFR} C677T genotype not associated with LINE-1 methylation\\
- Age inversely associated with LINE-1 methylation independent of plasma folate}\\

\parbox[t][5.2cm]{1.3cm}{Medium} &
\parbox[t][5.2cm]{1.5cm}{\centering FACIT (4)} &
\parbox[t][5.2cm]{2cm}{\centering Randomised placebo-controlled trial} &
\parbox[t][5.2cm]{2.1cm}{\centering Men and post-menopausal women aged 50-70 years with moderately elevated homocysteine (n=216)} &
\parbox[t][5.2cm]{2.1cm}{\centering 800 $\mu$g folic acid or placebo daily for three years} &
\parbox[t][5.2cm]{2.1cm}{\centering Global DNA methylation in peripheral blood leukocytes} &
\parbox[t][5.2cm]{4cm}{ \centering - No difference in global DNA methylation between the folic acid and the placebo group \\ - No difference in DNA methylation between treatment groups after stratification by \textit{MTHFR} C677T genotype, baseline erythrocyte folate concentrations, baseline DNA methylation levels, age, current smoking status, alcohol intake, body mass index, or sex}\\
\end{tabular}
%\end{adjustbox}
\end{sidewaystable}


% TABLE 7.1 PART 1 CONTINUED
\begin{sidewaystable}
\footnotesize
\caption*{\textbf{Table 7.1} {Summary of the studies presented in this thesis with folate or folic acid as an exposure.} \emph{(continued)}}
\label{table7_1}
%\begin{adjustbox}{width=18cm}
\begin{tabular}{L{1.3cm}C{1.5cm}C{2cm}C{2.1cm}C{2.1cm}C{2.1cm}C{4cm}}
\hline
\bfseries Colorectal cancer risk & \bfseries Name of study and chapter & \bfseries Study design & \bfseries Population & \bfseries Exposure & \bfseries Outcome & \bfseries Main findings\\
\hline
\parbox[t][5cm]{1.3cm}{High} &
\parbox[t][5cm]{1.5cm}{\centering POLIEP-follow-up (5)} &
\parbox[t][5cm]{2cm}{\centering Cross-sectional analysis of a cohort study} &
\parbox[t][5cm]{2.1cm}{\centering Men and women aged 18-75 years with at least one histologically-confirmed colorectal adenoma ever in their life (n=281)} &
\parbox[t][5cm]{2.1cm}{\centering Plasma folate (5-methyltetra-hydrofolate)} &
\parbox[t][5cm]{2.1cm}{\centering LINE-1 methylation in peripheral blood leukocytes} &
\parbox[t][5cm]{4cm}{\centering - Overall inverse association between plasma folate and LINE-1 methylation \\ - In those with \emph{MTHFR} 677CC genotype, plasma folate inversely associated with LINE-1 methylation \\ - In those with a family history of colorectal cancer, plasma folate inversely associated (borderline) with LINE-1 methylation}\\

\parbox[t][2.3cm]{1.3cm}{Very high} &
\parbox[t][2.3cm]{1.5cm}{\centering Lynch (6)} &
\parbox[t][2.3cm]{2cm}{\centering Cohort study} &
\parbox[t][2.3cm]{2.1cm}{\centering MMR gene mutation carriers aged 18-80 years (n=470)} &
\parbox[t][2.3cm]{2.1cm}{\centering Dietary folate intake} &
\parbox[t][2.3cm]{2.1cm}{\centering Colorectal adenoma and carcinoma occurrence} &
\parbox[t][2.3cm]{4cm}{ \centering - No association between dietary folate intake and colorectal tumour risk during a median person-time of 28.0 months}\\
\hline
\end{tabular}
%\end{adjustbox}
\end{sidewaystable}


% TABLE 7.2 HERE 
\begin{sidewaystable}
\footnotesize
%\begin{table}
\caption{Summary of the studies presented in this thesis with vitamin B2, vitamin B6, vitamin B12, and methionine as exposures.} 
\label{table7_2}
%\begin{adjustbox}{width=\textwidth,height=6cm}
%\begin{tabularx}{\textwidth}{XXXXXXX}
\begin{tabular}{L{1.3cm}C{1.5cm}C{2cm}C{2.1cm}C{2.1cm}C{2.1cm}C{4cm}}
\hline\bfseries Colorectal cancer risk & \bfseries Name of study and chapter & \bfseries Study design & \bfseries Population & \bfseries Exposure & \bfseries Outcome & \bfseries Main findings\\
\hline
\parbox[t][4cm]{1.3cm}{High} &
\parbox[t][4cm]{1.5cm}{\centering POLIEP-follow-up (5)} &
\parbox[t][4cm]{2cm}{\centering Cross-sectional analysis of a cohort study} &
\parbox[t][4cm]{2.1cm}{\centering Men and women aged 18-75 years with at least one histologically-confirmed colorectal adenoma ever in their life (n=281)} &
\parbox[t][4cm]{2.1cm}{\centering Plasma riboflavin, PLP, PL, PA, cobalamin, methionine, MMA} &
\parbox[t][4cm]{2.1cm}{\centering LINE-1 methylation in peripheral blood leukocytes} &
\parbox[t][4cm]{4cm}{\centering - In those without a family history of colorectal cancer, positive association between plasma methionine and LINE-1 methylation \\ - In those with \textit{MTHFR} 677TT genotype, inverse association between plasma riboflavin and LINE-1 methylation}\\

\parbox[t][4cm]{1.3cm}{Very high} &
\parbox[t][4cm]{1.5cm}{\centering Lynch (6)} &
\parbox[t][4cm]{2cm}{\centering Cohort study} &
\parbox[t][4cm]{2.1cm}{\centering MMR gene mutation carriers aged 18-80 years (n=470)} &
\parbox[t][4cm]{2.1cm}{\centering Dietary intake of vitamin B2, vitamin B6, vitamin B12, and methionine} &
\parbox[t][4cm]{2.1cm}{\centering Colorectal adenoma and carcinoma occurrence} &
\parbox[t][4cm]{4cm}{\centering - No association between dietary intake of any B vitamin nor methionine and colorectal tumour risk during a median person-time of 28.0 months \\ - No significant interactions between intake of B vitamins and methionine with \textit{MTHFR} C677T genotype}\\
\hline
%\end{tabularx}
\end{tabular}
%\end{adjustbox}
%\end{table}
\end{sidewaystable}


% TABLE 7.3 HERE
\begin{sidewaystable}
\small
\caption{Summary of studies: effect of folic acid supplementation on leukocyte global DNA methylation in low and medium risk populations.} 
\label{table7_3}
\begin{adjustbox}{width=\textwidth}
%\begin{tabularx}{\textwidth}{XXXXXC{5cm}XX}
\begin{tabular}{L{1.8cm}C{1.6cm}C{1.6cm}C{1.8cm}C{1.3cm}C{3.2cm}C{1.5cm}C{1.8cm}} 
\hline
\bfseries Study & \bfseries Number of participants & \bfseries Participant age range & \bfseries Daily dose & \bfseries Duration & \bfseries DNA methylation assessment method & \bfseries Country & \bfseries Treatment effect\\
\hline
\parbox[t][2.2cm]{1.8cm}{\raggedright Fenech \textit{et al}. 1998 \cite{c796}} &
\parbox[t][2.2cm]{1.6cm}{\centering 63 volunteers} &
\parbox[t][2.2cm]{1.6cm}{\centering 18-32 years} &
\parbox[t][2.2cm]{1.8cm}{\centering 2000 $\mu$g folic acid and 20 $\mu$g vitamin B12} &
\parbox[t][2.2cm]{1.3cm}{\centering 12 weeks} &
\parbox[t][2.2cm]{3.2cm}{\centering Methyl acceptance assay} &
\parbox[t][2.2cm]{1.5cm}{\centering Australia} &
\parbox[t][2.2cm]{1.8cm}{\centering No effect}\\

\parbox[t][1.5cm]{1.8cm}{\raggedright Basten \textit{et al}. 2006 \cite{c797}} &
\parbox[t][1.5cm]{1.6cm}{\centering 61 volunteers} &
\parbox[t][1.5cm]{1.6cm}{\centering 20-60 years} &
\parbox[t][1.5cm]{1.8cm}{\centering 1200 $\mu$g folic acid} &
\parbox[t][1.5cm]{1.3cm}{\centering 12 weeks} &
\parbox[t][1.5cm]{3.2cm}{\centering Methyl acceptance assay} &
\parbox[t][1.5cm]{1.5cm}{\centering United Kingdom} &
\parbox[t][1.5cm]{1.8cm}{\centering No effect}\\

\parbox[t][2.1cm]{1.8cm}{\raggedright Jung \emph{et al}. 2011 \cite{c798} Chapter 4} &
\parbox[t][2.1cm]{1.6cm}{\centering 216 volunteers} &
\parbox[t][2.1cm]{1.6cm}{\centering 50-70 years} &
\parbox[t][2.1cm]{1.8cm}{\centering 800 $\mu$g folic acid} &
\parbox[t][2.1cm]{1.3cm}{\centering 3 years} &
\parbox[t][2.1cm]{3.2cm}{\centering Liquid chromatography electron spray tandem mass spectrometry (LC-ES MS/MS)} &
\parbox[t][2.1cm]{1.5cm}{\centering Netherlands} &
\parbox[t][2.1cm]{1.8cm}{\centering No effect}\\
\hline
%\end{tabularx}
\end{tabular}
\end{adjustbox}
\end{sidewaystable}


% TABLE 7.4 HERE 
\begin{sidewaystable}
\caption{Summary of studies: effect of controlled folate and folic acid intake on leukocyte global DNA methylation in low risk populations.}
\label{table7_4}
\begin{adjustbox}{width=18cm}
\begin{tabular}{L{2.1cm}C{2cm}C{2.3cm}C{3.5cm}C{2.3cm}C{1.8cm}C{6.1cm}}
\hline\bfseries Study & \bfseries Number of participants & \bfseries Participant age range & \bfseries Daily dose and duration & \bfseries DNA methylation assessment
method & \bfseries Country & \bfseries Treatment effect\\
\hline

\parbox[t][4.3cm]{2.1cm}{\raggedright Jacob {\textit{et al}}{. 1998 \cite{c73}}} &
\parbox[t][4.3cm]{2cm}{\centering 8 post-menopausal women} &
\parbox[t][4.3cm]{2.3cm}{\centering 49-63 years} &
\parbox[t][4.3cm]{3.5cm}{\centering day 1-5 195 $\mu$g/day folate { day 6-41 56 $\mu$g/day folate}{ day 42-69 111 $\mu$g/day folate} { day 70-80 286 $\mu$g/day folate} day 81-91 516 $\mu$g/day folate} &
\parbox[t][4.3cm]{2.3cm}{\centering Methyl acceptance assay} &
\parbox[t][4.3cm]{1.8cm}{\centering United States of America} &
\parbox[t][4.3cm]{6.1cm}{\centering - Following folate depletion, increased \textsuperscript{3}Hmethyl incorporation \\ - Following repletion (286-516 $\mu$g/day folate), decreased \textsuperscript{3}Hmethyl incorporation}\\

\parbox[t][2.8cm]{2.1cm}{\raggedright Rampersaud \emph{et al}. 2000 \cite{c74}} &
\parbox[t][2.8cm]{2cm}{\centering 30 post-menopausal women} &
\parbox[t][2.8cm]{2.3cm}{\centering 60-85 years} &
\parbox[t][2.8cm]{3.5cm}{\centering 7 weeks 118 $\mu$g folate/day then 7 weeks 200 or 415 $\mu$g folate/day} &
\parbox[t][2.8cm]{2.3cm}{\centering Methyl acceptance assay} &
\parbox[t][2.8cm]{1.8cm}{\centering United States of America} &
\parbox[t][2.8cm]{6.1cm}{\centering - Following folate depletion, increase in \textsuperscript{3}Hmethyl incorporation in all subjects (both 200 and 415 $\mu$g folate/day groups), but no difference between groups \\ - Following folate repletion, no change in \textsuperscript{3}Hmethyl incorporation within nor between 200 and 415 $\mu$g folate/day groups}\\
\end{tabular}
\end{adjustbox}
\end{sidewaystable}


% TABLE 7.4 CONTINUED 
\begin{sidewaystable}
\footnotesize
\caption*{\textbf{Table 7.4.} Summary of studies: effect of controlled folate and folic acid intake on leukocyte global DNA methylation in low risk populations. \emph{(continued)}}
%\begin{adjustbox}
\begin{tabular}{L{1.6cm}C{2cm}C{1.7cm}C{2.3cm}C{2.5cm}C{1.6cm}C{3.4cm}}
\hline\bfseries Study & \bfseries Number of participants & \bfseries Participant age range & \bfseries Daily dose and duration & \bfseries DNA methylation assessment
method & \bfseries Country & \bfseries Treatment effect\\
\hline
\parbox[t][3.3cm]{1.6cm}{\raggedright Shelnutt \emph{et al}. 2004 \cite{c75}} &
\parbox[t][3.3cm]{2cm}{\centering 20 young women with \textit{MTHFR} 677CC or TT genotype} &
\parbox[t][3.3cm]{1.7cm}{\centering 20-30 years} &
\parbox[t][3.3cm]{2.3cm}{\centering 7 weeks 115 \textrm{${\pm}$}20 $\mu$g DFE/day then 7 weeks 400 $\mu$g DFE/day} &
\parbox[t][3.3cm]{2.5cm}{\centering Methyl acceptance assay} &
\parbox[t][3.3cm]{1.6cm}{\centering United States of America} &
\parbox[t][3.3cm]{3.4cm}{\centering No effect}\\

\parbox[t][3.2cm]{1.6cm}{\raggedright Shelnutt \emph{et al}. 2004 \cite{c75}} &
\parbox[t][3.2cm]{2cm}{\centering 20 young women with \textit{MTHFR} 677CC or TT genotype} &
\parbox[t][3.2cm]{1.7cm}{\centering 20-30 years} &
\parbox[t][3.2cm]{2.3cm}{\centering 7 weeks 115\textrm{${\pm}$}20 $\mu$g DFE/day then 7 weeks 400 $\mu$g DFE/day} &
\parbox[t][3.2cm]{2.5cm}{\centering Liquid chromatography tandem mass spectrometry} &
\parbox[t][3.2cm]{1.6cm}{\centering United States of America} &
\parbox[t][3.2cm]{3.4cm}{\centering - \emph{MTHFR} 677CC: No effect of depletion nor repletion \\ - \emph{MTHFR} 677TT:  Increase in mCyt/tCyt ratio following repletion}\\

\parbox[t][2.3cm]{1.6cm}{\raggedright Axume \emph{et al}. 2007 \cite{c799}} &
\parbox[t][2.3cm]{2cm}{\centering 28 women with \textit{MTHFR} 677CC genotype} &
\parbox[t][2.3cm]{1.7cm}{\centering 18-44 years} &
\parbox[t][2.3cm]{2.3cm}{\centering 7 weeks 135 $\mu$g DFE/day then 7 weeks 400 or 800 $\mu$g DFE/day} &
\parbox[t][2.3cm]{2.5cm}{\centering Cytosine extension assay} &
\parbox[t][2.3cm]{1.6cm}{\centering United States of America} &
\parbox[t][2.3cm]{3.4cm}{\centering No effect}\\
\hline
\end{tabular}
%\end{adjustbox}
\end{sidewaystable}


% TABLE 7.5 HERE 
\begin{center} 
\begin{sidewaystable}
\small
\caption{Summary of studies: association between blood folate or estimated folate intake (exposure) and leukocyte global DNA methylation (outcome) in low risk populations.} 
\label{table7_5}
\begin{tabular}{L{1.8cm}C{1.8cm}C{2cm}C{2.2cm}C{2.7cm}C{1.8cm}C{2.5cm}} 
\hline
~ & ~ & ~ & ~ & ~ & ~ & ~\\
\bfseries Study & \bfseries Exposure & \bfseries Study design & \bfseries Population & \bfseries DNA methylation assessment method & \bfseries Country & \bfseries Association\\
~ & ~ & ~ & ~ & ~ & ~ & ~\\
\hline
\parbox[t][3.4cm]{1.8cm}{\raggedright Stern \textit{et al}. 2000 \cite{c712}} &
\parbox[t][3.4cm]{1.8cm}{\centering Red blood cell folate} &
\parbox[t][3.4cm]{2cm}{\centering Cross-sectional} &
\parbox[t][3.4cm]{2.2cm}{\centering 10 volunteers with \textit{MTHFR} 677CC or TT genotype aged 25-75} &
\parbox[t][3.4cm]{2.7cm}{\centering Methyl acceptance assay} &
\parbox[t][3.4cm]{1.8cm}{\centering United States of America} &
\parbox[t][3.4cm]{2.5cm}{\centering Positive association (more folate, more methylation) in \textit{MTHFR} 677TT persons}\\

\parbox[t][2.5cm]{1.8cm}{\raggedright Friso \textit{et al}. 2002 \cite{c76}} &
\parbox[t][2.5cm]{1.8cm}{\centering Plasma folate} &
\parbox[t][2.5cm]{2cm}{\centering Cross-sectional} &
\parbox[t][2.5cm]{2.2cm}{\centering 292 volunteers with \textit{MTHFR} 677CC or TT genotype} &
\parbox[t][2.5cm]{2.7cm}{\centering Liquid chromatography mass spectrometry} &
\parbox[t][2.5cm]{1.8cm}{\centering Italy} &
\parbox[t][2.5cm]{2.5cm}{\centering Positive association in \textit{MTHFR} 677TT persons}\\

\parbox[t][1.5cm]{1.8cm}{\raggedright Kok \textit{et al}. 2007 \cite{c78}} &
\parbox[t][1.5cm]{1.8cm}{\centering Plasma folate} &
\parbox[t][1.5cm]{2cm}{\centering Cross-sectional} &
\parbox[t][1.5cm]{2.2cm}{\centering { 109 healthy volunteers} aged 18-62} &
\parbox[t][1.5cm]{2.7cm}{\centering LC-ES MS/MS} &
\parbox[t][1.5cm]{1.8cm}{\centering Netherlands} &
\parbox[t][1.5cm]{2.5cm}{\centering No association}\\
\hline
\end{tabular}
\end{sidewaystable}
\end{center}

% TABLE 7.5 CONTINUED 
\begin{center}
\begin{sidewaystable}
\small
\caption*{\textbf{Table 7.5} Summary of studies: association between blood folate or estimated folate intake (exposure) and leukocyte global DNA methylation (outcome) in low risk populations. \emph{(continued)}} 
%\begin{adjustbox}{width=\textwidth}
\begin{tabular}{L{1.8cm}C{1.8cm}C{2cm}L{2.2cm}C{2.7cm}C{1.8cm}C{2.5cm}}
\hline
~ & ~ & ~ & ~ & ~ & ~ & ~\\
\bfseries Study & \bfseries Exposure & \bfseries Study design & \bfseries Population & \bfseries DNA methylation assessment method & \bfseries Country & \bfseries Association\\
~ & ~ & ~ & ~ & ~ & ~ & ~\\
\hline

\parbox[t][2cm]{1.8cm}{\raggedright Ono \textit{et al}. 2012 \cite{c710}} &
\parbox[t][2cm]{1.8cm}{\centering Dietary folate intake} &
\parbox[t][2cm]{2cm}{\centering Cross-sectional} &
\parbox[t][2cm]{2.2cm}{\centering 384 female volunteers aged 20-74} &
\parbox[t][2cm]{2.7cm}{\centering Luminometric methylation assay (LUMA)} &
\parbox[t][2cm]{1.8cm}{\centering Japan} &
\parbox[t][2cm]{2.5cm}{\centering Inverse association}\\

\parbox[t][2cm]{1.8cm}{\raggedright Gomes \textit{et al}. 2012 \cite{c79}} &
\parbox[t][2cm]{1.8cm}{\centering Dietary folate intake} &
\parbox[t][2cm]{2cm}{\centering Cross-sectional} &
\parbox[t][2cm]{2.2cm}{\centering 126 volunteers aged 60-88} &
\parbox[t][2cm]{2.7cm}{\centering Imprint Methylated DNA Quantification} &
\parbox[t][2cm]{1.8cm}{\centering Brazil} &
\parbox[t][2cm]{2.5cm}{\centering No association}\\

\parbox[t][2cm]{1.8cm}{\raggedright Zhang \textit{et al}. 2012 \cite{c711}} &
\parbox[t][2cm]{1.8cm}{\centering Dietary folate intake} &
\parbox[t][2cm]{2cm}{\centering Cross-sectional} &
\parbox[t][2cm]{2.2cm}{\centering 165 volunteers aged 18-78} &
\parbox[t][2cm]{2.7cm}{\centering Bisulfite-PCR and pyrosequencing of LINE-1} &
\parbox[t][2cm]{1.8cm}{\centering United States of America} &
\parbox[t][2cm]{2.5cm}{\centering No association}\\

\parbox[t][1.7cm]{1.8cm}{\raggedright Jung \textit{et al}. submitted \cite{c77} Chapter 3} &
\parbox[t][1.7cm]{1.8cm}{\centering Plasma folate} &
\parbox[t][1.7cm]{1.8cm}{\centering Cross-sectional} &
\parbox[t][1.7cm]{2.2cm}{\centering 275 healthy volunteers aged 50-70} &
\parbox[t][1.7cm]{2.7cm}{\centering Bisulfite-PCR and pyrosequencing of LINE-1} &
\parbox[t][1.7cm]{1.8cm}{\centering Netherlands} &
\parbox[t][1.7cm]{2.5cm}{\centering No association}\\
\hline
\end{tabular}
%\end{adjustbox}
\end{sidewaystable}
\end{center}



% TABLE 7.6 HERE
\begin{sidewaystable}
\small
\caption{Summary of studies: association between folate (exposure) and leukocyte DNA methylation (outcome) in high risk populations.} 
\label{table7_6}
%\begin{adjustbox}{width=\textwidth}
\renewcommand{\arraystretch}{1}
\begin{tabular}{L{2cm}C{2cm}C{2cm}C{2cm}C{2.5cm}C{1.5cm}C{3cm}} 
\hline\bfseries Study & \bfseries Exposure & \bfseries Study design & \bfseries Population & \bfseries DNA methylation assessment 
method & \bfseries Country & \bfseries Association\\
\hline

\parbox[t][3.6cm]{2cm}{\raggedright Pufulete \textit{et al}. 2005 \cite{c713}} &
\parbox[t][3.6cm]{2cm}{\centering Folic acid supplementation (400 $\mu$g folic acid/day \textit{vs}. placebo)} &
\parbox[t][3.6cm]{2cm}{\centering Randomised, double-blind, placebo-controlled trial} &
\parbox[t][3.6cm]{2cm}{\centering 31 CRA patients (n=15 folic acid, n=16 placebo)} &
\parbox[t][3.6cm]{2.5cm}{\centering Methyl acceptance assay} &
\parbox[t][3.6cm]{1.5cm}{\centering United Kingdom} &
\parbox[t][3.6cm]{3cm}{\centering Compared with the placebo group, folic acid supplementation resulted in increased leukocyte DNA methylation}\\

\parbox[t][2.1cm]{2cm}{\raggedright Jung \emph{et al}. 2013 \cite{c714} Chapter 5} &
\parbox[t][2.1cm]{2cm}{\centering Plasma folate} &
\parbox[t][2.1cm]{2cm}{\centering Cross-sectional} &
\parbox[t][2.1cm]{2cm}{\centering 281 CRA patients} &
\parbox[t][2.1cm]{2.5cm}{\centering Bisulfite-PCR and pyrosequencing of LINE-1} &
\parbox[t][2.1cm]{1.5cm}{\centering Netherlands} &
\parbox[t][2.1cm]{3cm}{\centering Inverse association between plasma folate and LINE-1 methylation}\\
\hline
\end{tabular}
%\end{adjustbox}
\end{sidewaystable}

\begin{thebibliography}{12}
	\bibitem{c71}	Balaghi M, Wagner C. DNA methylation in folate deficiency: use of CpG methylase. Biochemical and biophysical research communications. 1993 Jun 30;193(3):1184-90.
	\bibitem{c72}	Food standards: amendment of standards of identity for enriched grain products to require addition of folic acid.  Federal Register 1996:8781-97.
	\bibitem{c73}	Jacob RA, Gretz DM, Taylor PC, James SJ, Pogribny IP, Miller BJ, et al. Moderate folate depletion increases plasma homocysteine and decreases lymphocyte DNA methylation in postmenopausal women. The Journal of nutrition. 1998 Jul;128(7):1204-12.
	\bibitem{c74}	Rampersaud GC, Kauwell GP, Hutson AD, Cerda JJ, Bailey LB. Genomic DNA methylation decreases in response to moderate folate depletion in elderly women. The American journal of clinical nutrition. 2000 Oct;72(4):998-1003.
	\bibitem{c75}	Shelnutt KP, Kauwell GP, Gregory JF, 3rd, Maneval DR, Quinlivan EP, Theriaque DW, et al. Methylenetetrahydrofolate reductase 677C$\rightarrow$T polymorphism affects DNA methylation in response to controlled folate intake in young women. The Journal of nutritional biochemistry. 2004 Sep;15(9):554-60.
	\bibitem{c76}	Friso S, Choi SW, Girelli D, Mason JB, Dolnikowski GG, Bagley PJ, et al. A common mutation in the 5,10-methylenetetrahydrofolate reductase gene affects genomic DNA methylation through an interaction with folate status. Proceedings of the National Academy of Sciences of the United States of America. 2002 Apr 16;99(8):5606-11.
	\bibitem{c77}	Jung AY, den Heijer M, Lute C, Kiemeney LALM, Ueland PM, Midttun O, et al. Age and sex but not plasma folate nor MTHFR C677T are associated with leukoyte LINE-1 methylation in a healthy population.
	\bibitem{c78}	Kok RM, Smith DE, Barto R, Spijkerman AM, Teerlink T, Gellekink HJ, et al. Global DNA methylation measured by liquid chromatography-tandem mass spectrometry: analytical technique, reference values and determinants in healthy subjects. Clin Chem Lab Med. 2007;45(7):903-11.
	\bibitem{c79}	Gomes MV, Toffoli LV, Arruda DW, Soldera LM, Pelosi GG, Neves-Souza RD, et al. Age-related changes in the global DNA methylation profile of leukocytes are linked to nutrition but are not associated with the MTHFR C677T genotype or to functional capacities. PloS one. 2012;7(12):e52570.
	\bibitem{c710}	Ono H, Iwasaki M, Kuchiba A, Kasuga Y, Yokoyama S, Onuma H, et al. Association of dietary and genetic factors related to one-carbon metabolism with global methylation level of leukocyte DNA. Cancer science. 2012 Dec;103(12):2159-64.
	\bibitem{c711}	Zhang FF, Santella RM, Wolff M, Kappil MA, Markowitz SB, Morabia A. White blood cell global methylation and IL-6 promoter methylation in association with diet and lifestyle risk factors in a cancer-free population. Epigenetics. 2012 Jun 1;7(6):606-14.
	\bibitem{c712}	Stern LL, Mason JB, Selhub J, Choi SW. Genomic DNA hypomethylation, a characteristic of most cancers, is present in peripheral leukocytes of individuals who are homozygous for the C677T polymorphism in the methylenetetrahydrofolate reductase gene. Cancer Epidemiol Biomarkers Prev. 2000 Aug;9(8):849-53.
	\bibitem{c713}	Pufulete M, Al-Ghnaniem R, Khushal A, Appleby P, Harris N, Gout S, et al. Effect of folic acid supplementation on genomic DNA methylation in patients with colorectal adenoma. Gut. 2005 May;54(5):648-53.
	\bibitem{c714}	Jung AY, Botma A, Lute C, Blom HJ, Ueland PM, Kvalheim G, et al. Plasma B vitamins and LINE-1 DNA methylation in leukocytes of patients with a history of colorectal adenomas. Molecular nutrition \& food research. 2013 Apr;57(4):698-708.
	\bibitem{c715}	Pande M, Chen J, Amos CI, Lynch PM, Broaddus R, Frazier ML. Influence of methylenetetrahydrofolate reductase gene polymorphisms C677T and A1298C on age-associated risk for colorectal cancer in a caucasian lynch syndrome population. Cancer Epidemiol Biomarkers Prev. 2007 Sep;16(9):1753-9.
	\bibitem{c716}	Reeves SG, Meldrum C, Groombridge C, Spigelman AD, Suchy J, Kurzawski G, et al. MTHFR 677 C>T and 1298 A>C polymorphisms and the age of onset of colorectal cancer in hereditary nonpolyposis colorectal cancer. Eur J Hum Genet. 2009 May;17(5):629-35.
	\bibitem{c717}	Verkleij-Hagoort AC, de Vries JH, Stegers MP, Lindemans J, Ursem NT, Steegers-Theunissen RP. Validation of the assessment of folate and vitamin B12 intake in women of reproductive age: the method of triads. European journal of clinical nutrition. 2007 May;61(5):610-5.
	\bibitem{c718}	Feunekes GI, Van Staveren WA, De Vries JH, Burema J, Hautvast JG. Relative and biomarker-based validity of a food-frequency questionnaire estimating intake of fats and cholesterol. The American journal of clinical nutrition. 1993 Oct;58(4):489-96.
	\bibitem{c719}	Suitor CW, Bailey LB. Dietary folate equivalents: interpretation and application. Journal of the American Dietetic Association. 2000 Jan;100(1):88-94.
	\bibitem{c720}	van Rossum CTM, Fransen HP, Verkaik-Kloosterman J, Buurma-Rethans EJM, Ock MC. Dutch National Food Consumption Survey 2007-2010. In: Environment NIfPHat, ed. 2011.
	\bibitem{c721}	Johansson I, Van Guelpen B, Hultdin J, Johansson M, Hallmans G, Stattin P. Validity of food frequency questionnaire estimated intakes of folate and other B vitamins in a region without folic acid fortification. European journal of clinical nutrition. 2010 Aug;64(8):905-13.
	\bibitem{c722}	Pufulete M, Emery PW, Nelson M, Sanders TA. Validation of a short food frequency questionnaire to assess folate intake. The British journal of nutrition. 2002 Apr;87(4):383-90.
	\bibitem{c723}	Colic Baric I, Satalic Z, Keser I, Cecic I, Sucic M. Validation of the folate food frequency questionnaire with serum and erythrocyte folate and plasma homocysteine. International journal of food sciences and nutrition. 2009;60 Suppl 5:10-8.
	\bibitem{c724}	Willett W. Nutritional Epidemiology. 2nd ed: Oxford University Press 1998.
	\bibitem{c725}	Subar AF, Kipnis V, Troiano RP, Midthune D, Schoeller DA, Bingham S, et al. Using intake biomarkers to evaluate the extent of dietary misreporting in a large sample of adults: the OPEN study. American journal of epidemiology. 2003 Jul 1;158(1):1-13.
	\bibitem{c726}	Prentice RL, Mossavar-Rahmani Y, Huang Y, Van Horn L, Beresford SA, Caan B, et al. Evaluation and comparison of food records, recalls, and frequencies for energy and protein assessment by using recovery biomarkers. American journal of epidemiology. 2011 Sep 1;174(5):591-603.
	\bibitem{c727}	Thompson FE, Subar AF. Nutrition in the Prevention and Treatment of Disease. 3 ed: Elsevier 2013.
	\bibitem{c728}	Wacholder S. When measurement errors correlate with truth: surprising effects of nondifferential misclassification. Epidemiology (Cambridge, Mass. 1995 Mar;6(2):157-61.
	\bibitem{c729}	Freedman LS, Schatzkin A, Midthune D, Kipnis V. Dealing with dietary measurement error in nutritional cohort studies. Journal of the National Cancer Institute. 2011 Jul 20;103(14):1086-92.
	\bibitem{c730}	Drogan D, Klipstein-Grobusch K, Wans S, Luley C, Boeing H, Dierkes J. Plasma folate as marker of folate status in epidemiological studies: the European Investigation into Cancer and Nutrition (EPIC)-Potsdam study. The British journal of nutrition. 2004 Sep;92(3):489-96.
	\bibitem{c731}	Fayet F, Flood V, Petocz P, Samman S. Relative and biomarker-based validity of a food frequency questionnaire that measures the intakes of vitamin B(12), folate, iron, and zinc in young women. Nutrition research (New York, NY. 2011 Jan;31(1):14-20.
	\bibitem{c732}	Cope EL, Shrubsole MJ, Cohen SS, Cai Q, Wu J, Ueland PM, et al. Intraindividual variation in one-carbon metabolism plasma biomarkers. Cancer Epidemiol Biomarkers Prev. 2013 Oct;22(10):1894-9.
	\bibitem{c733}	Molloy AM, Scott JM. Microbiological assay for serum, plasma, and red cell folate using cryopreserved, microtiter plate method. Methods in enzymology. 1997;281:43-53.
	\bibitem{c734}	Liu J, Hesson LB, Meagher AP, Bourke MJ, Hawkins NJ, Rand KN, et al. Relative distribution of folate species is associated with global DNA methylation in human colorectal mucosa. Cancer prevention research (Philadelphia, Pa. 2012 Jul;5(7):921-9.
	\bibitem{c735}	Kim YI, Fawaz K, Knox T, Lee YM, Norton R, Libby E, et al. Colonic mucosal concentrations of folate are accurately predicted by blood measurements of folate status among individuals ingesting physiologic quantities of folate. Cancer Epidemiol Biomarkers Prev. 2001 Jun;10(6):715-9.
	\bibitem{c736}	Kim YI, Fawaz K, Knox T, Lee YM, Norton R, Arora S, et al. Colonic mucosal concentrations of folate correlate well with blood measurements of folate status in persons with colorectal polyps. The American journal of clinical nutrition. 1998 Oct;68(4):866-72.
	\bibitem{c737}	Williams EA, Welfare M, Spiers A, Hill MH, Bal W, Gibney ER, et al. Systemic folate status, rectal mucosal folate concentration and dietary intake in patients at differential risk of bowel cancer (The FAB2 Study). European journal of nutrition. 2013 Oct;52(7):1801-10.
	\bibitem{c738}	Meenan J, O'Hallinan E, Lynch S, Molloy A, McPartlan J, Scott J, et al. Folate status of gastrointestinal epithelial cells is not predicted by serum and red cell folate values in replete subjects. Gut. 1996 Mar;38(3):410-3.
	\bibitem{c739}	Meenan J, O'Hallinan E, Scott J, Weir DG. Epithelial cell folate depletion occurs in neoplastic but not adjacent normal colon mucosa. Gastroenterology. 1997 Apr;112(4):1163-8.
	\bibitem{c740}	Ocke MC, Schrijver J, Obermann-de Boer GL, Bloemberg BP, Haenen GR, Kromhout D. Stability of blood (pro)vitamins during four years of storage at -20 degrees C: consequences for epidemiologic research. Journal of clinical epidemiology. 1995 Aug;48(8):1077-85.
	\bibitem{c741}	Lawrence JM, Umekubo MA, Chiu V, Petitti DB. Split sample analysis of serum folate levels after 18 days in frozen storage. Clinical laboratory. 2000;46(9-10):483-6.
	\bibitem{c742}	Nelson BC, Satterfield MB, Sniegoski LT, Welch MJ. Simultaneous quantification of homocysteine and folate in human serum or plasma using liquid chromatography/tandem mass spectrometry. Analytical chemistry. 2005 Jun 1;77(11):3586-93.
	\bibitem{c743}	Hutcheon JA, Chiolero A, Hanley JA. Random measurement error and regression dilution bias. BMJ (Clinical research ed. 2010;340:c2289.
	\bibitem{c744}	Snedecor GW, Cochran WG. Statistical Methods. Ames, Iowa: The Iowa State University Press 1979.
	\bibitem{c745}	Bjornsson HT, Sigurdsson MI, Fallin MD, Irizarry RA, Aspelund T, Cui H, et al. Intra-individual change over time in DNA methylation with familial clustering. Jama. 2008 Jun 25;299(24):2877-83.
	\bibitem{c746}	Bollati V, Baccarelli A, Hou L, Bonzini M, Fustinoni S, Cavallo D, et al. Changes in DNA methylation patterns in subjects exposed to low-dose benzene. Cancer research. 2007 Feb 1;67(3):876-80.
	\bibitem{c747}	Soriano-Tarraga C, Jimenez-Conde J, Giralt-Steinhauer E, Ois A, Rodriguez-Campello A, Cuadrado-Godia E, et al. DNA isolation method is a source of global DNA methylation variability measured with LUMA. Experimental analysis and a systematic review. PloS one. 2013;8(4):e60750.
	\bibitem{c748}	Karimi M, Johansson S, Stach D, Corcoran M, Grander D, Schalling M, et al. LUMA (LUminometric Methylation Assay)--a high throughput method to the analysis of genomic DNA methylation. Experimental cell research. 2006 Jul 1;312(11):1989-95.
	\bibitem{c749}	Laird PW. Principles and challenges of genomewide DNA methylation analysis. Nature reviews. 2010 Mar;11(3):191-203.
	\bibitem{c750}	Levy S, Sutton G, Ng PC, Feuk L, Halpern AL, Walenz BP, et al. The diploid genome sequence of an individual human. PLoS biology. 2007 Sep 4;5(10):e254.
	\bibitem{c751}	Ehrlich M. DNA hypomethylation in cancer cells. Epigenomics. 2009 Dec;1(2):239-59.
	\bibitem{c752}	Weisenberger DJ, Campan M, Long TI, Kim M, Woods C, Fiala E, et al. Analysis of repetitive element DNA methylation by MethyLight. Nucleic acids research. 2005;33(21):6823-36.
	\bibitem{c753}	Choi JY, James SR, Link PA, McCann SE, Hong CC, Davis W, et al. Association between global DNA hypomethylation in leukocytes and risk of breast cancer. Carcinogenesis. 2009 Nov;30(11):1889-97.
	\bibitem{c754}	Lim U, Flood A, Choi SW, Albanes D, Cross AJ, Schatzkin A, et al. Genomic methylation of leukocyte DNA in relation to colorectal adenoma among asymptomatic women. Gastroenterology. 2008 Jan;134(1):47-55.
	\bibitem{c755}	Ollikainen M, Smith KR, Joo EJ, Ng HK, Andronikos R, Novakovic B, et al. DNA methylation analysis of multiple tissues from newborn twins reveals both genetic and intrauterine components to variation in the human neonatal epigenome. Human molecular genetics. 2010 Nov 1;19(21):4176-88.
	\bibitem{c756}	Esteller M, Corn PG, Baylin SB, Herman JG. A gene hypermethylation profile of human cancer. Cancer research. 2001 Apr 15;61(8):3225-9.
	\bibitem{c757}	Merlo A, Herman JG, Mao L, Lee DJ, Gabrielson E, Burger PC, et al. 5' CpG island methylation is associated with transcriptional silencing of the tumour suppressor p16/CDKN2/MTS1 in human cancers. Nature medicine. 1995 Jul;1(7):686-92.
	\bibitem{c758}	Herman JG, Baylin SB. Gene silencing in cancer in association with promoter hypermethylation. The New England journal of medicine. 2003 Nov 20;349(21):2042-54.
	\bibitem{c759}	Schuebel KE, Chen W, Cope L, Glockner SC, Suzuki H, Yi JM, et al. Comparing the DNA hypermethylome with gene mutations in human colorectal cancer. PLoS genetics. 2007 Sep;3(9):1709-23.
	\bibitem{c760}	van den Donk M, van Engeland M, Pellis L, Witteman BJ, Kok FJ, Keijer J, et al. Dietary folate intake in combination with MTHFR C677T genotype and promoter methylation of tumor suppressor and DNA repair genes in sporadic colorectal adenomas. Cancer Epidemiol Biomarkers Prev. 2007 Feb;16(2):327-33.
	\bibitem{c761}	de Vogel S, Bongaerts BW, Wouters KA, Kester AD, Schouten LJ, de Goeij AF, et al. Associations of dietary methyl donor intake with MLH1 promoter hypermethylation and related molecular phenotypes in sporadic colorectal cancer. Carcinogenesis. 2008 Sep;29(9):1765-73.
	\bibitem{c762}	Casparie M, Tiebosch AT, Burger G, Blauwgeers H, van de Pol A, van Krieken JH, et al. Pathology databanking and biobanking in The Netherlands, a central role for PALGA, the nationwide histopathology and cytopathology data network and archive. Cell Oncol. 2007;29(1):19-24.
	\bibitem{c763}	Eads CA, Danenberg KD, Kawakami K, Saltz LB, Blake C, Shibata D, et al. MethyLight: a high-throughput assay to measure DNA methylation. Nucleic acids research. 2000 Apr 15;28(8):E32.
	\bibitem{c764}	Axume J, Smith SS, Pogribny IP, Moriarty DJ, Caudill MA. The MTHFR 677TT genotype and folate intake interact to lower global leukocyte DNA methylation in young Mexican American women. Nutrition research (New York, NY. 2007 Jan;27(1):1365-17.
	\bibitem{c765}	Gregory JF, 3rd, Williamson J, Liao JF, Bailey LB, Toth JP. Kinetic model of folate metabolism in nonpregnant women consuming [2H2]folic acid: isotopic labeling of urinary folate and the catabolite para-acetamidobenzoylglutamate indicates slow, intake-dependent, turnover of folate pools. The Journal of nutrition. 1998 Nov;128(11):1896-906.
	\bibitem{c766}	Butterworth AS, Higgins JP, Pharoah P. Relative and absolute risk of colorectal cancer for individuals with a family history: a meta-analysis. Eur J Cancer. 2006 Jan;42(2):216-27.
	\bibitem{c767}	Johns LE, Houlston RS. A systematic review and meta-analysis of familial colorectal cancer risk. The American journal of gastroenterology. 2001 Oct;96(10):2992-3003.
	\bibitem{c768}	Ogino S, Nishihara R, Lochhead P, Imamura Y, Kuchiba A, Morikawa T, et al. Prospective study of family history and colorectal cancer risk by tumor LINE-1 methylation level. Journal of the National Cancer Institute. 2013 Jan 16;105(2):130-40.
	\bibitem{c769}	Fuchs CS, Willett WC, Colditz GA, Hunter DJ, Stampfer MJ, Speizer FE, et al. The influence of folate and multivitamin use on the familial risk of colon cancer in women. Cancer Epidemiol Biomarkers Prev. 2002 Mar;11(3):227-34.
	\bibitem{c770}	Kastrinos F, Syngal S. Inherited colorectal cancer syndromes. Cancer journal (Sudbury, Mass. 2011 Nov-Dec;17(6):405-15.
	\bibitem{c771}	de la Chapelle A. Genetic predisposition to colorectal cancer. Nat Rev Cancer. 2004 Oct;4(10):769-80.
	\bibitem{c772}	Kampman E. A first-degree relative with colorectal cancer: what are we missing? Cancer Epidemiol Biomarkers Prev. 2007 Jan;16(1):1-3.
	\bibitem{c773}	Yun KE, Chang Y, Jung HS, Kim CW, Kwon MJ, Park SK, et al. Impact of body mass index on the risk of colorectal adenoma in a metabolically healthy population. Cancer research. 2013 Jul 1;73(13):4020-7.
	\bibitem{c774}	Sellers TA, Bazyk AE, Bostick RM, Kushi LH, Olson JE, Anderson KE, et al. Diet and risk of colon cancer in a large prospective study of older women: an analysis stratified on family history (Iowa, United States). Cancer Causes Control. 1998 Aug;9(4):357-67.
	\bibitem{c775}	Mitchell RJ, Brewster D, Campbell H, Porteous ME, Wyllie AH, Bird CC, et al. Accuracy of reporting of family history of colorectal cancer. Gut. 2004 Feb;53(2):291-5.
	\bibitem{c776}	Mai PL, Garceau AO, Graubard BI, Dunn M, McNeel TS, Gonsalves L, et al. Confirmation of family cancer history reported in a population-based survey. Journal of the National Cancer Institute. 2011 May 18;103(10):788-97.
	\bibitem{c777}	Madlensky L, Daftary D, Burnett T, Harmon P, Jenkins M, Maskiell J, et al. Accuracy of colorectal polyp self-reports: findings from the colon cancer family registry. Cancer Epidemiol Biomarkers Prev. 2007 Sep;16(9):1898-901.
	\bibitem{c778}	Murff HJ, Spigel DR, Syngal S. Does this patient have a family history of cancer? An evidence-based analysis of the accuracy of family cancer history. Jama. 2004 Sep 22;292(12):1480-9.
	\bibitem{c779}	Trimble KC, Molloy AM, Scott JM, Weir DG. The effect of ethanol on one-carbon metabolism: increased methionine catabolism and lipotrope methyl-group wastage. Hepatology (Baltimore, Md. 1993 Oct;18(4):984-9.
	\bibitem{c780}	Ulvik A, Ebbing M, Hustad S, Midttun O, Nygard O, Vollset SE, et al. Long- and short-term effects of tobacco smoking on circulating concentrations of B vitamins. Clinical chemistry. 2010 May;56(5):755-63.
	\bibitem{c781}	Connor Gorber S, Schofield-Hurwitz S, Hardt J, Levasseur G, Tremblay M. The accuracy of self-reported smoking: a systematic review of the relationship between self-reported and cotinine-assessed smoking status. Nicotine Tob Res. 2009 Jan;11(1):12-24.
	\bibitem{c782}	Feunekes GI, van 't Veer P, van Staveren WA, Kok FJ. Alcohol intake assessment: the sober facts. American journal of epidemiology. 1999 Jul 1;150(1):105-12.
	\bibitem{c783}	Ocke MC, Bueno-de-Mesquita HB, Goddijn HE, Jansen A, Pols MA, van Staveren WA, et al. The Dutch EPIC food frequency questionnaire. I. Description of the questionnaire, and relative validity and reproducibility for food groups. International journal of epidemiology. 1997;26 Suppl 1:S37-48.
	\bibitem{c784}	Ocke MC, Bueno-de-Mesquita HB, Pols MA, Smit HA, van Staveren WA, Kromhout D. The Dutch EPIC food frequency questionnaire. II. Relative validity and reproducibility for nutrients. International journal of epidemiology. 1997;26 Suppl 1:S49-58.
	\bibitem{c785}	Sharp L, Little J. Polymorphisms in genes involved in folate metabolism and colorectal neoplasia: a HuGE review. American journal of epidemiology. 2004 Mar 1;159(5):423-43.
	\bibitem{c786}	World Cancer Research Fund / American Institute for Cancer Research. Food, Nutrition, Physical Activity, and the Prevention of Cancer: a Global Perspective. Washington DC: AICR, 2007.
	\bibitem{c787}	Inoue-Choi M, Lazovich D, Prizment AE, Robien K. Adherence to the World Cancer Research Fund/American Institute for Cancer Research recommendations for cancer prevention is associated with better health-related quality of life among elderly female cancer survivors. J Clin Oncol. 2013 May 10;31(14):1758-66.
	\bibitem{c788}	Fryer AA, Nafee TM, Ismail KM, Carroll WD, Emes RD, Farrell WE. LINE-1 DNA methylation is inversely correlated with cord plasma homocysteine in man: a preliminary study. Epigenetics. 2009 Aug 16;4(6):394-8.
	\bibitem{c789}	Steegers-Theunissen RP, Obermann-Borst SA, Kremer D, Lindemans J, Siebel C, Steegers EA, et al. Periconceptional maternal folic acid use of 400 microg per day is related to increased methylation of the IGF2 gene in the very young child. PloS one. 2009;4(11):e7845.
	\bibitem{c790}	Thomas DC, Conti DV, Baurley J, Nijhout F, Reed M, Ulrich CM. Use of pathway information in molecular epidemiology. Human genomics. 2009 Oct;4(1):21-42.
	\bibitem{c791}	Baurley JW, Conti DV, Gauderman WJ, Thomas DC. Discovery of complex pathways from observational data. Statistics in medicine. 2010 Aug 30;29(19):1998-2011.
	\bibitem{c792}	Wilson MA, Baurley JW, Thomas DC, Conti DV. Complex system approaches to genetic analysis Bayesian approaches. Advances in genetics. 2010;72:47-71.
	\bibitem{c793}	Weggemans RM, Schaafsma G, Kromhout D. Toward an optimal use of folic acid: an advisory report of the Health Council of the Netherlands. European journal of clinical nutrition. 2009 Aug;63(8):1034-6.
	\bibitem{c794}	Nijhout HF, Reed MC, Budu P, Ulrich CM. A mathematical model of the folate cycle: new insights into folate homeostasis. The Journal of biological chemistry. 2004 Dec 31;279(53):55008-16.
	\bibitem{c795}	Quinlivan EP, Davis SR, Shelnutt KP, Henderson GN, Ghandour H, Shane B, et al. Methylenetetrahydrofolate reductase 677C->T polymorphism and folate status affect one-carbon incorporation into human DNA deoxynucleosides. The Journal of nutrition. 2005 Mar;135(3):389-96.
	\bibitem{c796}	Fenech M, Aitken C, Rinaldi J. Folate, vitamin B12, homocysteine status and DNA damage in young Australian adults. Carcinogenesis. 1998 Jul;19(7):1163-71.
	\bibitem{c797}	Basten GP, Duthie SJ, Pirie L, Vaughan N, Hill MH, Powers HJ. Sensitivity of markers of DNA stability and DNA repair activity to folate supplementation in healthy volunteers. British journal of cancer. 2006 Jun 19;94(12):1942-7.
	\bibitem{c798}	Jung AY, Smulders Y, Verhoef P, Kok FJ, Blom H, Kok RM, et al. No effect of folic acid supplementation on global DNA methylation in men and women with moderately elevated homocysteine. PloS one. 2011;6(9):e24976.
	\bibitem{c799}	Axume J, Smith SS, Pogribny IP, Moriarty DJ, Caudill MA. Global leukocyte DNA methylation is similar in African American and Caucasian women under conditions of controlled folate intake. Epigenetics. 2007 Jan-Mar;2(1):66-8.
\end{thebibliography}
