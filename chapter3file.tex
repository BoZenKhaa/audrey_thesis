\chapter{Age and sex but not plasma folate nor \emph{MTHFR} C677T are associated with leukocyte LINE-1 methylation in a healthy population} 
\chaptermark{B vitamins and LINE-1 in a healthy population}
\label{chap3_nbs} 

\quad\\

\noindent
Audrey Y. Jung\\
Martin den Heijer\\
Carolien Lute\\
Lambertus A.L.M. Kiemeney\\
Per M. Ueland\\
{\O}ivind Midttun\\
Suzanne Holewijn\\
Jacqueline de Graaf\\
Henk Blom\\
Wilma Steegenga\\
Ellen Kampman\\

\quad\\
\emph{Submitted}\\


\newpage

\section*{Abstract}

\noindent Plasma B vitamins and methionine may influence global DNA methylation, an early event in colorectal carcinogenesis. LINE-1 DNA methylation is often used as an indicator of global DNA methylation. Data describing the impact of plasma B vitamins on LINE-1 methylation in colorectal carcinogenesis via one-carbon metabolism in healthy individuals are sparse but are important given their potential for improving our understanding of cancer processes. We have used data from 1,142 age- and sex-stratified randomly selected individuals who had participated in a population-based survey. Those in the extremes (10th (2.69-5.62 nmol/L) and 90th (37.24-95.94 nmol/L) percentiles) of plasma folate concentrations were selected. In a total of 275 healthy, cancer-free persons with plasma folate in the 10th (n=138) and 90th (n=137) percentiles, we describe leukocyte LINE-1 DNA methylation in relation to circulating concentrations of B vitamins (folate, riboflavin, vitamin B6 species (pyridoxal 5'-phosphate, pyridoxal, 
and 4-pyridoxic acid), cobalamin, methylmalonic acid (MMA)) and methionine, and dietary and lifestyle characteristics. Age, sex, and \emph{MTHFR} C677T genotype were also examined as possible predictors of LINE-1 methylation in the extremes separately using multivariable linear regression. Our results suggest no difference in leukocyte LINE-1 DNA methylation between those in the 10th and in the 90th percentile for plasma folate. Age was inversely related to LINE-1 methylation in both extremes, and LINE-1 DNA methylation was lower in females compared with males in the 10th percentile of plasma folate, while there was no association with \emph{MTHFR} C677T genotype.

\newpage

\section[]{Introduction} % level 1
\noindent An early event in cancer progression garnering widespread attention is the global loss of methylated cytosine residues in DNA \cite{c31,c32}, which possibly acts by increasing mutation rates and affecting chromosomal stability \cite{c33,c34}. There is a strong interest in understanding determinants of global DNA methylation in cancer, other diseased states, and during good health.

\noindent Folate and other B vitamins are thought to influence DNA methylation because of their essential role in regulating one-carbon metabolism (OCM), which includes the pathway directly involved in producing methyl groups for DNA methylation processes. Several lifestyle and genetic factors may influence OCM: alcohol interferes with folate absorption and metabolism \cite{c35}, cigarette smoking reduces systemic circulating concentrations of B vitamins \cite{c36}, and the T allele of the common 677C$\rightarrow$T polymorphism in the methylenetetrahydrofolate reductase (MTHFR) gene decreases activity of this enzyme and subsequently the formation of 5-methyltetrahydrofolate, the most abundant folate species in blood plasma.

\noindent Direct quantification of total 5-methylcytosine content is ideal for measuring global DNA methylation but is often cumbersome in epidemiological studies, as large amounts of DNA and specialized equipment are needed. As such, PCR-based methods evaluating methylation at repeat elements, such as long interspersed nuclear element (LINE-1) repeats, have been developed and are commonly used as a substitute for global DNA methylation. Consensus about whether LINE-1 methylation is a good surrogate for global DNA methylation for studies involving folate has not been established, yet it is often used for this very purpose.

\noindent Previous studies with healthy persons have observed leukocyte LINE-1 DNA methylation in association with age \cite{c37,c38,c39}, sex \cite{c310,c313}, arsenic \cite{c311}, iron \cite{c311}, white blood cell count \cite{c310}, industrial exposures \cite{c314}, benzene \cite{c315}, physical activity \cite{c316} and dietary folate intake of fortified foods \cite{c317}. In healthy subjects, neither intake of natural folates nor dietary folate equivalents have been associated with leukocyte LINE-1 methylation \cite{c317}, and there are reports of positive \cite{c318} and null associations \cite{c319} between plasma folate and LINE-1 methylation. Measuring B vitamin and methionine concentrations in plasma reflect nutrient absorption, nutrient-nutrient interactions, tissue turnover, metabolism, and excretion, and the inherent biases related to dietary assessment tools are avoided \cite{c320}.

\noindent Given the potential importance of folate and other B vitamins as determinants of global DNA methylation in health and disease, we aimed to elucidate their cross-sectional association to LINE-1 DNA methylation using a population-based survey. Evaluating the relationships between B vitamins and LINE-1 methylation in healthy individuals will further our understanding of these associations in health and disease. Here, we describe circulating concentrations of B vitamins and methionine, dietary and lifestyle characteristics, and leukocyte LINE-1 DNA methylation in healthy, cancer-free individuals in the 10th and 90th percentiles of plasma folate concentrations. We also examine age, sex, \emph{MTHFR} C677T genotype as possible predictors of LINE-1 methylation in these extremes of folate status, separately. Folate extremes were selected in order to capture the widest range of plasma folate. 

\section[]{Materials and methods} % level 1
\subsection[]{Study population}
\noindent Details about the study design have previously been presented \cite{c321}. Briefly, the data were obtained from the Nijmegen Biomedical Study, a population-based survey conducted by the Department for Health Evidence and the Department of Clinical Chemistry at Radboud University Medical Centre. Between November 2001 and February 2002, 21,756 age- and sex-stratified randomly selected inhabitants of the municipality of Nijmegen in the Netherlands received an invitation to complete a postal questionnaire on, e.g., lifestyle and medical history, and to donate a 10 ml blood sample in a Vacutainer tube. The response to the questionnaire was 43\% (n=9350), and among these, 69\% (n=6468) donated a blood sample. Later, between July 2007 and December 2010, those between the ages of 50 and 70 and who had previously agreed to be contacted again (n=5363) were invited to complete a food frequency questionnaire (FFQ), and to give another blood sample, from which plasma B vitamins would be measured \cite{c322}. Of 
the 2506 participants who completed a food frequency questionnaire, 1149 also gave a blood sample. Plasma concentrations of vitamin metabolites were measured in these participants. The Institutional Review Board approved the study and all participants provided written informed consent.

\subsection{Biochemical analyses} % level 2
\noindent Venous blood samples from non-fasting participants were collected in Vacutainer brand tubes containing EDTA and were put on ice immediately after collection. Plasma and buffy coat were pipetted separately into cryogenic vials and stored at -80\textsuperscript{o}C. Plasma concentrations of folate (5-methyltetrahydrofolate) and cobalamin were determined by microbiological assays using a colistin-sulfate resistant strain of \emph{Lactobacillus leichmannii} and a chloramphenicol-resistant strain of \emph{Lactobacillus casei}, respectively \cite{c323,c324}. Plasma concentrations of riboflavin, vitamin 6 species (pyridoxal 5'-phosphate (PLP), pyridoxal (PL), and 4-pyridoxic acid (PA)), and methionine) were measured using liquid chromatography-tandem mass spectrometry \cite{c325,c326}, and methylmalonic acid (MMA) was measured by gas chromatography-tandem mass spectrometry \cite{c327}. Samples were analysed in batches of 86 and quality control included 6 calibration samples, 3 control samples, and 1 blank 
sample in each batch. Coefficients of variation were 4-5\% (folate), 6-13\% (riboflavin), 3-11\% (vitamin B6 species), 4-5\% (cobalamin), 1-3\% (methionine), and 2-3\% (MMA). Laboratory staff were blinded to LINE-1 methylation status of the participants donating a blood sample. All biomarkers were analysed at BEVITAL AS, Bergen, Norway (www.bevital.no).

\subsection{DNA extraction, bisulfite conversion, and quantification of LINE-1 methylation} % level 2
\noindent The Hamilton STAR workstation was used to isolate DNA from buffy coat, as described by the manufacturer (Hamilton Robotics). First, magnetic beads that bind DNA were added to the samples. A magnetic box was then used to separate the DNA from other blood components. After DNA was eluted in an elution buffer, the amount of DNA in each sample was determined using a Caliper automation system (Caliper Life Sciences). One hundred nanograms of DNA were bisulfate converted using EZ DNA Methylation Gold\texttrademark~ Kit (Zymo Research). We have used and adapted the method to quantify global DNA methylation using bisulfite-PCR and pyrosequencing developed by Yang \emph{et al}. \cite{c328}. Repetitive elements primers were designed towards a consensus LINE-1 sequence (3 tandem CpG sites at positions 318, 321, and 327 (GenBank accession number X58075)). Analysis of DNA methylation in LINE-1 repetitive elements was performed (Qiagen N.V.). The PCR product was bound to Sepharose HP (Amersham Biosciences) and 
the Sepharose beads containing the immobilised PCR product were prepared for pyrosequencing according to the manufacturer's instructions (Qiagen N.V.). Pyrosequencing was performed on PCR product with bound LINE-1 sequencing primer using the Pyromark Q24 System (Qiagen N.V.). The nucleotide dispensation order was GCT CGT GTA GTC AGT CG. For each CpG site, the degree of methylation was expressed as the percentage of 5-methylated cytosines (\%5mC) over the sum of methylated and unmethylated cytosines. Mean LINE-1 methylation was the average methylation across all three sites. Reproducibility was confirmed by analysing 10\% of the samples in duplicate. The within-sample coefficients of variation in duplicate runs were 3.9-6.8\%.

\subsection{Genotyping assay} % level 2
\noindent Genotyping of the MTHFR C677T polymorphism was determined, in DNA extracted from buffy coat, using TaqMan\textregistered~in accordance with the information provided by Applied Biosystems (http://www.appliedbiosystems.com). Laboratory staff were blinded to B vitamin and LINE-1 methylation status of the study participants. To assess reproducibility, genotyping was repeated for 10\% of the samples, and complete concordance was found between the two sets of results. Genotype frequencies were in Hardy-Weinberg equilibrium as tested by a $\chi$\textsuperscript{2} test (P>0.05).

\subsection{Statistical Analyses} % level 2
\noindent Of the 1149 participants who gave a blood sample, 7 participants were using antibiotics at the time of blood draw and were subsequently excluded from analyses, as antibiotics interfere with the microbiological assays used to measure plasma folate and cobalamin, resulting in artificially low concentrations of these biomarkers. From the remaining 1142 participants, those in the extremes (the 10\textsuperscript{th} and 90\textsuperscript{th} percentiles) for plasma folate were selected -- 151 participants in the 10th percentile of plasma folate (2.69-5.62 nmol/L) and 151 participants in the 90\textsuperscript{th} percentile of plasma folate (37.24-95.94 nmol/L). Of these 302 participants, there was no information about \emph{MTHFR} C677T genotype for 5 persons, and 22 individuals had a personal history of cancer. Therefore, 275 participants were included in our analyses -- 138 with low plasma folate and 137 with high plasma folate.

\noindent Descriptive statistics were used to characterise the total population (n=275) stratified by extremes of plasma folate. Median and corresponding interquartile range (IQR) were calculated for all continuous variables. Categorical variables were expressed in numbers and corresponding percentages. Comparisons of population characteristics between those in the 10\textsuperscript{th} and 90\textsuperscript{th} percentiles were made using Wilcoxon rank sum test for continuous variables, and $\chi$\textsuperscript{2} test or Fisher's exact test for categorical variables. As an additional examination of the data, we performed the same analyses stratified by \emph{MTHFR} C677T genotype (CC \emph{vs}. TT). As well, multivariable linear regression models were used to explore whether age (continuous), sex (male/female), or \emph{MTHFR} C677T genotype (CC/CT/TT) were related to LINE-1 DNA methylation levels for the 10th and 90th folate percentiles separately. All analyses were performed using SAS version 9.2 (
SAS Institute, Cary, NC). 

\section[]{Results} % level 1
\noindent Characteristics of the population stratified by plasma folate extremes are presented in Table 1. Overall, mean LINE-1 DNA methylation expressed as a percentage of methylated cytosines was not different between the two groups - 73.7 (IQR 72.6-74.6) for the 10\textsuperscript{th} percentile and 73.6 (IQR 72.5-74.7) for the 90\textsuperscript{th} percentile. There was also no difference when we looked at each CpG site separately. Compared with those in the 90\textsuperscript{th} percentile of plasma folate, those in the 10\textsuperscript{th} percentile were similar in age, were more likely to be male, were less educated, had higher BMI, consumed less alcohol, had similar family histories of colorectal cancer, and were more likely to have \emph{MTHFR} 677CT or TT genotype. As expected, all plasma vitamers were lower in the 10\textsuperscript{th} percentile group except for MMA and methionine, which were higher. While not statistically significant, there were more ever smokers in the 10\textsuperscript{
th} percentile folate group compared with the 90\textsuperscript{th} percentile group.

\noindent Table 2 depicts the associations between age, sex, and \emph{MTHFR} C677T genotype with LINE-1 methylation within plasma folate extremes using multivariable linear models. For those in the 10\textsuperscript{th} folate percentile, age was inversely related to LINE-1 methylation ($\beta$=-0.07, 95\%CI -0.13, -0.01). Compared with males, females had lower LINE-1 methylation levels ($\beta$=-0.68, 95\%CI -1.37, 0.01). There was no association between \emph{MTHFR} C677T genotype and LINE-1 methylation (compared with CC genotype, $\beta$=0.59, 95\%CI -0.15, 1.33 for CT, and $\beta$=0.26, 95\%CI -0.76, 1.29 for TT). For those in the 90th folate percentile, age was also inversely related to LINE-1 methylation ($\beta$=-0.08, 95\%CI -0.14, -0.02). There was no association between sex ($\beta$=-0.41, 95\%CI -1.20, 0.38) nor \emph{MTHFR} C677T genotype and LINE-1 methylation (compared with CC genotype, $\beta$=-0.18, 95\%CI -0.99, 0.63 for CT, and $\beta$=-0.28, 95\%CI -1.50, 0.95 for TT). 


% TABLE 3.1
%\begin{landscape}
\begin{center}
\begin{table}
%\definecolor{lightgray}{gray}{0.9}
%\rowcolors{3}{}{lightgray}
\caption{Characteristics of the population by extremes of plasma folate concentration.}
\label{table3_1}
\begin{adjustbox}{width=\textwidth}
\renewcommand{\arraystretch}{1.1}
\begin{tabular}{L{7.6cm}C{4.0cm}C{3.6cm}}
\hline
~ & ~ & ~\\
~ &\multicolumn{2}{m{8.191001cm}}{\centering\bfseries Plasma folate concentration(nmol/L)}\\
~ & ~ & ~\\
~ &\textbf{10}\textbf{\textsuperscript{th}} \textbf{percentile} & \textbf{90}\textbf{\textsuperscript{th}} \textbf{percentile} \\
~ & {\bfseries 2.69-5.62 nmol/L} & {\bfseries 37.24-95.94 nmol/L} \\
~ & \textbf{n=138} & \textbf{n=137} \\
\hline
Mean LINE-1 DNA methylation (\%), median (IQR) & 73.7 (72.6-74.6) & 73.6 (72.5-74.7)\\
Methylation at position 318, median(IQR) & 79.2 (77.9-80.0) & 78.8 (77.6-79.9)\\
Methylation at position 321, median(IQR) & 70.1 (68.7-71.0) & 70.1 (68.8-71.1)\\
Methylation at position 327, median(IQR) & 72.1 (70.7-73.3) & 72.3 (70.9-73.5)\\
Age at blood draw, median (IQR) & 62.3 (56.6-66.9) & 61.2 (56.0-66.8)\\
Sex, n (\%)\textsuperscript{a} &~ &~\\ \quad Female & 54 (39.1) & 89 (65.0)\\ \quad Male & 84 (60.9) & 48 (35.0)\\
High education\textsuperscript{b}, n (\%)\textsuperscript{a} & 39 (29.3) & 61 (48.0)\\
BMI, median (IQR)\textsuperscript{a} & 27.5 (25.2-30.1) & 25.7 (23.4-28.7)\\
Plasma concentrations, median (IQR) &~ &~\\
\quad Riboflavin (nmol/L)\textsuperscript{a} & 11.7 (8.2-17.5) & 19.8 (12.6-31.5)\\
\quad PLP (nmol/L)\textsuperscript{a} & 37.4 (28.3-47.1) & 96.0 (62.2-140.1)\\
\quad PL (nmol/L)\textsuperscript{a} & 11.9 (9.2-14.6) & 21.9 (15.9-33.3)\\
\quad PA (nmol/L)\textsuperscript{a} & 14.0 (10.2-18.3) & 28.5 (20.0-44.6)\\
\quad Sum vitamin B6 (nmol/L)\textsuperscript{a} & 63.1 (50.9-79.4) & 138.0 (104.7-218.3)\\
\quad Cobalamin (pmol/L)\textsuperscript{a} & 347.0 (272.1-436.8) & 435.8 (342.3-548.2)\\
\quad MMA ($\mu$mol/L)\textsuperscript{a} & 0.18 (0.14-0.23) & 0.16 (0.14-0.20)\\
\quad Methionine ($\mu$mol/L)\textsuperscript{a} & 25.5 (22.5-28.4) & 23.5 (21.2-26.1)\\
Alcohol intake (g/day), median (IQR)\textsuperscript{a} & 5.4 (0.9-15.0) & 9.4 (1.7-22.9)\\
Smoking status, n (\%) &~ &~\\
\quad Current smoker & 30 (21.7) & 17 (12.4)\\
\quad Former smoker & 70 (50.7) & 68 (49.6)\\
\quad Never smoker & 38 (27.5) & 52 (38.0)\\
\end{tabular}
\end{adjustbox}
\end{table}
\end{center}


% TABLE 3.1 CONTINUED
\begin{center}
\begin{table}
%\definecolor{lightgray}{gray}{0.9}
%\rowcolors{3}{}{lightgray}
\caption*{Table 3.1 \emph{(continued)}}
\label{table3_1}
\begin{adjustbox}{width=\textwidth}
\renewcommand{\arraystretch}{1.2}
\begin{tabular}{L{7.2cm}L{4.0cm}L{3.6cm}}
~ & ~ & ~\\
{\textit{MTHFR} C677T genotype}, n (\%)\textsuperscript{a} &~ &~\\
\quad CC & 42 (30.4) & 75 (54.7)\\ \quad CT & 75 (54.4) & 47 (34.3)\\ \quad TT & 21 (15.2) & 15 (11.0)\\
Family history of colorectal cancer, n (\%) & 11 (8.0) & 17 (12.4)\\\hline 
\end{tabular}
\end{adjustbox}
\caption*{\footnotesize{\textsuperscript{a}P value < 0.05; difference tested between 10\textsuperscript{th} and 90\textsuperscript{th} percentiles with $\chi$\textsuperscript{2} or Fisher's exact test for categorical variables and Wilcoxon rank sum test for continuous variables. \\ \textsuperscript{b}College or university.}}
\end{table}
\end{center}
%\end{landscape}



% TABLE 3.2
\begin{flushleft}
\begin{table}
\caption{Cross-sectional associations of LINE-1 DNA methylation with age, sex, and \emph{MTHFR} C677T genotype according to plasma folate extremes.}\label{table3_2}
\begin{adjustbox}{width=\textwidth}
\renewcommand{\arraystretch}{1.3}
\begin{tabular}{lcccccc}
\hline 
~ &\multicolumn{6}{c}{\centering \textbf{Plasma folate concentration(nmol/L)}}\\
\bfseries
&\multicolumn{3}{c}{\parbox[t]{3cm}{\centering \textbf{ 10\textsuperscript{th}percentile 2.69-5.62 nmol/L \ n=138\\}}} 
&\multicolumn{3}{c}{\parbox[t]{3cm}{\centering \textbf{ 90\textsuperscript{th}percentile 37.24-95.94 nmol/L \ n=137\\}}}\\
~ & \textbf{n} & \textbf{\%5mC (IQR)} & \textbf{$\beta$ (95\%CI)} & \textbf{n} & \textbf{ \%5mC (IQR)} ~ & \textbf{$\beta$ (95\%CI)}\\
\hline
Age at blood draw & 138 &~ & {}-0.07 (-0.13, -0.01) & 137 &~ & {}-0.08 (-0.14, -0.02)\\
Sex &~ &~ &~ &~ &~ &~\\ \quad Male & 84 & 73.8 (72.5-74.8) & Ref.\textsuperscript{b} & 48 & 73.6 (72.4-75.0) & Ref.\textsuperscript{b}\\ \quad Female & 54 & 73.5 (72.6-74.3) & {}-0.68 (-1.37, 0.01) & 89 & 73.6 (72.6-74.5) & {}-0.41 (-1.20, 0.38)\\
{\textit{MTHFR}}{ C677T genotype}&~ &~ &~ &~ &~ &~\\ \quad CC & 42 & 73.4 (72.1-74.5) & Ref. & 75 & 73.7 (72.6-74.7) & Ref.\\ \quad CT & 75 & 73.9 (72.6-74.7) & 0.59 (-0.15, 1.33) & 47 & 73.6 (72.6-74.6) & {}-0.18 (-0.99, 0.63)\\ \quad TT & 21 & 73.7 (73.2-74.6) & 0.26 (-0.76, 1.29) & 15 & 72.9 (72.4-74.5) & {}-0.28 (-1.50, 0.95)\\
\hline 
\end{tabular}
\end{adjustbox}
\caption*{\footnotesize{\textsuperscript{a}Results of multivariable linear regression.\\ \textsuperscript{b}Ref. indicates reference	category}}
\end{table}
\end{flushleft}



\section[]{Discussion} % level 1
\noindent In the present study, we have described leukocyte LINE-1 DNA methylation and circulating concentrations of B vitamins and methionine, and dietary and lifestyle characteristics in healthy, cancer-free individuals aged 50-70 in the extreme groups (10\textsuperscript{th} and 90\textsuperscript{th} percentiles) of plasma folate. We did not observe a difference in leukocyte LINE-1 DNA methylation between the extremes of plasma folate. In healthy individuals, blood folate concentrations have previously been positively associated \cite{c318} and not associated with global DNA methylation \cite{c319} while dietary folate intake was also not associated with LINE-1 methylation \cite{c317}. Unlike in the current study, Friso \emph{et al}. \cite{c318} and Kok \emph{et al}. \cite{c319} have measured global DNA methylation using liquid chromatography(LC)-electrospray ionization(ESI)-mass spectrometry(MS) and stable isotope dilution liquid chromatrography(LC)-electrospray ionization(ESI)-tandem mass spectrometry(
MS/MS), respectively, as opposed to LINE-1 as we did. However, our results are consistent with those presented by Kok \emph{et al}. despite the differences in methods used to measure global DNA methylation. The previously observed null associations between LINE-1 methylation and dietary folate from natural foods and dietary folate equivalents \cite{c317} are also in line with our own results.

\noindent We have also observed that, as expected, plasma concentrations of circulating B vitamins were all higher for those in the 90\textsuperscript{th} percentile of plasma folate compared with the 10\textsuperscript{th} percentile. This may reflect differences in diet or lifestyle between the groups and common dietary sources of most B vitamins \cite{c329}. MMA, as a marker of vitamin B12 status \cite{c330} was higher in participants with low folate, which could be explained by lower vitamin B12 status in these subjects. Our observation of higher frequency of MTHFR 677CC genotype and lower frequency of TT genotype in the 90th percentile of plasma folate compared to those in the 10th percentile is consistent with previous published data \cite{c318}. This is explained by the higher catalytic activity of MTHFR associated with the MTHFR 677CC variant, which more efficiently converts 5,10-methylenetetrahydrofolate into 5-methyltetrahydrofolate in the presence of adequate riboflavin \cite{c318}. This lends 
support for the integrity of our data.

\noindent Our observation of an inverse association between age and LINE-1 methylation in leukocytes has been previously reported \cite{c37}; the inverse association between age and leukocyte global DNA methylation has also been documented \cite{c38,c39}. Lower LINE-1 methylation \cite{c310,c312} and global DNA methylation \cite{c39} in females compared with males have been observed in the majority of studies. Reasons for sex-specific differences in LINE-1 and global DNA methylation have not yet been fully elucidated, but Singer \emph{et al}. recently confirmed the presence of LINE-1 hypomethylation on the inactive X chromosome, which could play a role in maintaining X chromosome inactivation \cite{c331}. Natural hormone cycles \cite{c332} have also been proposed as a possible explanation for sex-specific differences in LINE-1 DNA methylation.

\noindent In our dataset, alcohol intake was higher in the 90th percentile than in the 10th percentile, which may seem paradoxical since alcohol is a folate antagonist, but this could be partly explained by the adherence to more healthful lifestyle habits by highly educated females and never smokers in subjects with high folate status \cite{c333}.

\noindent The use of LINE-1 methylation as a surrogate for global DNA methylation is attractive because it is straightforward and has been reasonably correlated with global DNA methylation \cite{c328}. Additionally, age \cite{c37,c39}, sex \cite{c310,c313}, arsenic \cite{c311}, iron \cite{c311}, white blood cell count \cite{c310}, industrial exposures \cite{c314}, benzene \cite{c315}, physical activity \cite{c316} and dietary folate intake of fortified foods \cite{c317} have been associated with LINE-1 methylation in leukocytes, and LINE-1 methylation in tumours has been shown to predict cancer survival in colon cancer patients \cite{c334} and has been associated with microsatellite instability \cite{c335}. However, results from the current study do not support an association between plasma folate and leukocyte LINE-1 methylation in healthy cancer-free individuals.

\noindent Comparisons between different studies are difficult when different methods are used to measure global DNA methylation (for example, LC-ESI-MS/MS \cite{c319} \emph{vs}. LC-ESI-MS \cite{c318}) or when repeat elements such as LINE-1 are used as a surrogate for global DNA methylation. Different methods could yield dissimilar results. In the current study, we have quantified LINE-1 methylation at 3 tandem CpG sites in a LINE-1 promoter. Other studies have measured LINE-1 at four CpG sites \cite{c311}, and it may be crucial to consider the variability in methylation at different loci \cite{c331} when interpreting results.

\noindent Our study employs a cross-sectional design, which does not allow us to make causal inferences, but nonetheless provide important insight into the relationships between plasma folate and LINE-1 methylation in healthy individuals. We have measured LINE-1 methylation in leukocytes, which are comprised of a mixture of granulocytes and agranulocytes. Previous studies have shown differences in LINE-1 methylation between neutrophils and lymphocytes \cite{c310}. Blood counts were not measured in the current study, and this should be kept in mind when interpreting our results. A strength of our study is that we have measurements of both plasma folate and LINE-1 methylation in the same blood sample. Moreover, our study is based on concentrations of B vitamins and biomarkers in plasma, which help avoid inherent biases associated with dietary assessment using FFQs \cite{c320}. The plasma folate concentrations in our study were essentially similar to those in other Dutch populations (range 3.1-32.4 nmol/L) \
cite{c319,c336}.

\noindent In conclusion, leukocyte LINE-1 DNA methylation levels did not differ between those in the 10th percentile and those in the 90th percentile for plasma folate concentrations in healthy individuals. For those in the 10th percentile, females had lower LINE-1 methylation compared to males, whereas this difference was not present for individuals in the 90th percentile. Age was inversely related to LINE-1 methylation independent of plasma folate, and there was no indication of an association between MTHFR C677T genotype and LINE-1 methylation. 
 
\begin{thebibliography}{12} 
	\bibitem{c31}	Cravo M, Pinto R, Fidalgo P, Chaves P, Gloria L, Nobre-Leitao C, et al. Global DNA hypomethylation occurs in the early stages of intestinal type gastric carcinoma. Gut. 1996 Sep;39(3):434-8. 
	\bibitem{c32}	Feinberg AP, Gehrke CW, Kuo KC, Ehrlich M. Reduced genomic 5-methylcytosine content in human colonic neoplasia. Cancer research. 1988 Mar 1;48(5):1159-61. 
	\bibitem{c33}	Chen RZ, Pettersson U, Beard C, Jackson-Grusby L, Jaenisch R. DNA hypomethylation leads to elevated mutation rates. Nature. 1998 Sep 3;395(6697):89-93. 
	\bibitem{c34}	Eden A, Gaudet F, Waghmare A, Jaenisch R. Chromosomal instability and tumors promoted by DNA hypomethylation. Science (New York, NY. 2003 Apr 18;300(5618):455. 
	\bibitem{c35}	Trimble KC, Molloy AM, Scott JM, Weir DG. The effect of ethanol on one-carbon metabolism: increased methionine catabolism and lipotrope methyl-group wastage. Hepatology (Baltimore, Md. 1993 Oct;18(4):984-9. 
	\bibitem{c36}	Ulvik A, Ebbing M, Hustad S, Midttun O, Nygard O, Vollset SE, et al. Long- and short-term effects of tobacco smoking on circulating concentrations of B vitamins. Clinical chemistry. 2010 May;56(5):755-63. 
	\bibitem{c37}	Bollati V, Schwartz J, Wright R, Litonjua A, Tarantini L, Suh H, et al. Decline in genomic DNA methylation through aging in a cohort of elderly subjects. Mechanisms of ageing and development. 2009 Apr;130(4):234-9. 
	\bibitem{c38}	Bjornsson HT, Sigurdsson MI, Fallin MD, Irizarry RA, Aspelund T, Cui H, et al. Intra-individual change over time in DNA methylation with familial clustering. Jama. 2008 Jun 25;299(24):2877-83. 
	\bibitem{c39}	Fuke C, Shimabukuro M, Petronis A, Sugimoto J, Oda T, Miura K, et al. Age related changes in 5-methylcytosine content in human peripheral leukocytes and placentas: an HPLC-based study. Annals of human genetics. 2004 May;68(Pt 3):196-204. 
	\bibitem{c310}	Zhu ZZ, Hou L, Bollati V, Tarantini L, Marinelli B, Cantone L, et al. Predictors of global methylation levels in blood DNA of healthy subjects: a combined analysis. International journal of epidemiology. 2012 Feb;41(1):126-39. 
	\bibitem{c311}	Tajuddin SM, Amaral AF, Fernandez AF, Rodriguez-Rodero S, Rodriguez RM, Moore LE, et al. Genetic and Non-genetic Predictors of LINE-1 Methylation in Leukocyte DNA. Environmental health perspectives. 2013 Apr 3. 
	\bibitem{c312}	Zhang FF, Cardarelli R, Carroll J, Fulda KG, Kaur M, Gonzalez K, et al. Significant differences in global genomic DNA methylation by gender and race/ethnicity in peripheral blood. Epigenetics. 2011 May;6(5):623-9. 
	\bibitem{c313}	El-Maarri O, Becker T, Junen J, Manzoor SS, Diaz-Lacava A, Schwaab R, et al. Gender specific differences in levels of DNA methylation at selected loci from human total blood: a tendency toward higher methylation levels in males. Human genetics. 2007 Dec;122(5):505-14. 
	\bibitem{c314}	Peluso M, Bollati V, Munnia A, Srivatanakul P, Jedpiyawongse A, Sangrajrang S, et al. DNA methylation differences in exposed workers and nearby residents of the Ma Ta Phut industrial estate, Rayong, Thailand. International journal of epidemiology. 2012 Dec;41(6):1753-60; author response 61-3. 
	\bibitem{c315}	Bollati V, Baccarelli A, Hou L, Bonzini M, Fustinoni S, Cavallo D, et al. Changes in DNA methylation patterns in subjects exposed to low-dose benzene. Cancer research. 2007 Feb 1;67(3):876-80. 
	\bibitem{c316}	Zhang FF, Cardarelli R, Carroll J, Zhang S, Fulda KG, Gonzalez K, et al. Physical activity and global genomic DNA methylation in a cancer-free population. Epigenetics. 2011 Mar;6(3):293-9. 
	\bibitem{c317}	Zhang FF, Santella RM, Wolff M, Kappil MA, Markowitz SB, Morabia A. White blood cell global methylation and IL-6 promoter methylation in association with diet and lifestyle risk factors in a cancer-free population. Epigenetics. 2012 Jun 1;7(6):606-14. 
	\bibitem{c318}	Friso S, Choi SW, Girelli D, Mason JB, Dolnikowski GG, Bagley PJ, et al. A common mutation in the 5,10-methylenetetrahydrofolate reductase gene affects genomic DNA methylation through an interaction with folate status. Proceedings of the National Academy of Sciences of the United States of America. 2002 Apr 16;99(8):5606-11. 
	\bibitem{c319}	Kok RM, Smith DE, Barto R, Spijkerman AM, Teerlink T, Gellekink HJ, et al. Global DNA methylation measured by liquid chromatography-tandem mass spectrometry: analytical technique, reference values and determinants in healthy subjects. Clin Chem Lab Med. 2007;45(7):903-11. 
	\bibitem{c320}	Jenab M, Slimani N, Bictash M, Ferrari P, Bingham SA. Biomarkers in nutritional epidemiology: applications, needs and new horizons. Human genetics. 2009 Jun;125(5-6):507-25. 
	\bibitem{c321}	Hoogendoorn EH, Hermus AR, de Vegt F, Ross HA, Verbeek AL, Kiemeney LA, et al. Thyroid function and prevalence of anti-thyroperoxidase antibodies in a population with borderline sufficient iodine intake: influences of age and sex. Clinical chemistry. 2006 Jan;52(1):104-11. 
	\bibitem{c322}	Holewijn S, den Heijer M, Swinkels DW, Stalenhoef AF, de Graaf J. The metabolic syndrome and its traits as risk factors for subclinical atherosclerosis. The Journal of clinical endocrinology and metabolism. 2009 Aug;94(8):2893-9. 
	\bibitem{c323}	Molloy AM, Scott JM. Microbiological assay for serum, plasma, and red cell folate using cryopreserved, microtiter plate method. Methods in enzymology. 1997;281:43-53. 
	\bibitem{c324}	Kelleher BP, Broin SD. Microbiological assay for vitamin B12 performed in 96-well microtitre plates. Journal of clinical pathology. 1991 Jul;44(7):592-5. 
	\bibitem{c325}	Midttun O, Hustad S, Solheim E, Schneede J, Ueland PM. Multianalyte quantification of vitamin B6 and B2 species in the nanomolar range in human plasma by liquid chromatography-tandem mass spectrometry. Clinical chemistry. 2005 Jul;51(7):1206-16. 
	\bibitem{c326}	Windelberg A, Arseth O, Kvalheim G, Ueland PM. Automated assay for the determination of methylmalonic acid, total homocysteine, and related amino acids in human serum or plasma by means of methylchloroformate derivatization and gas chromatography-mass spectrometry. Clinical chemistry. 2005 Nov;51(11):2103-9. 
	\bibitem{c327}	Ueland PM, Midttun O, Windelberg A, Svardal A, Skalevik R, Hustad S. Quantitative profiling of folate and one-carbon metabolism in large-scale epidemiological studies by mass spectrometry. Clin Chem Lab Med. 2007;45(12):1737-45. 
	\bibitem{c328}	Yang AS, Estecio MR, Doshi K, Kondo Y, Tajara EH, Issa JP. A simple method for estimating global DNA methylation using bisulfite PCR of repetitive DNA elements. Nucleic acids research. 2004;32(3):e38. 
	\bibitem{c329}	van Rossum CTM, Fransen HP, Verkaik-Kloosterman J, Buurma-Rethans EJM, Ock MC. Dutch National Food Consumption Survey 2007-2010: National Institute for Public Health and the Environment; 2011. Report No.: 350050006/2011. 
	\bibitem{c330}	Refsum H, Smith AD, Ueland PM, Nexo E, Clarke R, McPartlin J, et al. Facts and recommendations about total homocysteine determinations: an expert opinion. Clinical chemistry. 2004 Jan;50(1):3-32. 
	\bibitem{c331}	Singer H, Walier M, Nusgen N, Meesters C, Schreiner F, Woelfle J, et al. Methylation of L1Hs promoters is lower on the inactive X, has a tendency of being higher on autosomes in smaller genomes and shows inter-individual variability at some loci. Human molecular genetics. 2012 Jan 1;21(1):219-35. 
	\bibitem{c332}	El-Maarri O, Walier M, Behne F, van Uum J, Singer H, Diaz-Lacava A, et al. Methylation at global LINE-1 repeats in human blood are affected by gender but not by age or natural hormone cycles. PloS one. 2011;6(1):e16252. 
	\bibitem{c333}	Fraser GE, Welch A, Luben R, Bingham SA, Day NE. The effect of age, sex, and education on food consumption of a middle-aged English cohort-EPIC in East Anglia. Preventive medicine. 2000 Jan;30(1):26-34. 
	\bibitem{c334}	Ogino S, Nosho K, Kirkner GJ, Kawasaki T, Chan AT, Schernhammer ES, et al. A cohort study of tumoral LINE-1 hypomethylation and prognosis in colon cancer. Journal of the National Cancer Institute. 2008 Dec 3;100(23):1734-8. 
	\bibitem{c335}	Estecio MR, Gharibyan V, Shen L, Ibrahim AE, Doshi K, He R, et al. LINE-1 hypomethylation in cancer is highly variable and inversely correlated with microsatellite instability. PloS one. 2007;2(5):e399. 
	\bibitem{c336}	Brouwer DA, Welten HT, Reijngoud DJ, van Doormaal JJ, Muskiet FA. Plasma folic acid cutoff value, derived from its relationship with homocyst(e)ine. Clinical chemistry. 1998 Jul;44(7):1545-50. 
\end{thebibliography} 
