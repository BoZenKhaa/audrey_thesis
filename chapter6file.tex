\chapter[Dietary B vitamins and CRT risk in Lynch Syndrome]{Dietary B vitamin and methionine intake and \emph{MTHFR} C677T genotype on risk of colorectal tumours in patients with Lynch Syndrome: the GEOLynch Cohort Study} 
\label{chap6_lynch}

\quad\\

Audrey Y. Jung\\ 
Fr{\"a}nzel J.B. van Duijnhoven\\ 
Fokko M. Nagengast\\ 
Akke Botma\\ 
Renate C. Heine-Br{\"o}ring\\ 
Jan H. Kleibeuker\\ 
Hans F.A. Vasen\\ 
Jan L. Harryvan\\ 
Renate M. Winkels\\ 
Ellen Kampman\\ 
 
\quad\\ 

\noindent \emph{Submitted}
 
\newpage 
 

\section*{Abstract}
\textbf{Purpose}: Dietary intake of B vitamins and methionine, essential components of DNA synthesis and methylation pathways, may influence colorectal tumour (CRT) development. The impact of B vitamins on colorectal carcinogenesis in individuals with Lynch Syndrome (LS) is unknown but is of great importance given their high lifetime risk of developing neoplasms. The role of \emph{MTHFR} C677T genotype in modifying these relationships in LS individuals is also unclear. We investigated associations between dietary intakes of folate, vitamins B2, B6, B12, and methionine and CRT development in a prospective cohort study of 470 mismatch repair gene mutation carriers.

\noindent \textbf{Methods}: Dietary intakes were assessed by food frequency questionnaire. Cox regression models with robust sandwich covariance estimation, adjusted for age, sex, physical activity, number of colonoscopies during person-time, NSAID use, and mutual vitamins were used to calculate hazard ratios (HR) and 95\% confidence intervals (95\% CI). Analyses were also stratified by \emph{MTHFR} C677T genotype.

\noindent \textbf{Results}: Compared to the lowest tertile of intake, adjusted HRs (95\%CI)s for CRT development in the highest tertile were 1.06 (0.59-1.91) for folate, 0.77 (0.39-1.51) for vitamin B2, 0.98 (0.59-1.62) for vitamin B6, 1.24 (0.77-2.00) for vitamin B12, and 1.36 (0.83-2.20) for methionine. Low vitamin B2 and low methionine intake was statistically significantly associated with an increased risk of CRT in \emph{MTHFR} 677TT individuals compared to a combined reference of persons with low intake and CC genotype.

\noindent \textbf{Conclusions}: There was no suggestion that intake of any dietary B vitamin or methionine was associated with CRT development among those with Lynch Syndrome.
 
\newpage
\section{Introduction} % level 1
Lynch Syndrome (LS), the most common form of hereditary colorectal cancer, is an autosomal dominant disorder. Individuals with LS carry inactivating germline mutations in one of the DNA mismatch repair (MMR) genes \emph{MLH1}, \emph{MSH2}, \emph{MSH6}, and \emph{PMS2} \cite{c61,c64}. Compared to the general population, LS individuals are at considerably higher risk for developing colorectal cancer (CRC) to age 70, develop colorectal adenomas (CRA) at a younger age, and have a higher lifetime risk for other types of cancers such as, but not exclusive to, those in the endometrium, ovary, and stomach \cite{c65,c67}. Colorectal adenomas, precursors of CRC \cite{c68,c69} progress through the adenoma-carcinoma sequence, which seems to be accelerated in LS \cite{c610,c611}.

\noindent Overweight \cite{c612}, smoking \cite{c613}, and a dietary pattern containing high amounts of fried and fast food snacks \cite{c614} have previously been reported to increase the risk of CRA in LS, whereas daily aspirin use decreased CRC risk \cite{c615}. This indicates that modifiable lifestyle factors in addition to genetic factors are important among this high risk population. 

\noindent B vitamins and methionine are essential for nucleotide synthesis, and DNA methylation and repair reactions. Folate plays a central role in mediating one-carbon units for both DNA synthesis and DNA methylation, the two main pathways in folate-mediated one-carbon metabolism. The other B vitamins act as cofactors in folate metabolism reactions, and methionine is converted to \emph{S}-adenosylmethionine (SAM), the universal methyl donor for cellular methylation reactions. Decades of research into the role of B vitamins and methionine in colorectal tumour (CRT) development in the general population have resulted in the majority of studies pointing towards an inverse association between folate intake and risk of sporadic CRA \cite{c616,c618}, but inconsistencies \cite{c619,c622} have prompted hesitation by the World Cancer Research Fund Continuous Update Project in drawing firm conclusions \cite{c623}. Similarly, the relationships between other B vitamins and methionine and CRT development remain 
unresolved.

\noindent Dietary folate intake in combination with the common 677 C $\rightarrow$ T single nucleotide polymorphism (SNP) in the methylenetetrahydrofolate reductase (MTHFR) gene, vital for folate-mediated one-carbon metabolism, is usually inversely associated with CRT risk \cite{c624,c625}, although this interaction is not consistently observed \cite{c621,c626,c627}. While extensive strides in research have been made to elucidate the associations between B vitamin and methionine intake and risk of sporadic CRT, there still remains a paucity of data describing these relationships in a LS population, where given their high lifetime risk of developing neoplasms and the current contentious nature of B vitamins in CRT development, uncovering whether B vitamin and methionine intake can modify this risk would be certainly pertinent.

\noindent To the best of our knowledge, no studies have previously examined dietary B vitamin intake in relation to colorectal adenomas or carcinomas in a LS population. We, therefore, conducted a prospective cohort study to evaluate the associations between dietary folate, vitamin B2, vitamin B6, vitamin B12, and methionine intake and colorectal tumour (colorectal adenoma or carcinoma) development in individuals with LS. We also assessed whether \emph{MTHFR} C677T genotype modified these relationships.

\section{Material and Methods} % level 1
\subsection{Study subjects} % level 2
The GEOLynch prospective cohort study has been previously described \cite{c612}. Briefly, all eligible MMR gene mutation carriers were identified at the Netherlands Foundation for the Detection of Hereditary Tumours (NFDHT) in Leiden, the Radboud University Nijmegen Medical Centre (RUNMC) in Nijmegen, and the University Medical Centre Groningen (UMCG) in Groningen (all in the Netherlands). Information about MMR gene mutation carrier status was collected at the NFDHT. Eligible patients were Dutch-speaking, Caucasian, mentally competent to participate men and women between 18 and 80 years of age who were screened regularly by colonoscopy. Terminally ill patients, and those with familial adenomatous polyposis (FAP), inflammatory bowel diseases, a history of a complete proctocolectomy or colostomy were excluded.

\noindent Between July 2006 and July 2008, a total of 713 MMR gene mutation carriers appeared to be eligible to participate in the study after approval by their medical specialist. Nine participants were ineligible. A total of 499 individuals (71\% of 704) were willing to participate in the study. We were unable to retrieve medical and personal information from 29 participants. In total, 470 persons from at least 161 families were included in this study. Approval for this study was given by the RUNMC medical ethical committee. All participants provided written informed consent.

\subsection{Exposure assessment} % level 2
\noindent Usual dietary intake was assessed by a validated 183-item semi-quantitative self-administered food frequency questionnaire (FFQ) \cite{c628,c629}. Using this FFQ, subjects reported types and portions of food they consumed within the last month. Response choices for frequency were never, 1 day/month, 2-3 days/month, 1 day/week, 2-3 days/week, 4-5 days/week, 6-7 days/week. Response choices for portion size relative to a standard portion were half portion, 1 portion, 1.5 portions, 2 or more portions. Dietary intake of folate ($\mu$g dietary folate equivalents/day), vitamin B2 (mg/day), vitamin B6 (mg/day), vitamin B12 ($\mu$g/day), methionine (mg/day), alcohol (g/day), and total energy (kJ/day) were calculated using the Dutch food composition table from 2006 \cite{c630}. Information about age, sex, medical history, and lifestyle factors, such as smoking status, physical activity \cite{c631}, BMI, NSAID use, and education level was obtained from a standardised self-administered questionnaire.

\subsection{\emph{MTHFR} genotyping} % level 2
\noindent Participants were provided an Oragene DNA self-collection kit for saliva collection. Genotyping of the MTHFR C677T polymorphism was determined using TaqMan\textsuperscript{\textregistered}~in accordance with the information provided from the Applied Biosystems website (http://www.appliedbiosystems.com) in DNA extracted from saliva (DNA Genoteck Inc.). Laboratory staff was blinded to the clinical outcome status of the patients donating a saliva sample. To assess reproducibility, genotyping was repeated for 10\% of the samples, and complete concordance was found between the two sets of results. Genotype frequencies were in Hardy-Weinberg equilibrium as tested by a $\chi$\textsuperscript{2} test (P>0.05).

\subsection{Outcome data} % level 2
\noindent Information about colorectal tumours from colonoscopies was determined at the NFDHT \cite{c632} and from medical records at RUNMC and UMCG. Information about colonic surgeries, CRCs, and CRAs from prior colonoscopies was also collected before recruitment and during follow-up until December 2010. Location, size, and histology for all documented tumours were obtained from pathology reports.

\subsection{Statistical analyses} % level 2
\noindent Total energy-adjusted nutrient intake was computed using the linear residuals method \cite{c633}. Descriptive statistics were used to describe the baseline characteristics for the total population of MMR carriers (n=470) stratified by tertiles of folate intake.

\noindent Cox proportional hazards regression was used to calculate hazard ratios (HRs) and 95\% confidence intervals (CIs) to estimate risk of developing colorectal tumours from dietary B vitamin and methionine intake. Robust sandwich covariance estimation was used to account for dependency of observations within families. The lowest tertile of B vitamin and methionine intake was set as the reference category. The crude model included age and sex as covariates. The following were evaluated as possible confounders: age (continuous), sex, BMI (continuous), history of colorectal tumours (yes/no), number of colonoscopies during person time (continuous), NSAID use (< one per week/$\geq$ one per week), education (high \emph{vs}. lower educated), physical activity (high \emph{vs}. lower physical activity), smoking (current/former/never), alcohol (continuous), energy intake (continuous), dietary fiber (continuous), supplement use (yes/no). A covariate was considered to be a confounder if it was associated with 
vitamin intake and was also a risk factor for CRT; additionally, the covariate must have changed one of the HRs by at least 10\% using manual backward selection. For all vitamins, the fully adjusted model included age, sex, physical activity (high \emph{vs}. lower), number of colonoscopies during person-time (continuous), and NSAID use (< one per week/$\geq$ one per week). A fully adjusted model including mutual B vitamins (continuous) and methionine (continuous) was also constructed.

\noindent Person-time started at the time of questionnaire completion and ended at the date of colonoscopy at which the first colorectal adenoma was diagnosed during follow-up. For colorectal carcinomas during follow-up, the end date was the date of carcinoma diagnosis. For those without a detectable CRT during follow-up, censoring was at the date of their last known colonoscopy.

\noindent Effect measure modification by \emph{MTHFR} C677T genotype was investigated by stratification. We performed a median split for each vitamin intake and we chose low folate, low vitamin B2, low vitamin B6, low vitamin B12, and low methionine intakes in those with the CC genotype as the reference category. To test for multiplicative nutrient-gene interaction, we created a product term between B vitamin intake (low/high) and \emph{MTHFR} C677T genotype and included this term into our final model. We also examined possible interactions between dietary intakes of folate, vitamin B2, vitamin B6, vitamin B12, methionine and alcohol (low/high) and smoking status (never, former, current), where a combination of low intakes with either low alcohol intake or never smokers was chosen as the reference category. For stratification by MMR gene mutation, we set low intake as the reference for each of the genes \emph{MLH1}, \emph{MSH2}, and \emph{MSH6}. There was not enough statistical power to stratify by \emph{
PMS2}.

\noindent The lifetest procedure was used to evaluate if the proportional hazards models met the assumption of proportionality; no violation of the proportional hazards assumption was observed by visual inspection. Sensitivity analyses were performed by 1) excluding those with a history of colorectal tumours before the start of the study period (n=232) and 2) excluding those who developed CRCs during follow-up (n=7). All analyses were performed using SAS version 9.2 (SAS Institute, Cary, NC).

\section{Results} % level 1
In this prospective cohort study of MMR gene mutation carriers, 131 of 470 persons developed a CRT during a median person-time of 28.0 months. Table 1 shows the baseline characteristics of the total cohort stratified by tertiles of folate intake. There was a greater proportion of females in the medium folate intake group compared to the low and high intake groups. Compared to individuals in the lowest tertile of folate, persons in the highest tertile of folate, were older, had a higher education, reported higher intakes of vitamin B2, B6, B12, methionine and a higher consumption of vegetables, fruit, and dietary fiber. In addition, those in the highest tertile of folate, were less likely to be current smokers and more likely to be highly physically active compared with the lowest folate tertile. While those in the highest tertile of folate intake were less likely to have a history of colorectal tumours, they were more likely to have a history of other cancers, compared to those in the lowest tertile.

\noindent As can be seen in table 2, there was no clear association between dietary intake of any B vitamin or methionine and colorectal tumour risk after adjusting for age, sex, physical activity, number of colonoscopies during person time, and NSAID use, nor when mutual vitamins and methionine were included as covariates. Compared to the lowest tertile of intake, adjusted HRs (including mutual B vitamins and methionine) for CRT risk and corresponding 95\%CIs for the highest tertile were 1.06 (0.59-1.91) for folate, 0.77 (0.39-1.51) for vitamin B2, 0.98 (0.59-1.62) for vitamin B6, 1.24 (0.77-2.00) for vitamin B12, and 1.36 (0.83-2.20) for methionine.

\noindent We did not observe an association between \emph{MTHFR} C677T genotype and risk of CRT (data not shown). Table \ref{table6_3} shows that there was no effect measure modification by \emph{MTHFR} C677T genotype in the associations between dietary vitamin B and methionine intake and CRT risk (p for interaction > 0.05 for all). However, it is worthy to note that there was a significant increased risk for developing CRT in LS patients homozygous for the variant \emph{MTHFR} C677T allele who consumed low amounts of vitamin B2 compared to those in the reference category (HR=2.44, 95\%CI=1.20-4.93), while a non-significant reduction in CRT risk was seen for high vitamin B2 intake in combination with the \emph{MTHFR} 677TT genotype. We observed a similar relationship between methionine intake and CRT risk Ð a significant increased risk for individuals with the \emph{MTHFR} 677TT genotype in combination with low methionine intake compared to low methionine intake in \emph{MTHFR} 677CC individuals (HR=2.64, 95\
%CI=1.06-6.58), whereas no association with CRT risk was seen for high methionine intake in combination with the TT genotype.

\noindent Investigations for possible interactions between dietary vitamins and smoking and alcohol in the development of CRT showed a significant interaction between dietary folate and alcohol intake (p for interaction=0.02). Compared to low folate/low alcohol, adjusted HRs and corresponding 95\%CIs were 0.56 (0.29-1.09) for high folate/low alcohol, 0.84 (0.46-1.54) for low folate/high alcohol, 1.23 (0.66-2.32) for high folate/high alcohol. There was no interaction between any vitamin and smoking, nor were there any interactions between the other vitamins (all p-interaction > 0.05). Stratification by MMR gene mutation showed no striking differences for the different mutations. Compared to low folate intake, adjusted HRs and 95\%CIs for high folate intake were 1.11 (0.49-2.51) for \emph{MSH1} mutation carriers, 1.64 (0.82-3.29) for \emph{MSH2} mutation carriers, and 0.42 (0.12-1.48) for \emph{MSH6} mutation carriers.

\noindent In sensitivity analyses, B vitamin and methionine intake were not associated with CRT risk when those with prevalent tumours during person-time were excluded. Compared to the lowest tertile of intake, adjusted HRs including mutual vitamins and corresponding 95\%CIs for the highest tertile of intake was 0.59 (0.19-1.85) for folate, 0.82 (0.25-2.69) for vitamin B2, 1.98 (0.77-5.06) for vitamin B6, 1.56 (0.69-3.51) for vitamin B12, and 1.65 (0.74-3.68) for methionine. Restricting analyses to those who developed a CRA during person-time, adjusted HRs including mutual vitamins for developing CRA and corresponding 95\%CIs for the highest tertile was 1.02 (0.56-1.86) for folate, 0.82 (0.42-1.60) for vitamin B2, 0.95 (0.57-1.58) for vitamin B6, 1.18 (0.73-1.93) for vitamin B12, and 1.60 (0.97-2.60) for methionine compared to the lowest tertile of intake.


% TABLE 6.1 HERE
\begin{table}
\small
\caption{Baseline characteristics of the population by total energy-adjusted folate intake.}
\label{table6_1}
\begin{adjustbox}{width=\textwidth}
\begin{tabular}{C{4.795cm}C{3.5cm}C{3.5cm}C{3.3cm}}

\hline 
~ & {\bfseries Low folate intake}{\bfseries ${\leq}$ 183.71 $\mu$g/day}\bfseries (n=155) & {\bfseries Medium folate intake}\bfseries 183.72-224.29 $\mu$g/day (n=156) & {\bfseries High folate intake}{\bfseries {\textgreater} 224.29 $\mu$g/day} \bfseries (n=155)\\
\hline
 Person-months, median (IQR) & 27.2 (17.3-38.9) & 28.9 (17.3-39.0) & 29.7 (16.5-40.0)\\
\hline
\bfseries Patient characteristics & ~ & ~ & ~ \\
\hline
{Age (years), median (IQR)} & 46.8 (39.1-56.7) & 48.3 (39.4-57.7) & 53.1 (45.1-60.7)\\
 Sex (female), n (\%) & 91 (58.7) & 96 (61.5) & 92 (59.4)\\
 {BMI (kg/m}{\textsuperscript{2}}{), median (IQR)} & 24.4 (21.8-27.4) & 24.8 (22.8-26.9) & 24.2 (22.7-26.3)\\
{Higher education, n (\%)}{\textit{\textsuperscript{a}}} & 41 (26.6) & 56 (36.1) & 61 (40.1)\\
\hline
{\textbf{Dietary vitamin intakes}}{, median (IQR)} & ~ & ~ & ~ \\
\hline
 Vitamin B2 (mg/day) & 1.5 (1.2-1.7) & 1.6 (1.4-1.9) & 1.9 (1.7-2.2)\\
 Vitamin B6 (mg/day) & 1.7 (1.5-1.9) & 1.9 (1.7-2.1) & 2.0 (1.8-2.2)\\
 Vitamin B12 ($\mu$g/day) & 3.4 (2.6-4.3) & 3.8 (3.1-4.7) & 4.2 (3.4-5.8)\\
 Methionine (mg/day) & 1 233.9 (1 106.1-1 413.9) & 1 359.9 (1 175.1-1 503.3) & 1 388.7 (1 269.0-1 569.2)\\
\hline
\bfseries Other dietary and lifestyle factors & ~ & ~ & ~ \\
\hline
 Energy (kJ/day), median (IQR) & 8 750.3 (7 095.4-10 735.9) & 8 181.1 (6 800.8-10 269.4) & 8 876.9 (7 337.2-10 969.7)\\
 Vegetable (g/day), median (IQR) & 72.8 (50.5-110.8) & 125.2 (88.3-161.9) & 177.2 (136.0-224.3)\\
 Fruit (g/day), median (IQR) & 80.8 (42.4-166.2) & 154.9 (79.8-232.2) & 226.6 (149.6-333.6)\\
 Dietary fiber (g/day), median (IQR) & 20.2 (15.7-25.2) & 24.0 (19.7-29.3) & 28.8 (24.3-34.1)\\
 Red meat intake (g/day), median (IQR) & 48.7 (40.5-72.0) & 47.3 (30.7-63.8) & 42.7 (24.4-54.1)\\
{Supplement
use}{\textit{\textsuperscript{b}}}{, n (\%)} & 39 (25.2) & 43 (27.6) & 46 (29.7)\\
 Alcohol (g/day), median (IQR) & 7.2 (1.6-17.3) & 7.3 (0.9-16.0) & 6.8 (1.8-16.5)\\
 Smoking status, n (\%) & ~ & ~ & ~ \\
 \ \ \ \ Current & 38 (24.5) & 30 (19.2) & 16 (10.3)\\
 \ \ \ \ Former & 61 (39.4) & 61 (39.1) & 80 (51.6)\\
 \ \ \ \ Never & 56 (36.1) & 65 (41.7) & 58 (37.4)\\
 NSAID use ({\textless}1/week), n (\%) & 134 (86.5) & 140 (89.7) & 140 (90.3)\\
{High physical activity, n (\%)}{\textit{\textsuperscript{c}}} & 46 (30.1) & 43 (28.1) & 62 (41.1)\\
\hline

{\textbf{Clinical characteristics}}{, n (\%)} & ~ & ~ & ~ \\
\hline
{History of colorectal tumours} & 79 (51.0) & 76 (48.7) & 76 (49.0)\\
 History of other cancers & 18 (11.6) & 25 (16.0) & 39 (25.2)\\
 Colonoscopies during person-time & ~ & ~ & ~ \\
 \ \ \ \ 1 & 64 (41.3) & 49 (31.4) & 61 (39.4)\\
 \ \ \ \ 2 & 65 (41.9) & 71 (45.5) & 59 (38.1)\\
 \ \ \ \ ${\geq}$3 & 26 (16.8) & 35 (22.4) & {34 (21.9)}\\
 MMR gene mutation & ~ & ~ & ~ \\
 \ \ \ \ MLH1 & 59 (38.1) & 66 (42.3) & 52 (33.6)\\
 \ \ \ \ MSH2 \ \ & 65 (41.9) & 64 (41.0) & 62 (40.0)\\
 \ \ \ \ MSH6 & 30 (19.4) & 23 (14.7) & 40 (25.8)\\
 \ \ \ \ PMS2 & 0 (0.0) & 2 (1.3) & 1 (0.7)\\
{\textit{MTHFR}}{C677T genotype, n (\%)} & ~ & ~ & ~ \\
 \ \ \ CC & 60 (42.0) & 60 (40.5) & 62 (41.6)\\
 \ \ \ CT & 64 (44.8) & 73 (49.3) & 78 (52.4)\\
 \ \ \ TT & 19 (13.3) & 15 (10.1) & 9 (6.0)\\
\hline
\end{tabular}
\end{adjustbox}
\end{table}

% TABLE 6.2 HERE
\begin{sidewaystable}
\caption{Hazard ratios for B vitamin and methionine intake and colorectal tumour risk in MMR mutation carriers.}
\label{table6_2}
\begin{adjustbox}{width=18cm}
%\begin{tabularx}{\textwidth}{XXXXXX}
\begin{tabular}{C{4cm}C{2.5cm}C{3cm}C{3cm}C{3cm}C{3cm}}
\hline 
{\bfseries Cases/cohort, n} (122/448) & \bfseries Person-time (months), median (IQR) & \bfseries Age & sex adjusted (95\%CI) & {\bfseries Fully adjusted HR (95\%CI)} \textit{\textsuperscript{a}} & \textbf{Fully adjusted HR + mutual vitamins (95\% CI)}\textit{\textsuperscript{a}}\\
\hline
\bfseries Dietary factor & ~ & ~ & ~ & ~ & ~ \\
\hline
{\textbf{Folate}}{($\mu$g/day)} & ~ & ~ & ~ & ~ & ~ \\
{\ \ \ }{\textrm{${\leq}$}}{183.71} & 38/150 & \ 27.2 (17.3-38.9) & 1.00 (Ref.) & 1.00 (Ref.) & 1.00 (Ref.)\\
 \ \ \ 183.72-224.29 & 42/149 & \ 28.9 (17.3-39.0) & 1.01 (0.65-1.56) & 1.19 (0.75-1.90) & 1.27 (0.77-2.08)\\
 \ \ \ {\textgreater} 224.29 & 42/149 & \ 29.7 (16.5-40.0) & 1.02 (0.64-1.62) & 0.91 (0.56-1.48) & 1.06 (0.59-1.91)\\ ~ & ~ & ~ & P trend & 0.52 & 0.91\\
\hline
{\textbf{VitaminB2}}{ (mg/day)} & ~ & ~ & ~ & ~ & ~ \\
{\ \ \ }{\textrm{${\leq}$}}{1.49} & 41/149 & 27.2 (18.5-38.9) & 1.00 (Ref.) & 1.00 (Ref.) & 1.00 (Ref.)\\
 \ \ \ 1.50-1.83 & 43/149 & 27.9 (15.5-38.4) & 1.04 (0.68-1.58) & 0.98 (0.62-1.55) & 1.01 (0.59-1.70)\\
 \ \ \ {\textgreater} 1.83 & 38/150 & 33.1 (18.3-40.8) & 0.82 (0.55-1.22) & 0.73 (0.46-1.15) & 0.77 (0.39-1.51)\\ ~ & ~ & ~ & P trend & 0.10 & 0.35\\
\hline
{\textbf{Vitamin B6}}{ (mg/day)} & ~ & ~ & ~ & ~ & ~ \\
{\ \ \ }{\textrm{${\leq}$}}{1.75} & 44/148 & 25.5 (15.6-38.1) & 1.00 (Ref.) & 1.00 (Ref.) & 1.00 (Ref.)\\
 \ \ \ 1.76-2.00 & 36/149 & 30.2 (18.5-40.6) & 0.69 (0.47-1.03) & 0.76 (0.51-1.13) & 0.81 (0.52-1.28)\\
 \ \ \ {\textgreater} 2.00 & 42/151 & 29.9 (16.4-40.0) & 0.83 (0.53-1.29) & 0.83 (0.55-1.26) & 0.98 (0.59-1.62)\\ ~ & ~ & ~ & P trend & 0.32 & 0.98\\
\hline
{\textbf{Vitamin B12}}{($\mu$g/day)} & ~ & ~ & ~ & ~ & ~ \\
{\ \ \ }{\textrm{${\leq}$}}{3.33} & 37/149 & 28.8 (16.5-39.7) & 1.00 (Ref.) & 1.00 (Ref.) & 1.00 (Ref.)\\
 \ \ \ 3.33-4.46 & 41/147 & 29.3 (19.2-39.5) & 1.11 (0.73-1.67) & 1.00 (0.64-1.59) & 1.14 (0.71-1.85)\\
 \ \ \ {\textgreater} 4.46 & 44/152 & 27.7 (16.2-39.3) & 1.13 (0.76-1.68) & 1.00 (0.66-1.52) & 1.24 (0.77-2.00)\\ ~ & ~ & ~ & P trend & 0.83 & 0.45\\
\hline
{\textbf{Methionine}}{(mg/day)} & ~ & ~ & ~ & ~ & ~ \\
{\ \ \ }{\textrm{${\leq}$}}{1226.33} & 42/153 & 27.9 (18.1-38.5) & 1.00 (Ref.) & 1.00 (Ref.) & 1.00 (Ref.)\\
 \ \ \ 1226.33-1450.02 & 35/152 & 28.0 (18.0-39.8) & 0.75 (0.46-1.22) & 0.88 (0.52-1.50) & 1.00 (0.57-1.74)\\
 \ \ \ {\textgreater} 1450.02 & 45/143 & 28.3 (15.4-39.4) & 1.08 (0.73-1.59) & 1.08 (0.71-1.66) & 1.36 (0.83-2.20)\\ ~ & ~ & ~ & P trend & 0.82 & 0.22\\
\hline
%\end{tabularx}
\end{tabular}
\end{adjustbox}
\end{sidewaystable}


% TABLE 6.3 HERE
\begin{sidewaystable}
\caption{Adjusted hazard ratiosa for B vitamin and methionine intake and development of colorectal tumours stratified by \emph{MTHFR} C677T genotype.}
\label{table6_3}
\begin{tabularx}{\textwidth}{XXXXXXXXXXXXX}
\hline ~ & ~ & ~ & \multicolumn{2}{c}{\centering
\bfseries Folate} & \multicolumn{2}{c}{\centering
\bfseries Vitamin B2} & \multicolumn{2}{c}{\centering
\bfseries Vitamin B6} & \multicolumn{2}{c}{\centering
\bfseries Vitamin B12} & \multicolumn{2}{c}{\centering
\bfseries Methionine}\\
\hline
 \textbf{\textit{MTHFR }}\textbf{C677T
genotype} & Person-time (months), \ median (IQR) & Cases/Cohort & Low (${\leq}$203.96) & High ({\textgreater}203.96) & Low (${\leq}$1.67) & High ({\textgreater}1.67) & Low (${\leq}$1.86) & High ({\textgreater}1.86) & Low (${\leq}$3.84) & High ({\textgreater}3.84) & Low (${\leq}$1 327.51) & High ({\textgreater}1 327.51)\\
\hline
 CC & 26.6 (17.3-40.0) & 46/175 & 1.00 (Ref.) & 0.82 (0.44-1.53) & 1.00 (Ref.) & 0.91 (0.45-1.86) & 1.00 (Ref.) & 0.94 (0.51-1.73) & 1.00 (Ref.) & 0.93 (0.54-1.59) & 1.00 (Ref.) & 1.70 (0.89-3.26)\\
\hline
 CT & 28.9 (16.9-39.2) & 59/209 & 0.99 (0.60-1.65) & 0.99 (0.51-1.90) & 0.95 (0.49-1.86) & 1.11 (0.61-2.00) & 0.91 (0.51-1.61) & 1.25 (0.72-2.19) & 0.91 (0.54-1.53) & 1.20 (0.76-1.89) & 1.13 (0.63-2.02) & 1.87 (1.06-3.30)\\
\hline
 TT & 30.8 (15.8-39.5) & 11/42 & 1.06 (0.39-2.85) & 1.81 (0.53-6.18) & 2.44 (1.20-4.93) & 0.56 (0.11-2.73) & 1.17 (0.44-3.11) & 1.62 (0.50-5.23) & 1.71 (0.75-3.87) & 0.81 (0.15-4.29) & 2.64 (1.06-6.58) & 1.14 (0.32-4.03)\\
\hline
 \multicolumn{4}{l}{P-interaction gene-nutrient} & 0.60 & ~ & 0.17 & ~ & 0.42 & ~ & 0.33 & ~ & 0.15\\
\hline
\end{tabularx}
\end{sidewaystable}

\section{Discussion} % level 1
In this prospective cohort study of MMR gene mutation carriers, our results do not suggest an association between dietary folate, vitamin B2, vitamin B6, vitamin B12, and methionine intake and risk of CRT development. There was also no effect measure modification by \emph{MTHFR} C677T genotype in the relationship between any B vitamin or methionine and CRT risk. There was, however, suggestion that low vitamin B2 intake in \emph{MTHFR} 677TT individuals was related to increased risk of CRT compared to low intake in 677CC. Likewise, the same pattern was observed for methionine intake -- low intake of methionine in \emph{MTHFR} 677TT was associated with an increased risk of CRT compared to low intake in \emph{MTHFR} 677CC persons.

\noindent The role of B vitamin and methionine intake in the development of CRT is an area of considerable and ongoing interest; although the evidence is not entirely consistent, there is general consensus that dietary intake of folate is inversely associated with risk of sporadic CRT in the general population. Indeed, for dietary folate intake, inverse \cite{c616,c617,c634}, null \cite{c619,c622,c635,c636}, and direct \cite{c621} associations with risk of CRT have been reported in both cohort \cite{c617,c622,c634,c635} and case-control studies \cite{c616,c619,c621,c636} investigating these relationships. Vitamin B2 intake has been directly associated with CRA risk in a Dutch case-control study \cite{c621}, but unassociated with CRT in other studies \cite{c619,c620,c622,c635}. A meta-analysis of prospective studies revealed an inverse association between vitamin B6 intake and CRC risk \cite{c637}, while other studies did not \cite{c619,c620,c622,c638}. There are also accounts of no association between 
vitamin B12 and CRT risk \cite{c619,c620}. Likewise, methionine intake has been both unassociated with CRT risk \cite{c620,c635}, and inversely associated \cite{c622}. Given the these inconsistencies, the relationship between folate \cite{c639}, other B vitamins and methionine and sporadic CRT development remains unresolved.

\noindent Our results are in agreement with previous reports of null associations between dietary folate intake in sporadic CRT \cite{c619,c620,c622}, as well as with those from randomized controlled trials of folic acid and combined folic acid, vitamin B6, and vitamin B12 supplementation and sporadic CRA development \cite{c640,c642}. Various explanations could possibly account for the mixed findings between observational studies. We considered that range of dietary folate intake could influence the direction of the associations, but examination of the data does not show this to be a factor, as narrow ranges of dietary folate intake have been reported to be inversely associated \cite{c643}, directly associated \cite{c621}, and unassociated \cite{c622} with risk of CRT (<150 $\mu$g/day \emph{vs}. > 212 $\mu$g/day, <191 $\mu$g/day \emph{vs}. > 220 $\mu$g/day, <168 $\mu$g/day \emph{vs}. > 297 $\mu$g/day, respectively). Additionally, dietary folate intake could be higher in the United States and Canada compared 
to countries in Europe such as the Netherlands, where mandatory food fortification does not exist, and this could also alter development of CRT. This does not seem to be the case, however, as studies supporting associations between folate and CRT came from both American \cite{c617} and non-American populations \cite{c616,c634}.

\noindent The absence of a clear association between dietary B vitamins and methionine and CRT risk in our study could be in part due to the disparity in timing of the adenoma-carcinoma sequence in the general population compared to that in LS, where the progression from adenoma to carcinoma is less than 4 years \cite{c610,c644} compared to roughly ten years in the general population \cite{c645}. Given the rapid process of initiation to malignant transformation of CRT in LS, exposure to B vitamins and methionine in this brief window could theoretically be too short to be meaningful for CRT risk.

\noindent Another explanation of our results could be that the pathways in CRT development in the general population are different from that in LS. Fearon and Vogelstein have proposed a genetic model of colorectal tumourigenesis that describes the transition from healthy colonic tissue to neoplasia through a series of steps involving the accumulation of genetic and epigenetic changes \cite{c68}. In sporadic CRC, a mutation in the adenomatous polyposis coli (APC) gene, is frequently one of the earliest events in initiating colorectal carcinogenesis \cite{c646} but is uncommon in HNPCC tumours \cite{c647}. We could infer that this and other differences in microsatellite instability and MLH1 promoter hypermethylation could hypothetically alter the still-unclear fundamental mechanisms connecting folate-mediated one-carbon metabolism to CRT development.

\noindent In stratified analyses, we observed an increased risk of CRT with low vitamin B2 intake in those with the \emph{MTHFR} 677TT genotype compared with \emph{MTHFR} 677CC, whereas this was not observed for high intakes in TT carriers. Our findings in this subgroup analysis were based on small numbers, so prudence should be taken when interpreting the results. The presence of the T allele results in a thermolabile protein and reduces MTHFR activity \emph{in vitro} by approximately two-thirds in TT-homozygotes \cite{c648}, and this coupled with low levels of its cofactor -- riboflavin in its coenzymatic form (flavin adenine dinucleotide) -- was associated with a significant increased risk of colorectal carcinogenesis in our population of MMR gene mutation carriers. Van den Donk \emph{et al}. have also reported inverse associations between vitamin B2 intake and risk of sporadic CRA for all MTHFR C677T genotypes but with more pronounced associations among those with the TT genotype \cite{c621}. Low vitamin 
B intake was non-significantly linked to an increased risk of CRA in those with the TT genotype \cite{c621}. The significant association between low methionine intake and increased CRT risk for \emph{MTHFR} 677TT individuals compared to low methionine intake in \emph{MTHFR} 677CC that we observed has also been seen in a general Portuguese population \cite{c649}. Nevertheless, based on the small number of subjects in our stratified analysis, our results should be interpreted with discretion and certainly necessitate confirmation by other studies.

\noindent We found an interaction between dietary folate intake and alcohol intake (p-interaction=0.02). Although not statistically significant, the reduction in CRT risk by high folate intake was mitigated by high alcohol intake (HR=1.23, 95\%CI=0.66-2.32 for high folate/high alcohol vs. low folate/low alcohol). Alcohol is a folate antagonist, and as such interferes with folate bioavailability, transport, and metabolism \cite{c650}.

\noindent The major strengths of this study are its relatively large size, and prospective cohort design with a reasonably long follow-up period. Furthermore, all participants in our study were exclusively confirmed MMR gene mutation carriers. A high response rate of 71\%, which reduces the possibility of selection bias and recall, would allow our results to be generalized to other LS patients. There was also extensive information about many possible confounding factors, making possible their inclusion as covariates in our analyses.

\noindent Our study also has some limitations. As is the case with all observational studies, the potential for residual confounding, resulting from unmeasured or inadequately measured confounders, cannot be negated. Furthermore, while this study is a large prospective cohort study of MMR gene mutation carriers, it can be considered a small prospective study when compared to cohort studies in the general population, and we had limited power in our subgroup analyses. Misclassification of exposure data is possible since dietary intake was assessed using a self-administered FFQ, and although this FFQ has been sufficiently validated \cite{c628,c629}, it may be worthwhile for future studies to consider determining vitamin status from blood samples. Measuring vitamin status in blood also has the added advantages of accounting for uptake, distribution, and metabolism of vitamins in the body.

\noindent In conclusion, in this Lynch Syndrome population, intake of dietary folate, other B vitamins, and methionine were unassociated with CRT development. These associations, however, may depend on \emph{MTHFR} genotype. Larger studies with longer follow-up times are needed to further clarify the relationships between dietary B vitamin and methionine intake and risk of CRT in LS patients. 

\section*{Acknowledgements} 
We are grateful to all the participants of the GEOLynch Cohort Study. We thank Leontien Witjes (Wageningen University), Maria van Vugt (Radboud University Medical Centre), Mary Velthuizen (Netherlands Foundation for the Detection for Hereditary Tumours), Alice Donselaar (Netherlands Foundation for the Detection of Hereditary Tumours), Carolien Lute (Wageningen University), and Ramazan Buyukcelik (Erasmus Medical Centre) for their assistance with recruitment, data collection, DNA preparation, and genotyping. The work described in this paper was carried out with the support of the Dutch Cancer Society (KWF, grant number KUN-2007-3842). 
 
\begin{thebibliography}{12} 
	\bibitem{c61}	Bronner CE, Baker SM, Morrison PT, Warren G, Smith LG, Lescoe MK, et al. Mutation in the DNA mismatch repair gene homologue hMLH1 is associated with hereditary non-polyposis colon cancer. Nature. 1994 Mar 17;368(6468):258-61. 
	\bibitem{c62}	Fishel R, Lescoe MK, Rao MR, Copeland NG, Jenkins NA, Garber J, et al. The human mutator gene homolog MSH2 and its association with hereditary nonpolyposis colon cancer. Cell. 1993 Dec 3;75(5):1027-38. 
	\bibitem{c63}	Hendriks YM, Jagmohan-Changur S, van der Klift HM, Morreau H, van Puijenbroek M, Tops C, et al. Heterozygous mutations in PMS2 cause hereditary nonpolyposis colorectal carcinoma (Lynch syndrome). Gastroenterology. 2006 Feb;130(2):312-22. 
	\bibitem{c64}	Miyaki M, Konishi M, Tanaka K, Kikuchi-Yanoshita R, Muraoka M, Yasuno M, et al. Germline mutation of MSH6 as the cause of hereditary nonpolyposis colorectal cancer. Nature genetics. 1997 Nov;17(3):271-2. 
	\bibitem{c65}	Aarnio M, Sankila R, Pukkala E, Salovaara R, Aaltonen LA, de la Chapelle A, et al. Cancer risk in mutation carriers of DNA-mismatch-repair genes. International journal of cancer. 1999 Apr 12;81(2):214-8. 
	\bibitem{c66}	Hampel H, Stephens JA, Pukkala E, Sankila R, Aaltonen LA, Mecklin JP, et al. Cancer risk in hereditary nonpolyposis colorectal cancer syndrome: later age of onset. Gastroenterology. 2005 Aug;129(2):415-21. 
	\bibitem{c67}	Watson P, Lynch HT. The tumor spectrum in HNPCC. Anticancer research. 1994 Jul-Aug;14(4B):1635-9. 
	\bibitem{c68}	Fearon ER, Vogelstein B. A genetic model for colorectal tumorigenesis. Cell. 1990 Jun 1;61(5):759-67. 
	\bibitem{c69}	Hill MJ, Morson BC, Bussey HJ. Aetiology of adenoma--carcinoma sequence in large bowel. Lancet. 1978 Feb 4;1(8058):245-7. 
	\bibitem{c610}	Edelstein DL, Axilbund J, Baxter M, Hylind LM, Romans K, Griffin CA, et al. Rapid development of colorectal neoplasia in patients with Lynch syndrome. Clin Gastroenterol Hepatol. 2011 Apr;9(4):340-3. 
	\bibitem{c611}	Rijcken FE, Hollema H, Kleibeuker JH. Proximal adenomas in hereditary non-polyposis colorectal cancer are prone to rapid malignant transformation. Gut. 2002 Mar;50(3):382-6. 
	\bibitem{c612}	Botma A, Nagengast FM, Braem MG, Hendriks JC, Kleibeuker JH, Vasen HF, et al. Body mass index increases risk of colorectal adenomas in men with Lynch syndrome: the GEOLynch cohort study. J Clin Oncol. 2010 Oct 1;28(28):4346-53. 
	\bibitem{c613}	Winkels RM, Botma A, Van Duijnhoven FJ, Nagengast FM, Kleibeuker JH, Vasen HF, et al. Smoking increases the risk for colorectal adenomas in patients with lynch syndrome. Gastroenterology. 2012 Feb;142(2):241-7. 
	\bibitem{c614}	Botma A, Vasen, F.A., van Duijnhoven, F.J.B., Kleibeuker, J.H., Nagengast, F.M., Kampman, E. Dietary Patterns and Colorectal Adenomas in Lynch Syndrome. Cancer. 2012. 
	\bibitem{c615}	Burn J, Gerdes AM, Macrae F, Mecklin JP, Moeslein G, Olschwang S, et al. Long-term effect of aspirin on cancer risk in carriers of hereditary colorectal cancer: an analysis from the CAPP2 randomised controlled trial. Lancet. 2011 Dec 17;378(9809):2081-7. 
	\bibitem{c616}	Benito E, Cabeza E, Moreno V, Obrador A, Bosch FX. Diet and colorectal adenomas: a case-control study in Majorca. International journal of cancer. 1993 Sep 9;55(2):213-9. 
	\bibitem{c617}	Lee JE, Willett WC, Fuchs CS, Smith-Warner SA, Wu K, Ma J, et al. Folate intake and risk of colorectal cancer and adenoma: modification by time. The American journal of clinical nutrition. 2011 2011;93(4):817-25. 
	\bibitem{c618}	Sanjoaquin MA, Allen N, Couto E, Roddam AW, Key TJ. Folate intake and colorectal cancer risk: a meta-analytical approach. International journal of cancer. 2005 Feb 20;113(5):825-8. 
	\bibitem{c619}	Key TJ, Appleby PN, Masset G, Brunner EJ, Cade JE, Greenwood DC, et al. Vitamins, minerals, essential fatty acids and colorectal cancer risk in the United Kingdom Dietary Cohort Consortium. International journal of cancer. 2012 Aug 1;131(3):E320-5. 
	\bibitem{c620}	Shrubsole MJ, Yang G, Gao YT, Chow WH, Shu XO, Cai Q, et al. Dietary B vitamin and methionine intakes and plasma folate are not associated with colorectal cancer risk in Chinese women. Cancer Epidemiol Biomarkers Prev. 2009 Mar;18(3):1003-6. 
	\bibitem{c621}	van den Donk M, Buijsse B, van den Berg SW, Ocke MC, Harryvan JL, Nagengast FM, et al. Dietary intake of folate and riboflavin, MTHFR C677T genotype, and colorectal adenoma risk: a Dutch case-control study. Cancer Epidemiol Biomarkers Prev. 2005 Jun;14(6):1562-6. 
	\bibitem{c622}	de Vogel S, Dindore V, van Engeland M, Goldbohm RA, van den Brandt PA, Weijenberg MP. Dietary folate, methionine, riboflavin, and vitamin B-6 and risk of sporadic colorectal cancer. The Journal of nutrition. 2008 Dec;138(12):2372-8. 
	\bibitem{c623}	Research. WCRFAIfC. Continuous Update Project Report Summary. Food, Nutrition, Physical Activity, and the Prevention of Colorectal Cancer. 2011. 
	\bibitem{c624}	Martinez ME, Thompson P, Jacobs ET, Giovannucci E, Jiang R, Klimecki W, et al. Dietary factors and biomarkers involved in the methylenetetrahydrofolate reductase genotype-colorectal adenoma pathway. Gastroenterology. 2006 Dec;131(6):1706-16. 
	\bibitem{c625}	Slattery ML, Potter JD, Samowitz W, Schaffer D, Leppert M. Methylenetetrahydrofolate reductase, diet, and risk of colon cancer. Cancer Epidemiol Biomarkers Prev. 1999 Jun;8(6):513-8. 
	\bibitem{c626}	Kim J, Cho YA, Kim DH, Lee BH, Hwang DY, Jeong J, et al. Dietary intake of folate and alcohol, MTHFR C677T polymorphism, and colorectal cancer risk in Korea. The American journal of clinical nutrition. 2012 Feb;95(2):405-12. 
	\bibitem{c627}	Lee JE, Wei EK, Fuchs CS, Hunter DJ, Lee IM, Selhub J, et al. Plasma folate, methylenetetrahydrofolate reductase (MTHFR), and colorectal cancer risk in three large nested case-control studies. Cancer Causes Control. 2012 Apr;23(4):537-45. 
	\bibitem{c628}	Feunekes GI, Van Staveren WA, De Vries JH, Burema J, Hautvast JG. Relative and biomarker-based validity of a food-frequency questionnaire estimating intake of fats and cholesterol. The American journal of clinical nutrition. 1993 Oct;58(4):489-96. 
	\bibitem{c629}	Verkleij-Hagoort AC, de Vries JH, Stegers MP, Lindemans J, Ursem NT, Steegers-Theunissen RP. Validation of the assessment of folate and vitamin B12 intake in women of reproductive age: the method of triads. European journal of clinical nutrition. 2007 May;61(5):610-5. 
	\bibitem{c630}	Netherlands Nutrition Center. NEVO Nederlandse Voedingsmiddelentabel ( Dutch Food Composition Table ), The Hague: Netherlands Nutrition Center. 2006.
	\bibitem{c631}	Baecke JA, Burema J, Frijters JE. A short questionnaire for the measurement of habitual physical activity in epidemiological studies. The American journal of clinical nutrition. 1982 Nov;36(5):936-42. 
	\bibitem{c632}	De Jong AE, Morreau H, Van Puijenbroek M, Eilers PH, Wijnen J, Nagengast FM, et al. The role of mismatch repair gene defects in the development of adenomas in patients with HNPCC. Gastroenterology. 2004 Jan;126(1):42-8. 
	\bibitem{c633}	Willett WC, Howe GR, Kushi LH. Adjustment for total energy intake in epidemiologic studies. The American journal of clinical nutrition. 1997 Apr;65(4 Suppl):1220S-8S; discussion 9S-31S. 
	\bibitem{c634}	Roswall N, Olsen A, Christensen J, Dragsted LO, Overvad K, Tjonneland A. Micronutrient intake and risk of colon and rectal cancer in a Danish cohort. Cancer epidemiology. 2010 Feb;34(1):40-6. 
	\bibitem{c635}	Kabat GC, Miller AB, Jain M, Rohan TE. Dietary intake of selected B vitamins in relation to risk of major cancers in women. British journal of cancer. 2008 Sep 2;99(5):816-21. 
	\bibitem{c636}	Baron JA, Sandler RS, Haile RW, Mandel JS, Mott LA, Greenberg ER. Folate intake, alcohol consumption, cigarette smoking, and risk of colorectal adenomas. Journal of the National Cancer Institute. 1998 Jan 7;90(1):57-62. 
	\bibitem{c637}	Larsson SC, Orsini N, Wolk A. Vitamin B6 and risk of colorectal cancer: a meta-analysis of prospective studies. Jama. 2010 Mar 17;303(11):1077-83. 
	\bibitem{c638}	Zhang X, Lee JE, Ma J, Je Y, Wu K, Willett WC, et al. Prospective cohort studies of vitamin B-6 intake and colorectal cancer incidence: modification by time? The American journal of clinical nutrition. 2012 Aug 8. 
	\bibitem{c639}	Food, Nutrition, Physical Activity, and the Prevention of Cancer: a Global Perspective. 2007 2007. 
	\bibitem{c640}	Logan RF, Grainge MJ, Shepherd VC, Armitage NC, Muir KR. Aspirin and folic acid for the prevention of recurrent colorectal adenomas. Gastroenterology. 2008 Jan;134(1):29-38. 
	\bibitem{c641}	Song Y, Manson JE, Lee IM, Cook NR, Paul L, Selhub J, et al. Effect of combined folic Acid, vitamin b6, and vitamin B12 on colorectal adenoma. Journal of the National Cancer Institute. 2012 Oct 17;104(20):1562-75. 
	\bibitem{c642}	Wu K, Platz EA, Willett WC, Fuchs CS, Selhub J, Rosner BA, et al. A randomized trial on folic acid supplementation and risk of recurrent colorectal adenoma. The American journal of clinical nutrition. 2009 Dec;90(6):1623-31. 
	\bibitem{c643}	Larsson SC, Giovannucci E, Wolk A. A prospective study of dietary folate intake and risk of colorectal cancer: modification by caffeine intake and cigarette smoking. Cancer Epidemiol Biomarkers Prev. 2005 Mar;14(3):740-3. 
	\bibitem{c644}	Vasen HF, Nagengast FM, Khan PM. Interval cancers in hereditary non-polyposis colorectal cancer (Lynch syndrome). Lancet. 1995 May 6;345(8958):1183-4. 
	\bibitem{c645}	Winawer SJ, Fletcher RH, Miller L, Godlee F, Stolar MH, Mulrow CD, et al. Colorectal cancer screening: clinical guidelines and rationale. Gastroenterology. 1997 Feb;112(2):594-642. 
	\bibitem{c646}	Powell SM, Zilz N, Beazer-Barclay Y, Bryan TM, Hamilton SR, Thibodeau SN, et al. APC mutations occur early during colorectal tumorigenesis. Nature. 1992 Sep 17;359(6392):235-7. 
	\bibitem{c647}	Konishi M, Kikuchi-Yanoshita R, Tanaka K, Muraoka M, Onda A, Okumura Y, et al. Molecular nature of colon tumors in hereditary nonpolyposis colon cancer, familial polyposis, and sporadic colon cancer. Gastroenterology. 1996 Aug;111(2):307-17. 
	\bibitem{c648}	Frosst P, Blom HJ, Milos R, Goyette P, Sheppard CA, Matthews RG, et al. A candidate genetic risk factor for vascular disease: a common mutation in methylenetetrahydrofolate reductase. Nature genetics. 1995 May;10(1):111-3. 
	\bibitem{c649}	Guerreiro CS, Carmona B, Goncalves S, Carolino E, Fidalgo P, Brito M, et al. Risk of colorectal cancer associated with the C677T polymorphism in 5,10-methylenetetrahydrofolate reductase in Portuguese patients depends on the intake of methyl-donor nutrients. The American journal of clinical nutrition. 2008 Nov;88(5):1413-8. 
	\bibitem{c650}	Mason JB, Choi SW. Effects of alcohol on folate metabolism: implications for carcinogenesis. Alcohol (Fayetteville, NY. 2005 Apr;35(3):235-41. 
\end{thebibliography} 
