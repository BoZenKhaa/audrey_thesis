\chapter[Plasma B vitamins and LINE-1 methylation]{Plasma B vitamins and LINE-1 DNA methylation in leukocytes of patients  with a history of  colorectal adenomas}
\label{chap5_poliep} 

\quad\\

Audrey Y. Jung\\
Akke Botma\\
Carolien Lute\\
Henk J. Blom\\
Per M. Ueland\\
Gry Kvalheim\\
{\O}ivind Midttun\\
Fokko Nagengast\\
Wilma Steegenga\\
Ellen Kampman\\

\quad\\

\emph{Mol Nutr Food Res. 2013 Apr;57(4):698-708}
 
\newpage 
 

\section[]*{Abstract}
\noindent \textbf{Scope}: Low concentrations of folate, other B vitamins and methionine are associated with colorectal cancer (CRC) risk, possibly by changing DNA methylation patterns. Here, we examine whether plasma concentrations of B vitamins and methionine are associated with methylation of long interspersed nuclear element-1 (LINE-1) among those at high risk of colorectal cancer, i.e. patients with at least one histologically confirmed colorectal adenoma (CRA) in their life.

\noindent \textbf{Methods and results}: We used LINE-1 bisulfite pyrosequencing to measure global DNA methylation levels in leukocytes of 281 CRA patients. Multivariable linear regression was used to assess associations between plasma B vitamin concentrations and LINE-1 methylation levels. Plasma folate was inversely associated with LINE-1 methylation in CRA patients, while plasma methionine was positively associated with LINE-1 methylation.

\noindent \textbf{Conclusion}: This study does not provide evidence that in CRA patients, plasma folate concentrations are positively related to LINE-1 methylation in leukocytes but does suggest a direct association between plasma methionine and LINE-1 methylation in leukocytes.
 
\newpage 
 
\section[]*{Introduction} % level 1
\noindent Changes in DNA methylation appear to be crucial in colorectal carcinogenesis. Both an increase of DNA methylation at CpG islands in or near the promoters of specific genes \cite{c51,c52} and a global loss of methylated cytosines \cite{c53} have been implicated in colorectal cancer development. DNA hypermethylation contributes to the silencing of certain genes such as tumour suppressor genes \cite{c54}, while global DNA hypomethylation is thought to contribute to carcinogenesis by affecting chromosomal stability, telomeric regulation, transposable element reactivation, and gene expression regulation \cite{c55}. Methylation of long interspersed nuclear element (LINE-1) repeats is often used as a surrogate for genome-wide DNA methylation. LINE-1 repeats comprise approximately 18\% of the human genome and are heavily methylated in normal somatic cells \cite{c56}.

\noindent Owing to the current challenges of obtaining target tissues in humans such as that in the large bowel, DNA methylation in blood is gaining widespread use as a proxy for DNA methylation in specific tissues, including colorectal tissue, in epidemiological studies \cite{c57,c510}. A decrease in LINE-1 methylation in leukocytes has been associated with an increased risk for CRC \cite{c511}; likewise, a decrease in global DNA methylation in leukocytes has been associated with risk for CRA \cite{c57,c58}, which are established precursors for colorectal cancer \cite{c512,c514}.

\noindent Folate, vitamins B2 (riboflavin), B6 (pyridoxal 5'-phosphate is the active form), B12 (cobalamin), and methionine are essential in one-carbon metabolism \cite{c515}; B vitamins are needed for DNA synthesis, and are vital to the production of \emph{S}-adenosylmethionine (SAM), the universal methyl donor required for DNA methylation. Alcohol may limit methyl group availability by interfering with folate absorption and folate-metabolising enzymes \cite{c516}; cigarette smoking reduces systemic circulating levels of B vitamins, possibly by increasing the activity of antioxidant defense enzymes in tissues, which require folate, B6 species and riboflavin \cite{c517}. Furthermore, the common 677 C $\rightarrow$ T polymorphism in the methylenetetrahydrofolate reductase (MTHFR) gene has been shown to increase the risk of colorectal cancer \cite{c518} and colorectal adenomas \cite{c519} when folate status is low, although this association has not been consistently observed \cite{c520,c522}. Riboflavin in its 
coenzymatic form of flavin adenine dinucleotide (FAD) is an essential cofactor for MTHFR. For those who have the \emph{MTHFR} 677TT genotype, MTHFR enzyme activity decreases as a result of its inability to retain the FAD cofactor. Folate may protect against activity loss by increasing the affinity of MTHFR for FAD in the presence of folate \cite{c523, c524}. Folate intake in combination with a low or medium intake of vitamin B2 was shown to be a risk factor for colorectal adenomas in a previous study in the same population \cite{c525}.

\noindent The relationships between B vitamin concentrations and global DNA methylation in leukocytes are unclear. Results from several feeding trials in healthy subjects have been inconclusive \cite{c526,c529}. Additionally, to our knowledge, data describing the relationships between circulating levels of B vitamins and global DNA methylation in leukocytes of subjects at relatively high risk for colorectal cancer, i.e. those with previous adenomas \cite{c530,c531} and/or those with a family history of colorectal cancer in a first degree relative \cite{c532}, do not yet exist.

\noindent Similarly, randomised controlled trials studying folic acid supplementation on global DNA methylation in leukocytes and colon tissue both with healthy subjects and with adenoma patients yielded equivocal findings \cite{c533,c540}. There is emerging evidence about the dual role of folate, where high levels of folate may protect against development of colorectal neoplasia, but could promote established colorectal adenomas \cite{c541,c542}. Additionally, results from a recent meta-analysis showed that the number of adenomas is strongly associated with risk of developing advanced and nonadvanced metachronous neoplasms \cite{c531}. 
We investigated the associations between circulating levels of (combined) B vitamins, methionine and LINE-1 methylation in leukocytes of colorectal adenoma patients. We also examine whether these relationships were influenced by number of lifetime adenomas, history of colorectal cancer among first degree relatives, \emph{MTHFR} C677T genotype, smoking status, and alcohol intake.

\section[]{Materials and methods} % level 1
\subsection{Study design and population} % level 2
\noindent The POLIEP follow-up study is a cohort study of the cases (n=767), who had formerly participated in the POLIEP study \cite{c543}, a Dutch case-control study comprising ten different hospitals in the Netherlands and conducted between 1997 and 2002. The study was designed to investigate gene-environment interactions and risk of colorectal adenomas. All participants in the study had at least one histologically-confirmed colorectal adenoma ever in their life. Additional inclusion criteria are that participants were Dutch speaking, of European origin, between the ages of 18-75 years at the time of an endoscopy for eligibility into the case-control study, had no hereditary colorectal cancer syndromes, no chronic inflammatory bowel disease, no history of colorectal cancer, and no (partial) bowel resection. All cases who agreed to be involved in the follow-up study and those for whom we had a recent address were invited (n=519) to complete a validated semi-quantitative food frequency questionnaire (FFQ) \
cite{c544}, a lifestyle questionnaire, and to give a blood sample. The FFQ was used to ascertain dietary intake and supplement use. Supplement use included any multivitamins, folic acid supplements, and B vitamin complex supplements. All participants who gave a blood sample, in which LINE-1 methylation levels could be assessed, were included in our analyses (n=281). Self-reported family history of colorectal cancer is defined as having at least one first degree family member diagnosed with this cancer. Participants had a full colonoscopy or a sigmoidoscopy at entry into the case-control study and medical and pathology information from past and subsequent colonoscopies from the earliest colonoscopy up to and including the most recent colonoscopy were collected. Adenoma occurrence was ascertained by colonoscopy and pathology reports. Written informed consent was obtained from each participant. This study was approved by the Radboud University Nijmegen Medical Centre Ethics Committee (CMO-nr: 2005/283) and by 
the review board of every participating outpatient clinic.

\subsection{Biochemical analyses} % level 2
\noindent Blood samples were collected from non-fasting participants in EDTA tubes and put directly on ice and protected from sunlight. Within 30 minutes of collection, the samples were centrifuged at 1000 rpm for 3 minutes followed by 2000 rpm for 5 minutes at 4\textsuperscript{o}C. Plasma and buffy coat were pipetted separately into cryogenic vials and stored at -80\textsuperscript{o}C. Plasma concentrations of folate and cobalamin were determined by microbiological assays using a colistin-sulfate resistant strain of \emph{Lactobacillus leichmannii} and a chloramphenicol-resistant strain of \emph{Lactobacillus casei}, respectively \cite{c545,c546}. Plasma concentrations of methionine, riboflavin, and vitamin B6 species (pyridoxal 5'-phosphate (PLP), pyridoxal (PL), and 4-pyridoxic acid (PA)) were measured using liquid chromatography-tandem mass spectrometry \cite{c547,c548}. Samples were analysed in batches of 86 and quality control included 6 calibration samples, 3 control samples, and 1 blank sample in 
each batch. Coefficients of variation were 4 - 5\% (folate), 0.7 - 2.9\% (methionine), 5.5 - 13.2\% (riboflavin), 2.6 - 11.1\% (vitamin B6 species), 4 - 5\% (cobalamin), and 1.1 - 8.1\% (methylmalonic acid (MMA)). Samples were analysed in random order. The laboratory staff was blinded to the clinical outcome status of the patients donating the blood samples. All biomarkers were analyzed at BEVITAL AS, Norway (www.bevital.no).

\subsection{Genotyping assay} % level 2
\noindent DNA was isolated from whole blood, and the \emph{MTHFR} C677T polymorphism was genotyped using a PCR-RFLP method \cite{c549}. PCR was performed with internal negative controls. Reproducibility was confirmed by analysing 20\% of the samples in duplicate. There was also an external quality control program, and results showed a 100\% match with expected genotype. 

\subsection{LINE-1 analysis} % level 2
\subsubsection{DNA extraction and bisulfite conversion of DNA} % level 3
\noindent DNA was isolated from buffy coat using the Hamilton STAR workstation, as described by the manufacturer (Hamilton Robotics, Bonaduz, Switzerland). Magnetic beads that bind DNA were added to the samples. A magnetic box was then used to separate the DNA from other components of blood. DNA was eluted in an elution buffer. The amount of DNA in each sample was determined using a Caliper automation system (Caliper Life Sciences, Hopkinton, MA, USA). One hundred nanograms of DNA (concentration 25 ng/$\mu$l) were bisulfite converted using EZ DNA Methylation Gold\texttrademark~Kit (Zymo Research, Orange, CA, USA) according to the manufacturer's protocol. Bisulfite-converted DNA was used for LINE-1 PCR immediately following conversion. 

\subsubsection{LINE-1 PCR and pyrosequencing} % level 3
\noindent The use of pyrosequencing to measure LINE-1 methylation is a validated method and previous studies have demonstrated that LINE-1 methylation is a good indicator of cellular 5-methylcytosine levels \cite{c56}. We used and adapted the method developed by Yang \emph{et al}. Global DNA methylation was quantified using bisulfite-PCR and pyrosequencing \cite{c56}. Repetitive elements primers were designed towards a consensus LINE-1 sequence (positions 331 to 318 for LINE-1 (GenBank accession number X58075)). Analysis of DNA methylation in LINE-1 repetitive elements was performed (Qiagen N.V., Germany). PCR conditions consisted of an initialization step of 95\textsuperscript{o}C for 15 minutes followed by touchdown PCR with denaturation at 95\textsuperscript{o}C for 30 seconds, annealing at 53\textsuperscript{o}C for 30 seconds (-1\textsuperscript{o}C each cycle), and extension at 72\textsuperscript{o}C for 40 seconds for a total of 5 cycles then 35 cycles of 95\textsuperscript{o}C for 30 seconds, 50\
textsuperscript{o}C for 30 seconds, 72\textsuperscript{o}C for 40 seconds, followed by a final extension at 72\textsuperscript{o}C for 10 minutes. The PCR product was bound to Sepharose HP (Amersham Biosciences, Uppsala, Sweden) and the Sepharose beads containing the immobilised PCR product were prepared for pyrosequencing according to the manufacturer's instructions (Qiagen N.V., Germany). Pyrosequencing was performed on PCR product with bound LINE-1 sequencing primer and using the Pyromark Q24 System (Qiagen N.V., Germany). The nucleotide dispensation order was GCT CGT GTA GTC AGT CG. Methylation quantification was performed using the provided software, Pyromark. The degree of methylation was expressed as the percentage of 5-methylated cytosines (\%5mC) over the sum of methylated and unmethylated cytosines. Reproducibility was confirmed by analysing 10\% of the samples in duplicate. The within-sample coefficients of variation in duplicate runs were 4.3 - 7.3\%.

\subsection{Statistical analyses} % level 2
\noindent Statistical analyses were performed using SAS version 9.2. Natural logarithmic transformations were applied to normalize distributions of all plasma metabolites, alcohol intake, and BMI because the original distributions were skewed towards higher values. Median and corresponding interquartile ranges (IQR) were calculated for all continuous variables with the exception of age and number of endoscopies, which were presented as a mean with the corresponding standard deviation. Categorical variables were expressed in numbers and corresponding percentages. Results are presented separately for number of lifetime adenomas (one adenoma versus at least two adenomas). We also tested for significant differences in dietary and lifestyle characteristics between cases with one adenoma and cases with at least two adenomas. For categorical variables, we used the chi-square test, and for none of the variables were there cells with less than 5 observations. For non-parametric variables, the Wilcoxon-Mann-Whitney 
test was applied, and a Student's unpaired t test was used for parametric data.

\noindent To evaluate the associations between log-transformed B vitamins and LINE-1 methylation, we used unadjusted and multivariable least squares linear regression analyses. We estimated individual associations of plasma folate, methionine, riboflavin, PLP, PL, PA, and cobalamin with LINE-1 methylation in separate models. We also constructed separate models for those with 1 adenoma and for those with $\geq$2 adenomas. We calculated beta estimates ($\beta$) and corresponding 95\% confidence intervals (95\%CIs) to determine the associations between the B vitamins and LINE-1 methylation levels. To test for confounding in the association between plasma B vitamins and LINE-1 methylation, we added potential confounders to the models and checked whether crude estimates changed by more than 10\% when added to the linear regression models. Potential confounders that were selected a priori were smoking status, family history of CRC, BMI, alcohol intake, age, sex, and number of endoscopies. Correlation between the 
analytes was checked using Spearman correlation coefficients. The final linear regression models included the covariates age, sex, BMI, alcohol intake, smoking status, family history of colorectal cancer, and other analytes (mutual B-vitamins and methionine). The assumption of linearity between LINE-1 methylation in leukocytes and independent variables was confirmed using graphic methods.

\noindent In order to examine if a family history of colorectal cancer in a first-degree relative would change associations between B vitamins and LINE-1 methylation, we performed a stratified analysis. We also examined whether the associations between each B vitamin and LINE-1 methylation were modified by MTHFR C677T genotype, smoking status, alcohol intake, and sex. Interaction between plasma folate and plasma vitamin B2 was studied by stratification by tertiles of plasma vitamin B2. Given that folate intake in combination with a low or medium intake of vitamin B2 was a risk factor for CRA in earlier analyses within the same population \cite{c525}, it would be worthwhile to explore this interaction in relation to LINE-1 methylation by stratifying by vitamin B2; in addition to stratification by vitamin B2, we also stratified by methionine, the sum of vitamin B6, cobalamin, and MMA. Because the independent variables were log-transformed, the resulting regression coefficient, $\beta$, should be interpreted as 
a 1\% change in the independent variable corresponds to a $\beta$/100 change in the LINE-1 methylation. Statistical significance was tested at the 0.05 level.

\section[]{Results} % level 1
\subsection{Characteristics of the study population} % level 2
Dietary and lifestyle characteristics of the population can be found in table 1. Two hundred eighty-one patients were included in our analyses. Patients with at least two lifetime adenomas were older (mean age=67.38 years, SD=7.71) than those with one adenoma (mean age=64.37 years, SD=9.44). The proportion of females was similar between the two groups (about 41\%). Use of supplements containing B vitamins was higher in those with only one adenoma (26.2\%) compared to those with at least 2 adenomas (21.2\%). With the exception of PA and cobalamin, B vitamin plasma concentrations were all lower in the $\geq$ 2 adenoma group compared to the one adenoma group (Table 1). LINE-1 methylation in leukocytes was also slightly lower in those with recurrent adenomas (72.71\%, IQR 68.87-74.96) compared to those with one adenoma (73.34\% IQR 70.91-74.77).

\noindent Plasma analytes were not correlated (Spearman rank correlation coefficients between analytes -0.25 < $\rho$ < 0.56) with the exception of the plasma B6 vitamers PLP, PL, and PA, which were highly correlated (PLP and PL $\rho$=0.86, PLP and PA $\rho$=0.68, PL and PA $\rho$=0.82). Therefore, we excluded the B6 vitamers as covariates, using instead the sum of these vitamers in the multivariable regression model.

\subsection{B vitamins and LINE-1 methylation levels according to number of lifetime adenomas} % level 2
\noindent The overall multivariable-adjusted $\beta$-estimate (95\%CI) between plasma folate and LINE-1 methylation for the entire population was -1.46 (-2.65, -0.27). The results from multivariable linear regression analyses according to number of lifetimes adenomas are shown in table 2. Plasma folate was inversely associated with LINE-1 methylation for those with one lifetime adenoma (multivariable adjusted $\beta$-estimate (95\%CI) of -2.01 (-3.85, -0.19) as well as for patients with at least two lifetime adenomas (multivariable adjusted $\beta$ of -1.70 (-3.35, -0.06). The significant inverse relationship between plasma folate and LINE-1 methylation means that for patients with $\geq$ 2 adenomas, a 1\% increase in plasma folate is associated with a decrease of -1.70/100=-0.017 ($\beta$/100) units in LINE-1 methylation (\%5-methylated cytosines/total amount of cytosines). There were null associations between other B vitamins and LINE-1 methylation according to number of lifetime adenomas.

\subsection{B vitamins and LINE-1 methylation levels according to family history of colorectal cancer} % level 2
\noindent The associations between B vitamins and LINE-1 methylation stratified by family history of colorectal cancer can be found in table 3. In patients with a family history of CRC, we found a borderline significant inverse association between plasma folate and LINE-1 methylation ($\beta$ of -2.77 (5.57, 0.02)), adjusted for age, sex, BMI, alcohol intake, smoking status, and other analytes. There was a positive association between plasma methionine and LINE-1 methylation (multivariable-adjusted $\beta$-estimate of 3.91 (0.39, 7.42)), but only in patients with no family history. No other statistically significant associations were observed.

\subsection{B vitamin and LINE-1 methylation levels according to \emph{MTHFR} C677T genotype} % level 2
\noindent The associations between plasma B vitamins and LINE-1 methylation levels according to \emph{MTHFR} C677T genotype are shown in table 4. We found an inverse association between plasma folate and LINE-1 methylation for those with the \emph{MTHFR} 677CC genotype ($\beta$ of -2.01 (-3.77, -0.25)). For those with the \emph{MTHFR} 677TT genotype, there was an inverse association between plasma riboflavin and LINE-1 methylation (multivariable adjusted $\beta$ of -9.08 (-15.47, -2.69) and a positive association between the sum of B6 vitamers and LINE-1 methylation (multivariable adjusted $\beta$ of 12.00 (0.85, 23.15). No interaction terms reached statistical significance (data not shown).

\subsection{Plasma folate and LINE-1 methylation levels according to plasma riboflavin, smoking status, and alcohol intake} % level 2
\noindent The associations between plasma folate and LINE-1 methylation levels stratified by plasma riboflavin, smoking status, and alcohol intake are shown in table 5. Plasma riboflavin and alcohol intake were each divided into tertiles while smoking status was categorized as current, former, or never. We found inverse associations of LINE-1 methylation levels with plasma folate in the lowest tertile of plasma riboflavin ($\beta$ of-3.84 (-6.27, -1.40)) and in never smokers ($\beta$ of -3.19 (-6.23, -0.15)), and only in the second tertile of alcohol consumption ($\beta$ of -3.56 (-5.88, -1.04)). There were inverse associations between plasma folate and LINE-1 methylation for all levels of plasma methionine, cobalamin, sum of vitamin B6, and MMA but none were significant (data not shown).



% TABLE 5.1 HERE
\begin{table}
\small
\caption{Dietary and lifestyle characteristics by number of lifetime adenomas.}
\label{table5_1}
\begin{adjustbox}{width=\textwidth}
\begin{tabular}{L{5.5cm}L{3.5cm}L{3cm}}

\hline ~ & \bfseries\color{black} One adenoma & \bfseries\color{black} ${\geq}$ 2
adenomas\\
\hline
\bfseries Demographics & ~ & ~ \\

 Number of patients (\%) & 149 (53.0) & 132 (47.0)\\
 Age at blood draw (years), mean (SD)\textsuperscript{c)} & 64.4 (9.4) & 67.4 (7.7)\\
 Sex, female (\%) & 59 (40.7) & 53 (41.7)\\
{Education, n (\%)}{ High,}{ Medium,} Low & ~ { 39 (26.2)}{ 64 (43.0)} 37 (24.8) & ~ { 30 (22.7)}{ 61 (46.2)} 36 (27.3)\\

 \textbf{Dietary intake}, median (IQR) & ~ & ~ \\

 Folate, \textrm{${\mu}$}g/day & 192.9 (154.5-240.7) & 185.5 (152.4-230.0)\\

 Vitamin B2 (riboflavin) & 1.6 (1.2-1.9) & 1.5 (1.2-2.0)\\

 Vitamin B6 & 1.7 (1.4-2.0) & 1.7 (1.5-2.1)\\

 Vitamin B12
(cobalamin)\textsuperscript{c)} & 3.9 (3.1-5.1) & 4.3 (3.3-5.6)\\

 Alcohol intake, g/day & 10.0 (2.8-25.0) & 9.8 (1.7-22.1)\\

 \textbf{Plasma concentrations}, median (IQR) & ~ & ~ \\

 Folate (nmol/L) & 8.1 (5.49-13.47) & 7.6 (5.2-11.5)\\

 Methionine (\textrm{${\mu}$}mol/L) & 28.5 (24.6-34.1) & 28.3 (24.6-32.9)\\

 Riboflavin (nmol/L) & 17.7 (11.4-30.6) & 16.02 (10.2-30.2)\\

 Vitamin B6 sum (nmol/L) & 120.6 (90.1-186.3) & 116.8 (85.5-177.2)\\

 PLP (nmol/L) & 78.1 (50.6-113.8) & 72.5 (53.8-111.5)\\

 PL (nmol/L) & 16.6 (12.2-25.4) & 16.4 (11.6-25.7)\\

 PA (nmol/L) & 25.7 (19.0-45.5) & 26.6 (19.9-39.7)\\

 Cobalamin (pmol/L) & 337.9 (278.8-411.2) & 356.0 (284.0-431.4)\\

{MMA(}\textrm{${\mu}$}{mol/L)} & 0.2 (0.2-0.3) & {0.2 (0.2-0.3})\\

\bfseries Lifestyle characteristics & ~ & ~ \\

{BMI(kg/m}{\textsuperscript{2}}{), median (IQR)} & 25.7 (23.5-27.7) & 25.8 (24.2-28.3)\\

 Supplement use, yes (\%)\textsuperscript{a)} & 39 (26.2) & 28 (21.2)\\

{Smoking habits, n (\%)}{ Current}{ Ever} Never & ~ { 21 (14.5\%)}{ 80 (55.2\%)} 44 (30.3\%) & ~ { 14 (11.0\%)}{ 80 (63.0\%)} 31 (24.4\%)\\

\bfseries Other characteristics & ~ & ~ \\

{\textit{MTHFR}}{ C677T genotype, n (\%)}{ CC}{ CT} TT & ~ { 72 (49.7\%)}{ 62 (42.8\%)} 11 (7.6\%) & ~ { 53 (40.5\%)}{ 64 (48.9\%)} 14 (10.7\%)\\

 First degree family history of colorectal cancer, n (\%) & 42 (28.2\%) & 39 (29.6\%)\\

 Number of endoscopies, mean (SD)\textsuperscript{c)} & 2.9 (1.5) & 4.7 (2.2)\\

 Advanced adenomas, n
(\%)\textsuperscript{b),c)} & 69 (46.3\%) & 94 (71.2\%)\\

 LINE-1 methylation (\%), median (IQR) & 73.3 (70.91-74.8) & 72.7 (68.9-75.0)\\
\hline
\end{tabular}
\end{adjustbox}
\caption*{this is a footnote oh yeah!}
\end{table}

% TABLE 5.2 HERE
\begin{sidewaystable}
\caption{Associations between plasma B vitamins and LINE-1 methylation according to number of lifetime adenomas using multivariable linear regression.}
\label{table5_2}
\begin{adjustbox}{width=\textwidth}
\begin{tabular}{C{2.5cm}C{4.0cm}C{4.0cm}C{4cm}C{4cm}}

\hline ~ & \multicolumn{2}{c}{\centering
\bfseries 1 adenoma (n=149)} & \multicolumn{2}{c}{\centering
\bfseries ${\geq}$ 2 adenomas (n=132)}\\
\hline ~ & { Unadjusted }{ $\beta $\textsuperscript{c)}} 95\% confidence interval & { Multivariable adjusted\textsuperscript{a)}}{ $\beta $\textsuperscript{c)}} 95\% confidence interval & { Unadjusted}{ $\beta $\textsuperscript{c)}} 95\% confidence interval & { Multivariable adjusted\textsuperscript{a)}}{ $\beta $\textsuperscript{c)}} 95\% confidence interval\\
\hline
 Folate\textsuperscript{b)} & { {}-0.86} {}-1.96, 0.24 & { {}-2.01} {}-3.85, -0.19 & { {}-1.17} {}-2.40, -0.06 & { {}-1.70} {}-3.35, -0.06\\
\hline
 Methionine\textsuperscript{b)} & { 3.36} 0.01, 6.72 & { 3.47} {}-0.40, 7.33 & { {}-0.33} {}-4.24, 3.57 & { {}-0.47} {}-4.87, 3.94\\
\hline
 Riboflavin\textsuperscript{b)} & { {}-0.46} {}-1.53, 0.61 & { {}-0.60} {}-2.08, 0.87 & { {}-0.20} {}-1.31, 0.91 & { 0.46} {}-1.04, 1.95\\
\hline
 Vitamin B6 sum\textsuperscript{b)} & { 0.12} {}-1.01, 1.25 & { 1.68} {}-0.27, 3.63 & { {}-1.24} {}-2.73, 0.25 & { {}-1.28} {}-3.51, 0.94\\
\hline
 PLP\textsuperscript{b)} & { 0.29} {}-0.95, 1.53 & { 1.79} {}-0.27, 3.85 & { {}-1.00} {}-2.67, 0.67 & { {}-0.94} {}-3.36, 1.49\\
\hline
 PL\textsuperscript{b)} & { {}-0.20} {}-1.17, 0.78 & { {}-0.84} {}-0.86, 2.54 & { {}-1.22} {}-2.41, -0.02 & { {}-1.27} {}-3.00, 0.46\\
\hline
 PA\textsuperscript{b)} & { {}-0.10} {}-1.05, 0.85 & { 1.00} {}-0.49, 2.49 & { {}-1.11} {}-2.27, 0.06 & { {}-0.98} {}-2.66, 0.69\\
\hline
 Cobalamin\textsuperscript{b)} & { 0.35} {}-1.94, 2.64 & { 1.09} {}-1.62, 3.79 & { 0.12} {}-2.49, 2.74 & { 0.82} {}-2.34, 3.99\\
\hline
 MMA\textsuperscript{b)} & { {}-0.92} {}-2.87, 1.03 & { 0.29} {}-2.19, 2.76 & { 0.70} {}-2.06, 3.47 & { {}-0.56} {}-3.88, 2.77\\
\hline
\end{tabular}
\end{adjustbox}
\end{sidewaystable}

% TABLE 5.3 HERE
\begin{table}
\small
\caption{Associations between plasma B vitamins and LINE-1 methylation according to family history using multivariable linear regression.}
\label{table5_3}
\begin{adjustbox}{width=\textwidth}
\begin{tabular}{C{3cm}C{4.5cm}C{4.6cm}}

\hline ~ & \bfseries Family history of CRC (n=81) & % \centering\arraybslash % arraybslash doesn't work so comment out
\bfseries No family history of CRC (n=200)\\
\hline ~ & { Multivariable adjusted\textsuperscript{a)}}

{ $\beta $\textsuperscript{c)}}

 95\% confidence interval & { Multivariable adjusted\textsuperscript{a)}}

{ $\beta $\textsuperscript{c)}}

 95\% confidence interval\\
\hline
 Folate\textsuperscript{b)} & { {}-2.77} {}-5.57, 0.02 & { {}-1.02} {}-2.32, 0.29\\
\hline
 Methionine\textsuperscript{b)} & { {}-2.37} {}-8.03, 3.30 & { 3.91} 0.39, 7.42\\
\hline
 Riboflavin\textsuperscript{b)} & { {}-1.30} {}-3.66, 1.06 & { {}-0.17} {}-1.36, 1.02\\
\hline
 Vitamin B6 sum\textsuperscript{b)} & { 1.88} {}-1.42, 5.17 & { {}-0.27} {}-1.93, 1.39\\
\hline
 PLP\textsuperscript{b)} & { 1.84} {}-1.79, 5.47 & { 0.20} {}-1.52, 1.91\\
\hline
 PL\textsuperscript{b)} & { 0.53} {}-1.95, 3.01 & { {}-0.83} {}-2.27, 0.61\\
\hline
 PA\textsuperscript{b)} & { 1.15} {}-1.15, 3.44 & { {}-0.69} {}-1.99, 0.61\\
\hline
 Cobalamin\textsuperscript{b)} & { 4.90} {}-1.03, 10.82 & { 0.35} {}-2.22, 2.92\\
\hline
 MMA\textsuperscript{b)} & { 2.26} {}-1.90, 6.43 & { {}-0.93} {}-3.31, 1.45\\
\hline
\end{tabular}
\end{adjustbox}
\end{table}

% TABLE 5.4 HERE
\begin{table}
\small
\caption{Associations between plasma B vitamins and LINE-1 methylation according to \emph{MTHFR} C677T genotype.}
\label{table5_4}
\begin{adjustbox}{width=\textwidth}
\begin{tabular}{C{2.4cm}C{3.5cm}C{3.5cm}C{3.2cm}}

\hline ~ & \multicolumn{3}{c}{\centering
 \textbf{\textit{MTHFR}}\textbf{ C677T
genotype}}\\ ~ & \textbf{\textit{CC}} (n=125) & \textbf{\textit{CT}} (n=126) & \textbf{\textit{TT}} (n=25)\\
\hline ~ & { Multivariable adjusted\textsuperscript{a)}}{ $\beta $\textsuperscript{c)}} 95\% confidence interval & { Multivariable adjusted\textsuperscript{a)}}{ $\beta $\textsuperscript{c)}} 95\% confidence interval & { Multivariable adjusted\textsuperscript{a)}}{ $\beta $\textsuperscript{c)}} 95\% confidence interval\\
\hline
 Folate\textsuperscript{b)} & { {}-2.01} {}-3.77, -0.25 & { {}-1.56} {}-3.57, 0.46 & { 5.48} {}-0.60, 11.56\\
\hline
 Methionine\textsuperscript{b)} & { 2.13} {}-2.93, 7.18 & { 0.49} {}-4.20, 5.18 & { 1.29} {}-11.46, 14.05\\
\hline
 Riboflavin\textsuperscript{b)} & { {}-0.75} {}-2.52, 1.01 & { {}-0.12} {}-1.72, 1.48 & { {}-9.08} {}-15.47, -2.69\\
\hline
 Vitamin B6 sum\textsuperscript{b)} & { 0.97} {}-1.29, 3.23 & { {}-0.09} {}-2.66, 2.47 & { 12.00} 0.85, 23.15\\
\hline
 PLP\textsuperscript{b)} & { 1.32} {}-1.09, 3.72 & { 0.29} {}-2.49, 3.08 & { 9.63} {}-2.76, 22.01\\
\hline
 PL\textsuperscript{b)} & { 0.13} {}-1.77, 2.03 & { {}-0.73} {}-2.79, 1.34 & { 7.25} {}-2.50, 16.99\\
\hline
 PA\textsuperscript{b)} & { 0.33} {}-1.29, 1.95 & { {}-0.56} {}-2.45, 1.33 & { 4.54} {}-6.14, 15.22\\
\hline
 Cobalamin\textsuperscript{b)} & { 0.01} {}-4.00, 4.03 & { 2.46} {}-0.90, 5.82 & { {}-5.68} {}-20.67, 9.31\\
\hline
 MMA\textsuperscript{b)} & { 0.08} {}-3.09, 3.24 & { {}-0.28} {}-3.32, 2.77 & { {}-9.22} {}-19.51, 1.07\\
\hline
\end{tabular}
\end{adjustbox}
\end{table}

% TABLE 5.5 HERE
\begin{table}
\caption{Associations between plasma folate and LINE-1 methylation stratified by plasma riboflavin, smoking status, and alcohol intake using a multivariable regression model.}
\label{table5_5}
\begin{adjustbox}{width=\textwidth}
\begin{tabular}{C{5.5cm}C{6.5cm}}

\hline ~ & %\centering\arraybslash % i think the arraybslash is annoying latex
\textbf{Association between plasma folate}\textsuperscript{b)}
\textbf{and LINE-1 methylation}\\
\hline ~ & { Multivariable adjusted\textsuperscript{a)}}

{ $\beta $\textsuperscript{c)}}

 95\%confidence interval\\
\hline
\bfseries Plasma riboflavin & ~ \\
\hline
 1 (2.48-12.34 nmol/L) (n=87) & { {}-3.84} {}-6.27, -1.40\\
\hline
 2 (12.35-24.19 nmol/L) (n=88) & { {}-1.16} {}-3.16, 0.85\\
\hline
 \foreignlanguage{dutch}{3
(24.20-329.76 nmol/L) (n=87)} & { {}-0.24} {}-2.10, 2.57\\
\hline
\bfseries Smoking status & ~ \\
\hline
 Never (n=75) & { {}-3.19} {}-6.23, -0.15\\
\hline
 Former (n=160) & { {}-1.38} {}-2.91, 0.16\\
\hline
 Current (n=35) & { {}-0.14} {}-2.82, 2.53\\
\hline
\bfseries Alcohol intake & ~ \\
\hline
 1 (0.00-4.38 g/day) (n=90) & { {}-0.41} {}-2.73, 1.90\\
\hline
 2 (4.39-17.90 g/day) (n=90) & { {}-3.56} {}-5.88, -1.04\\
\hline
 3 (17.91-143.50 g/day) (n=90) & { {}-1.32} {}-3.37, 0.73\\
\hline
\end{tabular}
\end{adjustbox}
\end{table}


\section[]{Discussion} % level 1
\noindent In the present cohort study with patients, who have had a history of at least one colorectal adenoma ever in their life, we observed inverse associations between plasma folate and LINE-1 methylation from peripheral blood leukocytes. An increase in plasma folate concentrations was related to a decrease in LINE-1 methylation in those with one adenoma and in those with at least two lifetime adenomas. Our results are somewhat unexpected, as we hypothesised that higher plasma folate concentrations would be associated with higher LINE-1 methylation, as folate is important for the availability of \emph{S}-adenosylmethionine, the universal methyl donor required for DNA methylation. On the other hand, there is growing evidence that high levels of folate may be protective in early carcinogenesis but may accelerate the growth of established colorectal neoplasia, as mentioned earlier. The complex dual role of folate in colorectal carcinogenesis depends on the timing of exposure in addition to dose and duration 
\cite{c541,c542,c550,c551}. In our population of exclusively adenoma patients, it may not be too surprising that plasma folate was inversely related to LINE-1 methylation in leukocytes, as increasing levels of plasma folate could promote adenoma growth, where increased levels of global DNA hypomethylation exist.

\noindent Results from a cross-sectional study with healthy Dutch participants showed no association between plasma folate and global DNA methylation, as measured by liquid chromatography-tandem mass spectrometry \cite{c529}, and plasma folate concentrations in this study were similar to those measured by us. In controlled folate feeding trials with healthy, postmenopausal women, folate status was positively associated with global DNA methylation levels in leukocytes \cite{c526,c527}. In these feeding trials, female volunteers consumed low-folate diets for 7 weeks \cite{c526} or 91 days \cite{c527} showing a concomitant gradual loss of global DNA methylation.

\noindent Furthermore, five randomised controlled trials have explored the effects of folic acid, the synthetic form of folate, on global DNA methylation in colorectal tissues \cite{c533,c534,c535,c536,c552} and leukocytes \cite{c552} of colorectal adenoma patients. While the aim of these studies was not to examine the relationship between plasma folate and global DNA methylation, they are relevant, as they demonstrate the influence of folic acid on changes in global DNA methylation levels. Following folic acid supplementation, three of these studies observed an increase in global DNA methylation \cite{c533,c536,c552}, while results were null in others \cite{c534,c535}.

\noindent Possible reasons for inconsistent results include differences in study design. Selection biases may occur in case-control studies but are unlikely in cohort studies, where loss to follow-up is not related to exposure. There are also differences in methods of measuring DNA methylation, type of tissue studied (colon \emph{vs}. leukocytes), study participants (healthy \emph{vs}. cancer patients), and adjustments for confounding factors. Furthermore, many of the aforementioned studies were conducted in the United States where folic acid fortification is mandatory, and subsequently plasma concentrations of study participants were relatively high.

\noindent The small RCT performed by Pufulete \emph{et al}. in the United Kingdom, where folic acid fortification is not compulsory, is interesting because global DNA methylation was measured in both leukocytes as well as colon tissue in adenoma patients \cite{c552}. Subjects were randomised to receive 400 $\mu$g/day folic acid for a short period of time (10 weeks), and there was a borderline significant increase in global DNA methylation in leukocytes (P=0.05), measured using [\textsuperscript{3}H] methyl incorporation assay, in the intervention group compared to those in the placebo group following supplementation. There was also a borderline significant increase in global DNA methylation in colon tissue following folic acid intervention (P=0.09), and the weaker increase in colon tissue may be a consequence of blood being ``closer'' than colon tissue to environmental exposures \cite{c553}. Additionally, the absorption and metabolism of natural folates and synthetic folates, i.e. folic acid, are different, 
and may therefore have dissimilar effects on one carbon metabolism \cite{c554,c555}.

\noindent Methionine is an essential amino acid, and like folate, it is required for the synthesis of \emph{S}-adenosylmethionine, and as such, we would expect a positive relationship between methionine and LINE-1 methylation. Indeed, we found a positive association between plasma methionine and LINE-1 methylation in patients without a family history of CRC. For those with a family history of CRC, there was no significant association between plasma methionine and LINE-1 methylation perhaps indicating that family history of CRC may modulate the associations between plasma methionine and LINE-1 methylation, although it may be premature to draw definitive conclusions based on a small subgroup analysis.

\noindent The significant inverse relationship in the current study between plasma folate and LINE-1 methylation in patients with the \emph{MTHFR} 677CC genotype is in contrast to those with the \emph{MTHFR} 677CT or \emph{MTHFR} 677TT genotype, where there was no association. Based on the premise that the C to T base transition resulting in an alanine to valine substitution impairs enzyme activity, and decreases plasma folate concentrations \cite{c523,c556} and that folate is required for the production of SAM for DNA methylation, we had expected direct associations between plasma folate and LINE-1 methylation in persons with the TT genotype. Our results are inconsistent with the results from an observational Italian study using cross-sectional data, which revealed lower global DNA methylation in subjects with the TT genotype compared to those with the CC genotype when plasma folate concentrations were low \cite{c557}. However, results from controlled folate feeding trials have also been unexpected. No 
changes in global leukocyte DNA methylation following short-term folate restriction in healthy women with the \emph{MTHFR} 677CC genotype have been reported \cite{c528}, while other studies in women have demonstrated changes in global DNA methylation in those with the TT genotype following folate repletion \cite{c558,c559}. Results from the feeding trials and our own results in the current study should be interpreted with prudence, as there were relatively few patients with the \emph{MTHFR} 677TT genotype.

\noindent Results from additional exploratory analyses indicated significant inverse associations between plasma folate and LINE-1 methylation for those in the lowest category of plasma riboflavin, those who had never smoked and those in the middle category for alcohol intake. We had expected that smoking and alcohol would decrease concentrations of B vitamins and subsequently decrease LINE-1 methylation.

\noindent Our study possesses several strengths. It is, to our knowledge, the first to provide a snapshot of the relationship between plasma B vitamins and global DNA methylation in patients with a history of adenomas. Our study population is fairly large and exclusively includes subjects with a history of colorectal adenomas, and is therefore at higher risk for developing colorectal cancer. We were able to measure plasma B vitamins rather than assessing exposure using recall methods such as an FFQ.

\noindent This study also has some limitations. Cross-sectional data does not allow us to make causal inferences from associations between plasma analytes and LINE-1 methylation. It may be of interest for future studies to include measurements of \emph{S}-adenosylmethionine and \emph{S}-adenosylhomocysteine, and to take repeated blood measurements at different time points. Red blood cell folate concentrations may represent a better long-term measure of folate status, but a single measurement of plasma folate per subject seems to be sufficient in large epidemiological studies \cite{c560}. Although we have measured LINE-1 methylation, we do not have information about specific promoter methylation, and we cannot eliminate the possibility that the correlations between plasma B vitamins and promoter methylation may differ in those with different number of adenomas, family history of colorectal cancer, \emph{MTHFR} C677T genotype, smoking, and alcohol intake. CpG island promoter methylation could, in fact, be more 
frequent among individuals with lower LINE-1 methylation, a proxy for global DNA methylation \cite{c561}. Furthermore, to date, while we are unaware of studies having definitively proven the relationship between leukocyte methylation and methylation in colon tissue, the results from Pufulete \emph{et al}.~\cite{c552} do suggest similarities in DNA methylation between the two tissues in response to folic acid. Additionally, because obtaining target tissues in humans is so often problematic and complicated, global DNA methylation in leukocytes are commonly used as a substitute for target tissues in epidemiological studies  \cite{c57,c510}.

\noindent For our analyses, we used the follow-up FFQ and blood sample, so indeed, some CRA patients could have been diagnosed over a decade prior to measurement of B-vitamin status and LINE-1 methylation. However, in the original case-control study, few cases could confirm that they changed their diets when asked about dietary changes after diagnosis of CRA \cite{c562}, and certainly while LINE-1 methylation levels in leukocytes have changed since diagnosis of CRA, we were interested in capturing the relationships between plasma B vitamins and methionine and LINE-1 methylation in adenoma patients during follow-up. We would reason then that the fact that measurements of B-vitamins and LINE-1 methylation were done after diagnosis of CRA would have little implication for the interpretation of our findings.

\noindent It is unclear why there is an inverse association between plasma folate and LINE-1 methylation. Patients with colorectal adenomas may experience changes in their one-carbon metabolism and may be more sensitive to these nutrients. Nevertheless, in light of the current unexpected results, more research is warranted to further elucidate the true relationships between B vitamins and global DNA methylation in healthy individuals as well as in individuals at high risk for developing colorectal cancer.

\noindent In conclusion, our study suggests that plasma folate is inversely associated with LINE-1 methylation in leukocytes of colorectal adenoma patients. Plasma methionine was positively associated with LINE-1 methylation. We also observed indications that these relationships may be potentially modified by family history of colorectal cancer, and \emph{MTHFR} 677 C $\rightarrow$ T genotype although drawing definitive conclusions may be difficult due to relatively small sample sizes in our subgroup analyses.

\begin{thebibliography}{12} 
	\bibitem{c51}	Robertson KD. DNA methylation, methyltransferases, and cancer. Oncogene. 2001 May 28;20(24):3139-55. 
	\bibitem{c52}	Herman JG, Baylin SB. Gene silencing in cancer in association with promoter hypermethylation. The New England Journal of Medicine. 2003 Nov 20;349(21):2042-54. 
	\bibitem{c53}	Feinberg AP, Vogelstein B. Hypomethylation distinguishes genes of some human cancers from their normal counterparts. Nature. 1983 Jan 6;301(5895):89-92. 
	\bibitem{c54}	Jones PA, Baylin SB. The epigenomics of cancer. Cell. 2007 Feb 23;128(4):683-92. 
	\bibitem{c55}	Liu JJ, Ward RL. Folate and one-carbon metabolism and its impact on aberrant DNA methylation in cancer. Advances in Genetics.71:79-121. 
	\bibitem{c56}	Yang AS, Estecio MR, Doshi K, Kondo Y, Tajara EH, Issa JP. A simple method for estimating global DNA methylation using bisulfite PCR of repetitive DNA elements. Nucleic Acids Research. 2004;32(3):e38. 
	\bibitem{c57}	Lim U, Flood A, Choi SW, Albanes D, Cross AJ, Schatzkin A, et al. Genomic methylation of leukocyte DNA in relation to colorectal adenoma among asymptomatic women. Gastroenterology. 2008 Jan;134(1):47-55. 
	\bibitem{c58}	Pufulete M, Al-Ghnaniem R, Leather AJ, Appleby P, Gout S, Terry C, et al. Folate status, genomic DNA hypomethylation, and risk of colorectal adenoma and cancer: a case control study. Gastroenterology. 2003 May;124(5):1240-8. 
	\bibitem{c59}	Moore LE, Pfeiffer RM, Poscablo C, Real FX, Kogevinas M, Silverman D, et al. Genomic DNA hypomethylation as a biomarker for bladder cancer susceptibility in the Spanish Bladder Cancer Study: a case-control study. The Lancet Oncology. 2008 Apr;9(4):359-66. 
	\bibitem{c510}	Bjornsson HT, Sigurdsson MI, Fallin MD, Irizarry RA, Aspelund T, Cui H, et al. Intra-individual change over time in DNA methylation with familial clustering. Jama. 2008 Jun 25;299(24):2877-83. 
	\bibitem{c511}	Zhu ZZ, Sparrow D, Hou L, Tarantini L, Bollati V, Litonjua AA, et al. Repetitive element hypomethylation in blood leukocyte DNA and cancer incidence, prevalence, and mortality in elderly individuals: the Normative Aging Study. Cancer Causes Control.  Mar;22(3):437-47. 
	\bibitem{c512}	Fearon ER, Vogelstein B. A genetic model for colorectal tumorigenesis. Cell. 1990 Jun 1;61(5):759-67. 
	\bibitem{c513}	Hill MJ, Morson BC, Bussey HJ. Aetiology of adenoma--carcinoma sequence in large bowel. Lancet. 1978 Feb 4;1(8058):245-7. 
	\bibitem{c514}	Neugut AI, Jacobson JS, De Vivo I. Epidemiology of colorectal adenomatous polyps. Cancer Epidemiol Biomarkers Prev. 1993 Mar-Apr;2(2):159-76. 
	\bibitem{c515}	Mason JB. Biomarkers of nutrient exposure and status in one-carbon (methyl) metabolism. The Journal of nutrition. 2003 Mar;133 Suppl 3:941S-7S. 
	\bibitem{c516}	Trimble KC, Molloy AM, Scott JM, Weir DG. The effect of ethanol on one-carbon metabolism: increased methionine catabolism and lipotrope methyl-group wastage. Hepatology (Baltimore, Md. 1993 Oct;18(4):984-9. 
	\bibitem{c517}	Ulvik A, Ebbing M, Hustad S, Midttun O, Nygard O, Vollset SE, et al. Long- and short-term effects of tobacco smoking on circulating concentrations of B vitamins. Clinical chemistry.  May;56(5):755-63. 
	\bibitem{c518}	Ma J, Stampfer MJ, Giovannucci E, Artigas C, Hunter DJ, Fuchs C, et al. Methylenetetrahydrofolate reductase polymorphism, dietary interactions, and risk of colorectal cancer. Cancer research. 1997 Mar 15;57(6):1098-102. 
	\bibitem{c519}	Marugame T, Tsuji E, Kiyohara C, Eguchi H, Oda T, Shinchi K, et al. Relation of plasma folate and methylenetetrahydrofolate reductase C677T polymorphism to colorectal adenomas. International journal of epidemiology. 2003 Feb;32(1):64-6. 
	\bibitem{c520}	Eussen SJ, Vollset SE, Igland J, Meyer K, Fredriksen A, Ueland PM, et al. Plasma folate, related genetic variants, and colorectal cancer risk in EPIC. Cancer Epidemiol Biomarkers Prev.  May;19(5):1328-40. 
	\bibitem{c521}	Van Guelpen B, Hultdin J, Johansson I, Hallmans G, Stenling R, Riboli E, et al. Low folate levels may protect against colorectal cancer. Gut. 2006 Oct;55(10):1461-6. 
	\bibitem{c522}	Chen J, Ma J, Stampfer MJ, Palomeque C, Selhub J, Hunter DJ. Linkage disequilibrium between the 677C>T and 1298A>C polymorphisms in human methylenetetrahydrofolate reductase gene and their contributions to risk of colorectal cancer. Pharmacogenetics. 2002 Jun;12(4):339-42. 
	\bibitem{c523}	van der Put NM, Steegers-Theunissen RP, Frosst P, Trijbels FJ, Eskes TK, van den Heuvel LP, et al. Mutated methylenetetrahydrofolate reductase as a risk factor for spina bifida. Lancet. 1995 Oct 21;346(8982):1070-1. 
	\bibitem{c524}	Guenther BD, Sheppard CA, Tran P, Rozen R, Matthews RG, Ludwig ML. The structure and properties of methylenetetrahydrofolate reductase from Escherichia coli suggest how folate ameliorates human hyperhomocysteinemia. Nature structural biology. 1999 Apr;6(4):359-65. 
	\bibitem{c525}	van den Donk M, Buijsse B, van den Berg SW, Ocke MC, Harryvan JL, Nagengast FM, et al. Dietary intake of folate and riboflavin, MTHFR C677T genotype, and colorectal adenoma risk: a Dutch case-control study. Cancer Epidemiol Biomarkers Prev. 2005 Jun;14(6):1562-6. 
	\bibitem{c526}	Rampersaud GC, Kauwell GP, Hutson AD, Cerda JJ, Bailey LB. Genomic DNA methylation decreases in response to moderate folate depletion in elderly women. The American journal of clinical nutrition. 2000 Oct;72(4):998-1003. 
	\bibitem{c527}	Jacob RA, Gretz DM, Taylor PC, James SJ, Pogribny IP, Miller BJ, et al. Moderate folate depletion increases plasma homocysteine and decreases lymphocyte DNA methylation in postmenopausal women. The Journal of nutrition. 1998 Jul;128(7):1204-12. 
	\bibitem{c528}	Axume J, Smith SS, Pogribny IP, Moriarty DJ, Caudill MA. Global leukocyte DNA methylation is similar in African American and Caucasian women under conditions of controlled folate intake. Epigenetics. 2007 Jan-Mar;2(1):66-8. 
	\bibitem{c529}	Kok RM, Smith DE, Barto R, Spijkerman AM, Teerlink T, Gellekink HJ, et al. Global DNA methylation measured by liquid chromatography-tandem mass spectrometry: analytical technique, reference values and determinants in healthy subjects. Clin Chem Lab Med. 2007;45(7):903-11. 
	\bibitem{c530}	Loeve F, van Ballegooijen M, Boer R, Kuipers EJ, Habbema JD. Colorectal cancer risk in adenoma patients: a nation-wide study. International journal of cancer. 2004 Aug 10;111(1):147-51. 
	\bibitem{c531}	Martinez ME, Baron JA, Lieberman DA, Schatzkin A, Lanza E, Winawer SJ, et al. A pooled analysis of advanced colorectal neoplasia diagnoses after colonoscopic polypectomy. Gastroenterology. 2009 Mar;136(3):832-41. 
	\bibitem{c532}	Butterworth AS, Higgins JP, Pharoah P. Relative and absolute risk of colorectal cancer for individuals with a family history: a meta-analysis. Eur J Cancer. 2006 Jan;42(2):216-27. 
	\bibitem{c533}	Cravo M, Fidalgo P, Pereira AD, Gouveia-Oliveira A, Chaves P, Selhub J, et al. DNA methylation as an intermediate biomarker in colorectal cancer: modulation by folic acid supplementation. Eur J Cancer Prev. 1994 Nov;3(6):473-9. 
	\bibitem{c534}	Cravo ML, Pinto AG, Chaves P, Cruz JA, Lage P, Nobre Leitao C, et al. Effect of folate supplementation on DNA methylation of rectal mucosa in patients with colonic adenomas: correlation with nutrient intake. Clinical nutrition (Edinburgh, Scotland). 1998 Apr;17(2):45-9. 
	\bibitem{c535}	Figueiredo JC, Grau MV, Wallace K, Levine AJ, Shen L, Hamdan R, et al. Global DNA hypomethylation (LINE-1) in the normal colon and lifestyle characteristics and dietary and genetic factors. Cancer Epidemiol Biomarkers Prev. 2009 Apr;18(4):1041-9. 
	\bibitem{c536}	Kim YI, Baik HW, Fawaz K, Knox T, Lee YM, Norton R, et al. Effects of folate supplementation on two provisional molecular markers of colon cancer: a prospective, randomized trial. The American journal of gastroenterology. 2001 Jan;96(1):184-95. 
	\bibitem{c537}	Pufulete M, Al-Ghnaniem R, Rennie JA, Appleby P, Harris N, Gout S, et al. Influence of folate status on genomic DNA methylation in colonic mucosa of subjects without colorectal adenoma or cancer. British journal of cancer. 2005 Mar 14;92(5):838-42. 
	\bibitem{c538}	Basten GP, Duthie SJ, Pirie L, Vaughan N, Hill MH, Powers HJ. Sensitivity of markers of DNA stability and DNA repair activity to folate supplementation in healthy volunteers. British journal of cancer. 2006 Jun 19;94(12):1942-7. 
	\bibitem{c539}	Fenech M, Aitken C, Rinaldi J. Folate, vitamin B12, homocysteine status and DNA damage in young Australian adults. Carcinogenesis. 1998 Jul;19(7):1163-71. 
	\bibitem{c540}	Jung AY, Smulders Y, Verhoef P, Kok FJ, Blom H, Kok RM, et al. No effect of folic acid supplementation on global DNA methylation in men and women with moderately elevated homocysteine. PloS one.6(9):e24976. 
	\bibitem{c541}	Ulrich CM, Potter JD. Folate and cancer--timing is everything. Jama. 2007 Jun 6;297(21):2408-9. 
	\bibitem{c542}	Cole BF, Baron JA, Sandler RS, Haile RW, Ahnen DJ, Bresalier RS, et al. Folic acid for the prevention of colorectal adenomas: a randomized clinical trial. Jama. 2007 Jun 6;297(21):2351-9. 
	\bibitem{c543}	Tiemersma EW, Wark PA, Ocke MC, Bunschoten A, Otten MH, Kok FJ, et al. Alcohol consumption, alcohol dehydrogenase 3 polymorphism, and colorectal adenomas. Cancer Epidemiol Biomarkers Prev. 2003 May;12(5):419-25. 
	\bibitem{c544}	Verkleij-Hagoort AC, de Vries JH, Stegers MP, Lindemans J, Ursem NT, Steegers-Theunissen RP. Validation of the assessment of folate and vitamin B12 intake in women of reproductive age: the method of triads. European journal of clinical nutrition. 2007 May;61(5):610-5. 
	\bibitem{c545}	Kelleher BP, Broin SD. Microbiological assay for vitamin B12 performed in 96-well microtitre plates. Journal of clinical pathology. 1991 Jul;44(7):592-5. 
	\bibitem{c546}	Molloy AM, Scott JM. Microbiological assay for serum, plasma, and red cell folate using cryopreserved, microtiter plate method. Methods in enzymology. 1997;281:43-53. 
	\bibitem{c547}	Windelberg A, Arseth O, Kvalheim G, Ueland PM. Automated assay for the determination of methylmalonic acid, total homocysteine, and related amino acids in human serum or plasma by means of methylchloroformate derivatization and gas chromatography-mass spectrometry. Clinical chemistry. 2005 Nov;51(11):2103-9. 
	\bibitem{c548}	Midttun O, Hustad S, Solheim E, Schneede J, Ueland PM. Multianalyte quantification of vitamin B6 and B2 species in the nanomolar range in human plasma by liquid chromatography-tandem mass spectrometry. Clinical chemistry. 2005 Jul;51(7):1206-16. 
	\bibitem{c549}	Frosst P, Blom HJ, Milos R, Goyette P, Sheppard CA, Matthews RG, et al. A candidate genetic risk factor for vascular disease: a common mutation in methylenetetrahydrofolate reductase. Nature genetics. 1995 May;10(1):111-3. 
	\bibitem{c550}	Kim YI. Role of folate in colon cancer development and progression. The Journal of nutrition. 2003 Nov;133(11 Suppl 1):3731S-9S. 
	\bibitem{c551}	Ulrich CM, Potter JD. Folate supplementation: too much of a good thing? Cancer Epidemiol Biomarkers Prev. 2006 Feb;15(2):189-93. 
	\bibitem{c552}	Pufulete M, Al-Ghnaniem R, Khushal A, Appleby P, Harris N, Gout S, et al. Effect of folic acid supplementation on genomic DNA methylation in patients with colorectal adenoma. Gut. 2005 May;54(5):648-53. 
	\bibitem{c553}	McKay JA, Xie L, Harris S, Wong YK, Ford D, Mathers JC. Blood as a surrogate marker for tissue-specific DNA methylation and changes due to folate depletion in post-partum female mice. Molecular nutrition \& food research.  Jul;55(7):1026-35. 
	\bibitem{c554}	Hubner RA, Houlston RS. Folate and colorectal cancer prevention. British journal of cancer. 2009 Jan 27;100(2):233-9. 
	\bibitem{c555}	Tibbetts AS, Appling DR. Compartmentalization of Mammalian folate-mediated one-carbon metabolism. Annual review of nutrition.  Aug 21;30:57-81. 
	\bibitem{c556}	Yamada K, Chen Z, Rozen R, Matthews RG. Effects of common polymorphisms on the properties of recombinant human methylenetetrahydrofolate reductase. Proceedings of the National Academy of Sciences of the United States of America. 2001 Dec 18;98(26):14853-8. 
	\bibitem{c557}	Friso S, Choi SW, Girelli D, Mason JB, Dolnikowski GG, Bagley PJ, et al. A common mutation in the 5,10-methylenetetrahydrofolate reductase gene affects genomic DNA methylation through an interaction with folate status. Proceedings of the National Academy of Sciences of the United States of America. 2002 Apr 16;99(8):5606-11. 
	\bibitem{c558}	Shelnutt KP, Kauwell GP, Gregory JF, 3rd, Maneval DR, Quinlivan EP, Theriaque DW, et al. Methylenetetrahydrofolate reductase 677C$\rightarrow$T polymorphism affects DNA methylation in response to controlled folate intake in young women. The Journal of nutritional biochemistry. 2004 Sep;15(9):554-60. 
	\bibitem{c559}	Axume J, Smith SS, Pogribny IP, Moriarty DJ, Caudill MA. The MTHFR 677TT genotype and folate intake interact to lower global leukocyte DNA methylation in young Mexican American women. Nutrition research (New York, NY. 2007 Jan;27(1):1365-17. 
	\bibitem{c560}	Drogan D, Klipstein-Grobusch K, Wans S, Luley C, Boeing H, Dierkes J. Plasma folate as marker of folate status in epidemiological studies: the European Investigation into Cancer and Nutrition (EPIC)-Potsdam study. The British journal of nutrition. 2004 Sep;92(3):489-96. 
	\bibitem{c561}	Esteller M. Epigenetics in cancer. The New England journal of medicine. 2008 Mar 13;358(11):1148-59. 
	\bibitem{c562}	Tijhuis MJ, Wark PA, Aarts JM, Visker MH, Nagengast FM, Kok FJ, et al. GSTP1 and GSTA1 polymorphisms interact with cruciferous vegetable intake in colorectal adenoma risk. Cancer Epidemiol Biomarkers Prev. 2005 Dec;14(12):2943-51. 
\end{thebibliography} 
