\chapter*{Samenvatting}
\chaptermark{Samenvatting}
\label{samenvatting}
\addcontentsline{toc}{chapter}{Samenvatting}
%\AddEverypageHook{%
%	\ifthenelse{\equal{\Chaptername}{Samenvatting}}{
%		\ifthenelse{\isodd{\thepage}}%
%		{\TileWallPaper{17.4cm}{24.4cm}{thumbSa.pdf}}%
%		{\ClearWallPaper}
%	} {\ClearWallPaper}
%} 


\newpage

\noindent Per jaar krijgen meer dan \'e\'en miljoen mensen dikke darmkanker. Daarmee is dikke darm-kanker de derde meest voorkomende vorm van kanker in de wereld. Diverse klinische, genetische, en voedingsfactoren kunnen het risico op dikke darmkanker be\"invloeden.

\noindent Dikke darmkanker bij \'e\'en of meer eerstegraads familie leden verdubbelt het persoonlijk risico op dikke darmkanker. Mensen met eerdere dikke darmpoliepen (adenoom), een niet-kwaadaardig voorstadium van dikke darmkanker, hebben eveneens een verhoogt risico op het ontwikkelen van dikke darmkanker. Dit is vooral het geval als het adenoom een villi vorm heeft, groter is dan 1 cm, of meervoudig aanwezig is.

\noindent Ook kan dikke darmkanker erfelijk zijn. Lynch syndroom is een bekende erfelijke vorm van kanker. Mensen met het syndroom van Lynch erven een pathogene germline mutatie in \'e\'en van de mismatch repair (MMR) genen. Deze mutatie in een MMR gen zorgt voor microsatelliet instabiliteit bij tumoren. Het risico op het ontwikkelen van dikke darmkanker is in de algemene bevolking 5-6\%. Mensen met het Lynch syndroom hebben een risico van 70-85\% op het ontwikkelen van dikke darmkanker voor hun zeventigste jaar.

\noindent Naast klinische en genetische factoren be\"invloeden voeding en andere levensstijlfactoren eveneens het risico op dikke darmkanker. Dit gebeurt mogelijk in samenspel met een verhoogde genetische gevoeligheid voor deze factoren als gevolg van specifiek varianten in genen betrokken bij het metabolisme, zoals het gen dat codeert voor het enzym MTHFR. Het World Cancer Research Fund/American Institute for Cancer Research (WCRF/AICR) schatten dat ruwweg \'e\'en derde, en in sommige landen zelfs tot \'e\'en half, van alle dikke darmkanker kan worden voorkomen door regelmatige lichaamsbeweging, beperken van lichaamsvet binnen de standaard normen van lichaamsgewicht, en gezonde voeding. Micronutri\"enten, zoals foliumzuur en andere B vitamines, dragen ook bij aan het verlagen van het risico op het ontwikkelen van dikke darm kanker.

\noindent Tijdens de ontwikkeling van dikke darmkanker ontstaat een afwijkende DNA methylering in de tumor. In dikke darm tumoren is zowel globale hypomethylering als gen specifieke hypermethylering van promotor regios gevonden. Deze beide processen zijn afhankelijk van de effici\"entie van het zogenaamde \'e\'en-koolstof (1-C) metabolisme. In dit metabolisme spelen foliumzuur en andere B-vitamines een sleutelrol waarin foliumzuur \'e\'en-koolstof (methyl) groepen aanlevert en de vitamines B2, B6 en B12 cofactoren zijn voor biologische reacties. Methionine is een essentieel aminozuur dat wordt omgezet naar \emph{S}-adenosylmethionine, de universele donor van methyl groepen voor een verscheidenheid aan reacties, waaronder DNA. De studies die in dit proefschrift zijn beschreven zijn bedoeld om onze kennis te verbeteren over het effect van B-vitamine status op globale DNA methylering in personen met een laag tot hoog risico op dikke darmkanker, nl. 1) mensen waarbij nog geen darmpoliepen of darmkanker is gediagnosticeerd, 2) mensen met eerdere darmpoliepen en 3) mensen met Lynch Syndroom.

\noindent \textbf{Hoofdstuk 1} is een algemene inleiding over B vitaminen, DNA methylering en dikke darm-kanker. De inleiding bevat tevens een beschrijving van de uitgevoerde studies die onderdeel zijn van dit proefschrift.

\noindent \textbf{Hoofdstuk 2} is een literatuur overzicht over de impact van voeding op globale en genspecifieke DNA methylering bij verschillende vormen van kanker zoals onderzocht in epidemiologische studies.

\noindent \textbf{Hoofdstuk 3} is gebaseerd op data verzameld van personen met een laag risico voor het ontwikkelen van dikke darmkanker. Een cross-sectionele studie is uitgevoerd om de rol van B vitamines in plasma op methylering van LINE-1 elementen te bestuderen. Hiervoor zijn gegevens gebruikt van 1,142 willekeurig geselecteerde individuen binnen de Nijmegen Biomedische Studie. Individuen met uiterste plasma waarden van folaat (10de and 90ste percentielen) werden geselecteerd. Onze resultaten wijzen niet op een verschil in LINE-1 DNA methylering tussen personen in de 10de en 90ste percentielen van folaat in plasma. Leeftijd op het moment van bloed afname was omgekeerd evenredig gerelateerd aan LINE-1 methylering in beide extremen. De LINE-1 DNA methylering was lager in vrouwen dan mannen terwijl er met het \emph{MTHFR} C677T genotype geen associatie gevonden werd.

\noindent Het effect van lange termijn foliumzuur suppletie op globale DNA methylering is vervolgens onderzocht in een populatie met een gemiddeld risico voor dikke darm kanker, in individuen met een verhoogd plasma homocysteine (\textbf{Hoofdstuk 4}). Ook is onderzocht of het \emph{MTHFR} C677T genotype de associatie be\"invloedde. Van 818 deelnemers van een gerandomiseerd dubbelblinde placebo gecontroleerde trial werden 216 deelnemers geselecteerd, gestratificeerd op \emph{MTHFR} C677T genotype, leeftijd en rookgewoonten. De deelnemers slikten drie jaar lang foliumzuur (0.8 mg/dag; n = 105) of placebo pillen (n = 111). DNA methylering in witte bloedcellen en de folaat concentraties in rode bloedcellen en serum werden bepaald voor en na suppletie. Onze resultaten wijzen erop dat bij mannen en vrouwen met gematigd hyper-homocysteinemie, de globale DNA methylering niveaus in witte bloedcellen niet verhoogd worden door lange termijn foliumzuur suppletie.

\noindent \textbf{Hoofdstuk 5} beschrijft de associatie tussen plasma B vitamines en LINE-1 DNA methylering in witte bloedcellen van personen met een verhoogd risico op het ontwikkelen van dikke darmkanker, (i.e. individuen met tenminste \'e\'en histologisch bevestigde colorectale adenoom). Plasma folaat bleek omgekeerd evenredig geassocieerd met LINE-1 methylering in de gehele populatie van dikke darm adenoom pati\"enten. Plasma methionine bleek positief geassocieerd met LINE-1 methylering in personen met \'e\'en adenoom. Geen belangrijke verschillen werden gevonden wanneer de analyses werden gestratificeerd voor het aantal adenomas (\'e\'en \emph{vs} tenminste twee), familiale achtergrond van dikke darm kanker in een eerstegraads familielid, en \emph{MTHFR} C677T genotype.

\noindent \textbf{Hoofdstuk 6} presenteert de resultaten van  een prospectieve cohort studie (GeoLynch studie) bij 470 mensen met het Lynch syndroom, mensen met een heel hoog risico op het krijgen van dikke darmkanker. Hierin werd gekeken naar de associatie tussen het ontstaan van colorectale adenomen en de inname van folaat, vitamine B2, vitamine B6, vitamine B12 en methionine. Inname van de nutrienten werd bepaald door middel van een voedselfrequentievragenlijst. Cox regressie modellen met robuuste sandwich covariatie schatting, gecorrigeerd voor leeftijd, geslacht, lichamelijke activiteit, aantal colonoscopieen tijdens de studie en het gebruik van NSAIDs werden gebruikt om de hazard ratio (HR) en de bijbehorende 95\% btbhi te berekenen. De inname van B vitaminen en methionine uit de voeding leek niet geassocieerd te zijn met adenoom ontwikkeling in mensen met het Lynch Syndroom, ook niet wanneer werd gecorrigeerd voor leeftijd, geslacht, lichamelijke activiteit, aantal colonoscopieen tijdens de studie en het gebruik van NSAIDs.

\noindent Het laatste hoofdstuk (\textbf{Hoofdstuk 7}) bevat de algemene discussie, waarin de resultaten, zoals gepresenteerd in de voorgaande hoofdstukken, als geheel worden besproken en bediscussieerd. De algehele bevindingen van dit proefschrift duiden erop dat de status van folaat en andere B vitaminen mogelijk niet geassocieerd zijn met globale DNA methylering, noch met LINE-1 DNA methylering in witte bloedcellen, behalve mogelijk in personen met een voorgeschiedenis van dikke darm adenomen. Omdat zowel onze resultaten als die van anderen niet genoeg bewijs leveren dat een lage of hoge folaat status leukocyt DNA methylering kan be\"invloeden in individuen met een verschillend onderliggend risico op dikke darmkanker, is er geen reden om de huidige aanbevelingen om het risico op kanker te verlagen aan te passen. Omdat DNA methylering patronen niet statisch zijn, en veranderen tijdens het leven o.a. als reactie op verschillende omgevingsfactoren kan het informatief zijn om personen te volgen vanaf hun vroege levensjaren en om de tien jaar een epigenomisch-brede associatie studie uit te voeren. Tevens hebben wiskundige modellen en \emph{in vivo} studies aangetoond dat DNA methylering concurreert met DNA synthese. Een lange termijn gerandomiseerde gecontroleerde studie met foliumzuur suppletie in gezonde jonge mensen zou onze kennis vergroten over de impact van folium zuur op DNA synthese en methylering vanaf een jonge leeftijd.