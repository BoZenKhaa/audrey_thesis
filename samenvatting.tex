\chapter*[Samenvatting]{Samenvatting}
\label{samenvatting}
\addcontentsline{toc}{chapter}{Samenvatting}

\quad\\

\quad\\

\newpage

Per jaar krijgen meer dan ��n miljoen mensen een vorm van colon kanker of kanker van het rectum. Daarmee is dikke darmkanker de derde meest voorkomende vorm van kanker in de wereld. Diverse klinische, genetische, en voedingsfactoren kunnen het risico op dikke darmkanker be�nvloeden. Dikke darmkanker bij ��n of meer eerstegraads familie leden  verdubbelt het persoonlijk risico op dikke darmkanker. . Mensen met eerder dikke darmpoliepen (adenoman),  een niet-kwaadaardig voorstadium van  dikke darmkanker hebben eveneens een verhoogt  risico op het ontwikkelen van dikke darmkanker. Dit is vooral het geval als het adenoom een villi vorm heeft, groter is dan 1 cm, of meervoudig aanwezig is.  Ook kan dikke darmkanker erfelijk zijn. Lynch syndroom is een bekende erfelijke vorm van kanker. Mensen met het syndroom van Lynch erven een pathogene germline mutatie in ��n van de mismatch repair (MMR) genen. Deze mutatie in een MMR gen zorgt voor microsatelliet instabiliteit bij tumoren. Het risico op het ontwikkelen van dikke darmkanker is in de algemene bevolking  5-6\%. Mensen met het Lynch syndroom hebben een risico van 70-85\% op het ontwikkelen van dikke darmkanker voor hun zeventigste jaar . Voeding en andere leefstijlfactoren be�nvloeden ook het risico op het ontwikkelen van dikke darmkanker. Het World Cancer Research Fund/ American Institute for Cancer Research (WCRF/AICR) schatten dat ruwweg ��n derde, en in sommige landen zelfs tot ��n half, van alle dikke darm kanker kan worden voorkomen door regelmatige lichaamsbeweging, beperken van lichaamsvet binnen de standaard normen van lichaamsgewicht, en  gezonde voeding. Micronutri�nten, zoals foliumzuur en andere B vitamines, dragen ook bij aan het verlagen van het risico op het ontwikkelen van dikke darm kanker.

Tijdens de ontwikkeling van dikke darmkanker ontstaat een afwijkende DNA methylering in de tumor. In dikke darm tumoren is zowel globale hypomethylering als gen specifieke hypermethylering van promotor regios  gevonden. Deze beide processen zijn afhankelijk van de effici�ntie van het zogenaamde ��n-koolstof (1-C) metabolisme. In dit metabolisme spelen foliumzuur en andere B- vitamines een sleutelrol waarin foliumzuur ��n-koolstof (methyl)  groep aanlevert en  de vitamines B2, B6 en B12 cofactoren zijn voor biologische reacties. Methionine is een essentieel aminozuur dat wordt omgezet naar \emph{S}-adenosylmethionine, de universele donor van methyl groepen voor een verscheidenheid aan reacties, waaronder DNA. De studies die in dit proefschrift zijn beschreven zijn bedoeld om onze kennis te verbeteren over het effect van B- vitamine status op globale DNA methylering in personen  met een laag tot hoog risico op dikke darmkanker.

Hoofdstuk 1 is een algemene inleiding  over B vitaminen, DNA methylering en dikke darmkanker. De inleiding bevat tevens een beschrijving van de uitgevoerde studies die onderdeel zijn van dit proefschrift.
Hoofdstuk 2 is een literatuur overzicht  over de impact van voeding op globale en genspecifieke DNA methylering bij  verschillende vormen van kanker zoals onderzocht in epidemiologische studies. 
Hoofdstuk 3 is gebaseerd op data verzameld van personen met een laag risico voor het ontwikkelen van dikke darmkanker. Een cross-sectionele studie is uitgevoerd om de rol  van B vitamines in plasma op methylering van LINE-1 elementen te bestuderen.  Hiervoor zijn gegevens gebruikt van 1,142 willekeurig geselecteerde individuen binnen de Nijmeegse Biomedische Studie  Individuen met uiterste plasma waarden van folaat (10de (2.69-5.62 nmol/L) and 90ste (37.24-95.94 nmol/L) percentielen) werden geselecteerd. Van de totaal 275 kankervrije personen met plasma folaat concentraties in de 10de (n= 138) en 90ste (n=137) percentielen, is de LINE-1 DNA methylering in leukocyten vastgesteld en gerelateerd aan  de circulerende concentraties van methionine en B vitamines (folaat, riboflavine, vitamine B6 soorten (pyridoxaal-5�-fosfaat, pyridoxaal en 4-pyridoxide zuur), cobalamine en methylmalonic zuur (MMA)). Voedingsgewoonten en diverse leefstijlfactoren (o.a. roken, alcohol consumptie) , leeftijd, geslacht en MTHFR C677T genotype werden meegenomen in multivariabele lineaire regressie modellen.  Onze resultaten wijzen niet op een verschil in LINE-1DNA methylering tussen personen in de 10de en 90ste percentielen van folaat in plasma. Het gemiddelde percentage van LINE-1 DNA methylering (IQR) voor personen in de 10de percentiel was 73.7 (72.6-74.6) en 73.6 (72.5-74.7) voor personen in de 90ste percentiel.

Leeftijd op het moment van bloed afname was omgekeerd evenredig gerelateerd aan LINE-1 methylering in beide extremen (? schatting (95\%CI) of 0.07 (-0.13, -0.01) voor de 10de percentiel en -0.08 (-0.14, -0.02) voor de 90ste percentiel). De LINE-1 DNA methylering was lager in vrouwen in vergelijking tot  mannen in de , terwijl er met MTHFRC677T genotype geen associatie gevonden werd.

Het effect van lange termijn foliumzuur supplementatie  op globale DNA methylering is vervolgens onderzocht in een populatie met een gemiddeld risico voor dikke darm kanker, (Hoofdstuk 4), nl in individuen met een verhoogd plasma homocysteine.  of Ook is onderzocht of het MTHFRC677T genotype  de associatie beinvloedde. Van 818 deelnemers van een gerandomiseerd dubbelblinde placebo gecontroleerde trial werden 216 deelnemers geselecteerd, gestratificeerd op MTHFR C677T genotype, leeftijd en rookgewoonten  De deelnemers slikten drie jaar lang f foliumzuur (0.8 mg/dag; n = 105) �f placebo pillen (n = 111). DNA methylering  in witte bloedcellen en de folaat concentraties in rode bloedcellen en serum werden bepaald voor en na suppletie. Globale DNA methylering werd gemeten door middel van vloeistofchromatografie - massaspectrometrie  (LCMS) en uitgedrukt als een percentage van 5-methylcytosine versus de totale hoeveelheid cytosine. Na drie jaar suppletie met foliumzuur was er geen verschil in globale DNA methylering tussen  de twee groepen (verschil = 0.008, 95\%CI =20.05,0.07, P = 0.79). Ook was er geen verschil tussen de behandel groepen als die werden gestratificeerd voor MTHFR C677T genotype (CC, n = 76;
CT, n = 70; TT, n = 70), baseline rode bloedcel folaat status of baseline DNA methylering niveau. Onze resultaten wijzen erop dat bij mannen en vrouwen met gematigd hyper-homocysteinemie, de globale DNA methylering niveaus in witte bloedcellen niet verhoogt worden door lange termijn foliumzuur suppletie.

Hoofdstuk 5 beschrijft de associatie tussen plasma B vitamines en LINE-1 DNA methylering in witte bloedcellen van personen met een verhoogd risico op het ontwikkelen dikke darmkanker (i.e. individuen met tenminste ��n histologisch bevestigde colorectale adenoom). LINE-1 bisulfiet pyrosequencing  is gebruikt om de globale DNA methylering niveaus te meten  in witte bloedcellen van 281 mensen met eerdere colorectale adenomen . Plasma folaat bleek omgekeerd evenredig geassocieerd met LINE-1 methylering in de gehele populatie van dikke darm adenoom pati�nten ( ? (95\% betrouwbaarheidsinterval (btbhi) : - 1.46 (-2.65, -0.27)) in lineaire regressie modellen gecorrigeerd voor leeftijd, geslacht, BMI, alcohol inname, roken, familiale achtergrond van dikke darm kanker, en andere gemeten analyten (i.e. gemeenschappelijke B vitamines en methionine. Plasma methionine bleek positief geassocieerd met LINE-1 methylering (gecorrigeerde ? en 95\% btbhi: 3.47 (-0.40, 7.33)) in personen met ��n adenoom.   Geen belangrijke verschillen werden gevonden wanneer de analyses werden  gestratificeerd voor het aantal adenomas (��n versus tenminste twee), familiale achtergrond van dikke darm kanker in een eerstegraads familielid, en MTHFR C677T genotype.

Hoofdstuk 6 presenteert een prospectieve cohort studie die was uitgevoerd om de associatie tussen het ontstaan van colorectale adenomen en de inname van folaat, vitamine B2, vitamine B6, vitamine B12 en methionine te onderzoeken bij mensen met een heel hoog risico op dikke darmkanker, nl 470  individuen met Lynch Syndroom.  Inname van de nutrienten werd bepaald door middel van een voedsel frequentie vragenlijst. Cox regressie modellen met robuuste sandwich covariatie schatting, gecorrigeerd voor leeftijd, geslacht, lichamelijke activiteit, aantal colonoscopieen tijdens person-time, gebruik van NSAIDs, en  �andere� vitamines werden gebruikt om de hazard ratio (HR) te en de bijbehorende 95\% btbhi te berekenen.  De analyses werden ook gestratificeerd voor MTHFR C677T genotype. Vergeleken met  het laagste tertiel  van inname, zijn  de gecorrigeerde HRs (95\%CI)s in het hoogste tertiel: 1.06 (0.59-1.91) voor folaat, 0.77 (0.39-1.51) voor vitamine B2, 0.98 (0.59-1.62) voor vitamine B6, 1.24 (0.77-2.00) voor vitamine B12, en 1.36 (0.83-2.20) voor methionine.  Samenvattend lijkt de inname van B vitaminen en methionine  uit de voeding  niet  geassocieerd met adenoom ontwikkeling in  mensen met het Lynch Syndroom.

Het laatste hoofdstuk (Hoofdstuk 7) bevat de algemene discussie, waarin  de resultaten, zoals gepresenteerd in de voorgaande hoofdstukken, als geheel worden besproken en bediscussieerd. De algehele bevindingen van dit proefschrift duiden erop dat de status van folaat en andere B vitaminen mogelijk niet geassocieerd zijn met globale DNA methylering, noch met LINE-1 DNA methylering in witte bloedcellen, behalve mogelijk in personen met een voorgeschiedenis van dikke darm adenomen. Omdat zowel onze resultaten als die van anderen niet genoeg bewijs leveren dat een lage of hoge folaat status leukocyt DNA methylering kan be�nvloeden in individuen met een verschillend onderliggend risico op dikke darmkanker,  is er geen reden om de huidige aanbevelingen om het risico op kanker te verlagen aan te passen. Omdat DNA methylering patronen niet statisch zijn, en veranderen tijdens het leven o.a. als reactie op verschillende omgeving factoren, . informatief zijn om personen een levenlang te volgen en een epigenomisch-brede associatie studie uit te voeren om de tien jaar vanaf hun vroege levensjaren. Tevens hebben wiskundige modellen en in vivo studies aangetoond dat DNA methylering concurreert met DNA synthese . Een  lange termijn gerandomiseerde gecontrolleerde  studie met foliumzuur suppletie  in gezonde jonge mensen zou onze kennis vergroten over de impact van folium zuur op DNA synthese en methylering vanaf een jonge leeftijd.