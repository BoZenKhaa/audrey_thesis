\chapter[No effect of folic acid supplementation on global DNA methylation in men and women with moderately elevated homocysteine]{No effect of folic acid supplementation on global DNA methylation in men and women with\\moderately elevated homocysteine}
\chaptermark{Folic acid supplementation and global DNA methylation}
\label{chap4_FACIT} 
\AddEverypageHook{%
	\ifthenelse{\equal{\Chaptername}{Folic acid supplementation and global DNA methylation}}{
		\ifthenelse{\isodd{\thepage}}%
		{\TileWallPaper{17.4cm}{24.4cm}{thumbCh4.pdf}}%
		{\ClearWallPaper}
	} {\ClearWallPaper}
} 

\quad\\

\quad\\
\noindent
Audrey Y. Jung\\ 
Yvo Smulders\\ 
Petra Verhoef\\ 
Frans J. Kok\\ 
Henk Blom\\ 
Robert M. Kok\\ 
Ellen Kampman\\ 
Jane Durga\\ 

\begin{table}[b]
\emph{PLoS ONE 6(9): e24976. doi:10.1371/journal.pone.0024976}
\end{table}

\newpage 

\section*{Abstract}

\noindent A global loss of cytosine methylation in DNA has been implicated in a wide range of diseases. There is growing evidence that modifications in DNA methylation can be brought about by altering the intake of methyl donors such as folate. We examined whether long-term daily supplementation with 0.8 mg of folic acid would increase global DNA methylation compared with placebo in individuals with elevated plasma homocysteine. We also investigated if these effects were modified by \emph{MTHFR} C677T genotype. Two hundred sixteen participants out of 818 subjects who had participated in a randomised double-blind placebo-controlled trial were selected, pre-stratified on \emph{MTHFR} C677T genotype and matched on age and smoking status. They were allocated to receive either folic acid (0.8 mg/d; n = 105) or placebo treatment (n = 111) for three years. Peripheral blood leukocyte DNA methylation and serum and erythrocyte folate were assessed. Global DNA methylation was measured using liquid chromatography-tandem mass spectrometry and expressed as a percentage of 5-methylcytosines \emph{versus} the total number of cytosine. There was no difference in global DNA methylation between those randomised to folic acid and those in the placebo group (difference = 0.008, 95\%CI = -0.05,0.07, P = 0.79). There was also no difference between treatment groups when we stratified for \emph{MTHFR} C677T genotype (CC, n = 76; CT, n = 70; TT, n = 70), baseline erythrocyte folate status or baseline DNA methylation levels. In moderately hyperhomocysteinemic men and women, long-term folic acid supplementation does not increase global DNA methylation in peripheral blood leukocytes.

\newpage

\section[]{Introduction} % level 1 
\noindent Disturbances in DNA methylation, one of several epigenetic mechanisms, have been implicated in many diseases ranging from cancers to cardiovascular disease. Two main types of changes in DNA methylation patterns have been observed -- global hypomethylation \cite{c41}, and regional hypermethylation, particularly in the promoter regions of certain genes such as tumour suppressor genes or imprinted genes \cite{c42,c43,c44}. Global DNA methylation provides genomic stability and structure \cite{c45} while promoter hypermethylation inhibits the transcription of associated genes, resulting in gene silencing \cite{c46}.

\noindent Folate-mediated one-carbon metabolism is key for DNA methylation as well as for nucleotide synthesis, and DNA repair and stability. Folate, a water soluble B-vitamin, functions as a donor and acceptor of one-carbon units in its various forms in cellular metabolism. As such, manipulation of intake of folate or other factors involved in one-carbon metabolism can largely influence DNA methylation, and methyl-deficient and repletion diets have been shown to alter global DNA methylation patterns in animal studies \cite{c47} and also in human studies \cite{c48,c49} although the evidence is not unequivocal \cite{c410,c411}.

\noindent Many factors affect the availability of methyl groups in folate-mediated one-carbon metabo-lism. One of the most extensively studied genetic polymorphisms which can modify folate metabolism is the 677 C$\rightarrow$T polymorphism of the gene that codes for the enzyme methylenetetrahydrofolate reductase (MTHFR). Methylenetetrahydrofolate reductase converts 5,10-methylenetetrahydrofolate to 5-methyltetrahydrofolate, which is required for the conversion of homocysteine to methionine. Rates of folate metabolism have been shown to differ between those who have the TT genotype compared to those with the CC genotype; production of methyl group donors is higher in those with the CC genotype compared to those with the TT genotype \cite{c412}.

\noindent Indeed, on a functional level, since homocysteine has to be remethylated to form methionine, individuals with the \emph{MTHFR} 677TT polymorphism have higher concentrations of plasma homocysteine when folate status is low compared to those with the other two genotypes \cite{c413,c414}. \emph{MTHFR} 677TT individuals were also found to have lowered DNA methylation in their leukocytes when folate status is low compared to those who have the \emph{MTHFR} 677CC genotype \cite{c415}.

\noindent Given the importance of DNA methylation in epigenetic regulation, and the documented effects of experimental folate deprivation and repletion, it is relevant to know the ways in which folic acid supplementation impacts DNA methylation. We investigated whether daily supplementation of 800 mg folic acid for 3 years would increase global DNA methylation in peripheral blood leukocytes compared with placebo in individuals with elevated homocysteine concentrations. In addition, we addressed effects separately in strata of the \emph{MTHFR} C677T genotypes.

\section[]{Materials and methods} % level 1 

\paragraph*{Ethics statement} 
The study protocol was approved by the Medical Ethics Committee of Wageningen University. All participants gave written informed consent.

\subsection{Study participants} % level 2 
\noindent Participants were a subset of Dutch men and post-menopausal women aged 50-70 years from a central-eastern region of the Netherlands and who had participated in the Folic Acid and Carotid Intima-media Thickness (FACIT) trial \cite{c416}. The FACIT trial was designed to investigate the effect of folic acid supplementation on atherosclerotic progression. Additional outcomes of the trial were age-related decline in cognitive function and hearing \cite{c417,c418,c419}. Here we present data for the effect of folic acid on global DNA methylation for a subset of participants.

\noindent Detailed participant recruitment for this randomised double blind placebo-controlled trial has been described previously \cite{c417}. Briefly, participants were recruited using municipal and blood-bank registries and randomised between November 1999 and April 2001. Inclusion criteria were plasma total homocysteine concentrations between 13 mmol/L and 26 mmol/L, vitamin B12 less than 200 pmol/L, no intestinal disease, and no current use of B-vitamin supplements or other medications that could influence folate metabolism or atherosclerotic progression. Finally, more than 80\% self-reported compliance during a 6-week placebo run-in period was required.

\noindent After the initial measurement sessions, participants were randomly allocated to treatment of 800 mg folic acid per day or placebo with permuted blocks of sizes four and six, which varied randomly. Specialised staff who were not involved in the study allocated and labelled the capsule boxes with participants' unique sequence number. Those participants in the same household received the same treatment. The folic acid and placebo capsules were produced by Swiss-Caps Benelux (Heerhugowaard, Netherlands) and were identical in appearance. Capsules were individually packaged in foil strips containing 28 capsules per strip with the days of the week printed on the back. Every year, participants received a 13-month supply of capsules. Compliance was determined by capsule-return counts and a diary that registered missed capsules. Diaries and capsules were returned by participants every 12 weeks.

\noindent For the present study, changes in global DNA methylation were assessed from stored frozen aliquots from a sub-sample of the FACIT trial participants. Using only the data from the participants who had completed the study (n=818), participants in the folic acid arm of the trial were pre-stratified by \emph{MTHFR} C677T genotype and matched to participants in the placebo arm on age and current smoking status. Subjects were matched on these variables because they are thought to influence global DNA methylation \cite{c420,c421,c422,c423,c424,c425}. Additionally, sampling was done such that each group of \emph{MTHFR} C677T genotype was of the same size. This was done to increase our power of hypotheses testing related to the genotypes and their effects on global DNA methylation. Initially, 120 folic acid participants were matched to 120 in the placebo group, but 24 samples were not measured due to human error in sample retrieval.

\subsection{Laboratory measurements} % level 2 
\noindent Fasting venous blood was processed and samples were stored at -80\textsuperscript{o}C. We measured serum folate, erythrocyte folate, serum vitamin B12, plasma total homocysteine, plasma vitamin B6 (pyridoxal 5'-phosphate (PLP)), in addition to serum creatinine and lipids as described elsewhere \cite{c416}. In brief, serum and erythrocyte folate and serum vitamin B12 were measured using a chemiluminescent immunoassay (Immulite\textsuperscript{\textregistered} 2000, diagnostic Products Corporation). Erythrocyte folate measurements were done in duplicate. Plasma homocysteine was determined with high performance liquid chromatography (HPLC) and fluorimetric detection \cite{c426}. Vitamin B6 was measured by HPLC. DNA was extracted from blood leukocytes and the C677T polymorphism in the methylenetetrahydrofolate reductase (\emph{MTHFR}) gene was determined by PCR and restriction digest with \emph{HinF1} and attained at the beginning of the study. Intra- and inter-assay variation coefficients for laboratory analyses were <15\%. A validated food frequency questionnaire was used at baseline and after the intervention to estimate dietary folate intake during the past 3 months \cite{c427}. 
 
\noindent Quantification of global DNA methylation was measured by liquid chromatography-tandem mass spectrometry \cite{c428}. Global DNA methylation is expressed as a percentage of 5-methyl-cytosines \emph{versus} the total number of cytosines present in the genome. 
 
\subsection{Statistical methods} % level 2 
\noindent Non-parametric tests were used to compare differences within groups and between groups before and after the intervention. Analysis of covariance (ANCOVA) was used to estimate differences between groups at baseline and after the intervention and also to compare the effect of supplementation on global DNA methylation adjusted for baseline DNA methylation and covariates. Medians, inter-quartile ranges, and 95\% confidence intervals are presented. The variables on which subjects were matched (i.e. age and current smoking status) were used as covariates in our model. We performed subgroup analyses for those with deficient to low-normal erythrocyte folate status at baseline and those with low and high baseline DNA methylation. We also tested if the effect of folic acid on global DNA methylation was modified by \emph{MTHFR} C677T genotype, age, current smoking status, alcohol intake, body mass index (BMI), and sex. Previous literature suggests that BMI may be associated with global DNA methylation \cite{c429,c430} and sex has been shown to influence DNA methylation \cite{c431,c432,c433}. There was no missing data on any of our outcome variables. Differences were considered significant at P<0.05. All analyses were computed using SAS version 9.2 (SAS Institute Inc., Cary, NC). 
 
\section{Results} % level 1 
\noindent Two hundred sixteen participants (folic acid n= 105 and placebo n= 111) were included in our analyses. Table \ref{table4_1} depicts the baseline characteristics of the population. There were slightly more men in the placebo group than in the folic acid group (66.7\% \emph{vs}. 62.9\%), and more current smokers in the folic acid group (16.2\% \emph{vs}. 13.5\%). Concordant with the stratification procedure, there were roughly equal numbers of people in each \emph{MTHFR} C677T genotype for both treatment groups. Folate intake was similar between the treatment groups, but alcohol consumption was higher in the folic acid group. Overall, participants had a low median folate intake (188 mg/day), but there were only 20 serum folate deficient ($\leq$7 nmol/L) individuals and five individuals, who were deficient based on erythrocyte folate concentrations ($\leq$305 nmol/L).
 
\noindent Table \ref{table4_2} shows the effects of folic acid supplementation on several metabolites throughout the study. As expected, there were large increases in serum folate (535\% increase), and erythrocyte folate (207\% increase), and a fall in plasma total homocysteine (21.6\% decrease) after the intervention in the folic acid group compared to placebo (P<0.0001 for all). There were no changes in dietary folate intake during the course of the intervention.

\noindent Median global DNA methylation levels throughout the study can be found in Table \ref{table4_3}. There was no significant change in DNA methylation within treatment groups after three years (folic acid difference = 0.02, 95\%CI = -0.05,0.02, P = 0.39; placebo difference = 0.02, 95\%CI = -0.06,0.01, P = 0.62). There was also no difference in the effect of supplementation after three years between the two groups (difference = 0.008, 95\%CI = -0.05,0.07, P = 0.79). There was no significant treatment effect of folic acid on global DNA methylation for those with less than normal baseline red blood cell (RBC) folate ($\leq$500 nmol/L) (Table \ref{table4_3}) when compared with the placebo group. Additionally, there was no trend towards higher methylation following supplementation in those with below median DNA methylation at baseline (median difference = 0.003, 95\%CI = -0.06,0.07, P = 0.92) compared with placebo. When we stratified according to \emph{MTHFR} C677T genotype, there was no effect of folic acid on global DNA methylation in any of the genotype subgroups (from Table \ref{table4_3}, P = 0.65 for CC, P = 0.32 for CT, P = 0.45 for TT), nor did we observe an effect when we stratified according to alcohol intake (data not shown). Furthermore, for those who were both \emph{MTHFR} 677TT and having less than normal baseline RBC folate status, there was also no significant effect of folic acid on global DNA methylation compared with the placebo group (data not shown). In other subgroup analyses, there was no difference overall in global DNA methylation between males and females. Finally, no differences in global DNA methylation due to folic acid supplementation were observed when we stratified by age, current smoking status, body mass index (BMI), or sex (data not shown).


% TABLE 4.1 HERE
\begin{table}[hp!]
\caption{Baseline characteristics of the population.}
\label{table4_1}
\begin{adjustbox}{width=\textwidth}
\renewcommand{\arraystretch}{1.1}
\begin{tabular}{L{6.3cm}C{3cm}C{3cm}}
\hline
~ & ~ & ~\\
 ~ & \bfseries Folic acid (n=105) & \bfseries Placebo (n=111)\\
 ~ & ~ & ~\\
\hline
\multicolumn{3}{l}{\bfseries Demographics}\\
Age (years), mean (SD) & 60.9 (5.4) & 60.9 (5.5)\\
Sex, male (\%) & 66 (62.9\%) & 74 (66.7\%)\\
Education\\
\quad High & 32 (30.5\%) & 25 (22.5\%)\\
\quad Middle & 41 (39.1\%) & 43 (38.7\%)\\
\quad Low & 32 (30.5\%) & 43 (38.7\%)\\
~ & ~ & ~\\
\multicolumn{3}{l}{\bfseries Dietary factors}\\
Alcohol intake, g/day, median (IQR) & 14.5 (5.5-24.4) & 10.3 (3.5-22.0)\\
Folate intake, $\mu$g/day, median (IQR) & 185.8 (159.7-220.7) & 194.3 (156.0-236.4)\\
Serum vitamin B12, pmol/L, median (IQR) & 295 (259-359) & 283 (247-374)\\
Serum vitamin B6, nmol/L, median (IQR) & 31.8 (25.5-41.4) & 31.4 (23.7-44.5)\\
~ & ~ & ~\\
\multicolumn{3}{l}{\bfseries Non-dietary factors}\\
BMI, mean (SD) & 26.9 (3.7) & 26.4 (4.2)\\
Current smokers, n (\%) & 17 (16.2\%) & 15 (13.5\%)\\
\emph{MTHFR} C677T genotype, n (\%) & ~  & ~ \\
\quad CC & 36 (34.3\%) & 40 (36.0)\\
\quad CT & 36 (34.3\%) & 34 (30.6\%)\\
\quad TT & 33 (31.4\%) & 37 (33.3\%)\\
\hline
\end{tabular}
\end{adjustbox}
\end{table}

% TABLE 4.2 HERE
\begin{table}[hp!]
\caption{Biochemical measurements throughout the study.}
\small
\label{table4_2}
\begin{adjustbox}{width=\textwidth}
\renewcommand{\arraystretch}{1.3}
\begin{tabular}{L{3cm}C{3.3cm}C{3cm}C{1cm}}
\hline
~ & ~ & ~ & ~\\
\bfseries Metabolite & \bfseries Folic acid (n=105) & \bfseries Placebo (n=111) & \bfseries p-value\\
~ & ~ & ~ & ~\\
\hline
\multicolumn{4}{l}{\bfseries Serum folate (nmol/L)}\\
Baseline & 12.0 (9.9-14.0)\textsuperscript{a} & 10.8 (8.5-13.9) & 0.01\\
Year 3 & 76.3 (53.4-106.0) & 12.5 (10.3-16.2) & \multicolumn{1}{c}{\textless0.0001}\\
\multicolumn{4}{l}{\bfseries Erythrocyte folate (nmol/L)}\\
Baseline & 682.7 (528.8-895.8) & 677.5 (548.2-807.6) & 0.37\\
Year 3 & 2100.7 (1767.2-2611.2) & 679.0 (564.6-864.3) & \multicolumn{1}{c}{\textless0.0001}\\
\multicolumn{4}{l}{\bfseries Dietary folate intake ($\mu$g/day)}\\
Baseline & 185.8 (159.7-220.7) & 194.3 (156.0-236.4) & 0.48\\
Year 3 & 173.0 (152.0-197.0) & 173.0 (146.0-209.0) & 0.48\\
\multicolumn{4}{l}{\bfseries Plasma total homocysteine ($\mu$mol/L)}\\
Baseline & 12.5 (11.3-14.5) & 13.0 (11.4-15.5) & 0.24\\
Year 3 & 9.8 (8.8-11.0) & 13.4 (11.6-15.1) & \multicolumn{1}{c}{\textless0.0001}\\
\hline
\end{tabular}
\end{adjustbox}
\caption*{\footnotesize{\textsuperscript{a}values given are median (inter-quartile range).}}
\end{table}

\FloatBarrier

% TABLE 4.3 HERE
\begin{table}
\small
\renewcommand*{\arraystretch}{1.1}
\caption{Folic acid supplementation on global DNA methylation for the whole study population, for those with less than normal RBC folate status at baseline, low and high baseline methylation, and stratified by \emph{MTHFR} C677T genotype.}
\label{table4_3}
%\begin{adjustbox}{width=\textwidth}
\begin{tabular}{L{2.6cm}C{2.5cm}C{2.4cm}C{2.7cm}C{1cm}}

%\endfirsthead
%\caption{\emph{(continued)}}\\
%\endhead
\hline
\multicolumn{5}{l}{\bfseries Whole study population}\\
 ~ & \textbf{Folic acid }(n=105) & \textbf{Placebo }(n=111) & \bfseries Difference (95\%CI) & \bfseries p-value\\
 Baseline & 4.63 (4.53,4.74)\textsuperscript{a,b} & 4.60 (4.51,4.72) & {}-0.01 (-0.07,0.05) & 0.71\\
 Year 3 & 4.62 (4.49,4.73) & 4.56 (4.56,4.63) & 0.004 (-0.06,0.07) & 0.90\\
 Difference (95\%CI) & {}-0.02 (-0.05,0.02) & {}-0.02 (-0.06,0.01) & 0.008 (-0.05,0.07) & 0.79\\
 p-value & 0.39 & 0.62 & ~ & ~ \\
 \\
\multicolumn{5}{l}{\textbf{Deficient to low-normal RBC folate status} (${\leq}$500nmol/L)}\\
~ & \textbf{Folic acid }(n=22) & \textbf{Placebo }(n=22) & \bfseries Difference (95\%CI) & \bfseries p-value\\
 Baseline & 4.62 (4.53,4.64) & 4.59 (4.51,4.70) & 0.039 (-0.13,0.21) & 0.65\\
 Year 3 & 4.59 (4.52,4.76) & 4.69 (4.52,4.73) & {}-0.046 (-0.19,0.10) & 0.53\\
 Difference (95\%CI) & 0.01 (-0.09,0.10) & 0.04 (-0.09,0.13) & {}-0.058 (-0.20,0.08) & 0.41\\
 p-value & 0.81 & 0.80 & ~ & ~ \\
 \\
\multicolumn{5}{l}{\textbf{Low/high baseline methylation} (median split)}\\
~ & \bfseries Folic acid & \bfseries Placebo & \bfseries Difference (95\%CI) & \bfseries p-value\\
\bfseries Low {\textless}4.615 & n=48 & n=60 & ~ & ~ \\
 Baseline & 4.51 (4.40,4.56) & 4.51 (4.42,4.57) & {}-0.002 (-0.05,0.04) & 0.92\\
 Year 3 & 4.53 (4.40,4.66) & 4.54 (4.45,4.66) & 0.003 (-0.06,0.07) & 0.93\\
 Difference (95\%CI) & 0.04 (-0.01,0.08) & 0.02 (-0.02,0.08) & 0.003 (-0.06,0.07) & 0.92\\
 p-value & 0.045 & 0.045 & ~ & ~ \\
\bfseries High ${\geq}$ 4.615 & n=57 & n=51 & ~ & ~ \\
 Baseline & 4.72 (4.66,4.82) & 4.73 (4.66,4.89) & 0.02 (-0.03,0.10) & 0.29\\
 Year 3 & 4.68 (4.59,4.78) & 4.66 (4.54,4.78) & 0.02 (-0.08,0.14) & 0.63\\
 Difference (95\%CI) & {}-0.06 (-0.09,-0.03) & {}-0.08 (-0.14,-0.02) & 0.008 (-0.10,0.11) & 0.87\\
 p-value & 0.002 & 0.006 & ~ & ~ \\
\hline
\end{tabular}
%\end{adjustbox}
\caption*{\footnotesize{\textsuperscript{a}values are given as median (inter-quartile range).\\\textsuperscript{b}global DNA methylation is expressed as a percentage of 5-methylcytosines \emph{versus} the total number of cytosines present in the genome.}}
\end{table}

% TABLE 4.3 CONTINUED
\begin{table}
\small
\renewcommand*{\arraystretch}{1.1}
\caption*{\textbf{Table 4.3} Folic acid supplementation on global DNA methylation for the whole study population, for those with less than normal RBC folate status at baseline, low and high baseline methylation, and stratified by \emph{MTHFR} C677T genotype. \emph{(continued)}}
%\begin{adjustbox}{width=\textwidth}
\begin{tabular}{L{2.6cm}C{2.5cm}C{2.4cm}C{2.7cm}C{1cm}}
\multicolumn{5}{l}{\textbf{\textit{MTHFR}}\textbf{ C677T genotype}}\\
~ & \bfseries Folic acid & \bfseries Placebo & \bfseries Difference (95\%CI) & \bfseries p-value\\
\bfseries CC & n=36 & n=40 & ~ & ~ \\
 Baseline & 4.63 (4.60,4.64) & 4.60 (4.57,4.65) & {}-0.035 (-0.12,0.05) & 0.41\\
 Year 3 & 4.58 (4.49,4.69) & 4.58 (4.51,4.63) & {}-0.035 (-0.12,0.05) & 0.42\\
 Difference (95\%CI) & {}-0.07 (-0.13,-0.01) & {}-0.06 (-0.10,-0.01) & {}-0.017 (-0.09,0.06) & 0.65\\
 p-value & 0.02 & 0.03 & ~ & ~ \\
\bfseries CT & n=36 & n=34 & ~ & ~ \\
 Baseline & 4.63 (4.56,4.69) & 4.54 (4.47,4.64) & {}-0.06 (-0.16,0.04) & 0.27\\
 Year 3 & 4.63 (4.57,4.69) & 4.66 (4.55,4.70) & 0.06 (-0.07,0.19) & 0.37\\
 Difference (95\%CI) & {}-0.01 (-0.06,0.04) & 0.05 (0.00,0.13) & 0.07 (-0.06,0.20) & 0.32\\
 p-value & 0.78 & 0.10 & ~ & ~ \\
\bfseries TT & n=33 & n=37 & ~ & ~ \\
 Baseline & 4.64 (4.54,4.71) & 4.63 (4.58,4.70) & 0.06 (-0.05,0.18) & 0.29\\
 Year 3 & 4.62 (4.56,4.72) & 4.59 (4.54,4.64) & {}-0.01 (-0.14,0.12) & 0.86\\
 Difference (95\%CI) & 0.02 (-0.03,0.10) & {}-0.04 (-0.09,0.04) & {}-0.04 (-0.16,0.07) & 0.45\\
 p-value & 0.32 & 0.61 & ~ & ~ \\
%\endfirsthead
%\caption{\emph{(continued)}}\\
%\endhead
\hline
\end{tabular}
%\end{adjustbox}
\caption*{\footnotesize{\textsuperscript{a}values are given as median (inter-quartile range).\\\textsuperscript{b}global DNA methylation is expressed as a percentage of 5-methylcytosines \emph{versus} the total number of cytosines present in the genome.}}
\end{table}


\FloatBarrier

% TABLE 4.4 HERE

\begin{sidewaystable}[hp!]
\scriptsize
\caption{Overview of all randomised controlled trials of folic acid with global DNA methylation as an endpoint.}
\label{table4_4}
%\begin{adjustbox}{width=18cm}
%\renewcommand{\arraystretch}{1.3}
\begin{tabular}{L{1.3cm}C{1.5cm}C{1.5cm}C{1.2cm}C{1.4cm}C{1.5cm}C{2cm}C{2cm}C{1.8cm}}
\hline
\multicolumn{9}{l}{\bfseries CANCER-FREE POPULATION}\\
\bfseries Study & \bfseries Number of participants & \bfseries Dose & \bfseries Duration & \bfseries Endpoint & \bfseries DNA methylation assessment method & \bfseries Baseline concentrations of folate\textsuperscript{a} & \bfseries Baseline levels of DNA methylation & \bfseries Treatment effect/Outcome\\
\hline
\parbox[t][2.2cm]{1.3cm}{\raggedright Fenech \textit{et al}. 1998 \cite{c434}} & 
\parbox[t][2.2cm]{1.5cm}{\centering 63 volunteers} &
\parbox[t][2.2cm]{1.5cm}{\centering 2 mg folic acid and 20 $\mu$g vitamin B12} &
\parbox[t][2.2cm]{1.2cm}{\centering 12 weeks} &
\parbox[t][2.2cm]{1.4cm}{\centering Global DNA methylation in leukocytes} &
\parbox[t][2.2cm]{1.5cm}{\centering [\textsuperscript{3}H] methyl incorporation} &
\parbox[t][2.2cm]{2cm}{\centering RBC folate (nmol/L) Men 440.0 (20.4); women 363.8 (17.2)} &
\parbox[t][2.2cm]{2.0cm}{\centering Intervention 195200 dpm; placebo 169600 dpm} &
\parbox[t][2.2cm]{1.8cm}{\centering No effect}\\

\parbox[t][2.5cm]{1.3cm}{\raggedright Basten \textit{et al}. 2006 \cite{c435}} &
\parbox[t][2.5cm]{1.5cm}{\centering 61 healthy volunteers} &
\parbox[t][2.5cm]{1.5cm}{\centering 1.2 mg folic acid} &
\parbox[t][2.5cm]{1.2cm}{\centering 12 weeks} &
\parbox[t][2.5cm]{1.4cm}{\centering Global DNA methylation in lymphocytes} &
\parbox[t][2.5cm]{1.5cm}{\centering [\textsuperscript{3}H] methyl incorporation} &
\parbox[t][2.5cm]{2cm}{\centering RBC folate (nmol/L) intervention 552 (469-655), placebo 668 (508-796) \cite{c448}} &
\parbox[t][2.5cm]{2.0cm}{\centering Intervention 17508 dpm; placebo 16099 dpm} &
\parbox[t][2.5cm]{1.8cm}{\centering No effect}\\

\parbox[t][4.1cm]{1.3cm}{\raggedright Present study} &
\parbox[t][4.1cm]{1.5cm}{\centering 216 healthy volunteers (elevated Hcy)} &
\parbox[t][4.1cm]{1.5cm}{\centering 800 $\mu$g folic acid} &
\parbox[t][4.1cm]{1.2cm}{\centering 3 years} &
\parbox[t][4.1cm]{1.4cm}{\centering Global DNA methylation in leukocytes} &
\parbox[t][4.1cm]{1.5cm}{\centering Liquid chromatography-tandem mass spectrometry (LC-ES MS/MS)} &
\parbox[t][4.1cm]{2cm}{\centering RBC folate (nmol/L) intervention 682.7 (528.8-895.8), placebo 677.5 (548.2-807.6); serum folate (nmol/L) intervention 12.0 (9.9-14.0), placebo 10.8 (8.5-13.9)} &
\parbox[t][4.1cm]{2.0cm}{\centering Intervention 4.63\% (4.60-4.66), placebo 4.60\% (4.58-4.64)} &
\parbox[t][4.1cm]{1.8cm}{\centering No effect}\\
\hline
\end{tabular}
%\end{adjustbox}
\caption*{\footnotesize{\textsuperscript{a}conversion factor of 2.266 for folate from ng/mL to nmol/L.}}
\end{sidewaystable}
%TABLE 4.4 CONTINUED PART 1
\begin{sidewaystable}[hp!]
\scriptsize
\caption*{\textbf{Table 4.4} Overview of all randomised controlled trials of folic acid with global DNA methylation as an endpoint. \emph{(continued)}}
%\begin{adjustbox}{width=17cm}
\begin{tabular}{L{1.3cm}C{1.5cm}C{1.5cm}C{1.2cm}C{1.4cm}C{1.5cm}C{2cm}C{2cm}C{1.8cm}}
\hline
\multicolumn{9}{l}{\bfseries CANCER POPULATION}\\
\bfseries Study & \bfseries Number of participants & \bfseries Dose & \bfseries Duration & \bfseries Endpoint & \bfseries DNA methylation assessment method & \bfseries Baseline concentrations of folate\textsuperscript{a} & \bfseries Baseline levels of DNA methylation & \bfseries Treatment effect/Outcome\\
\hline
\parbox[t][2.85cm]{1.3cm}{\raggedright Cravo \textit{et al}. 1994 \cite{c439}} & 
\parbox[t][2.85cm]{1.5cm}{\centering 22 patients with colorectal adenomas or carcinomas and 8 healthy controls} &
\parbox[t][2.85cm]{1.5cm}{\centering 10 mg folic acid} &
\parbox[t][2.85cm]{1.2cm}{\centering 6 months} &
\parbox[t][2.85cm]{1.4cm}{\centering Global DNA methylation in colorectal tissues} &
\parbox[t][2.85cm]{1.5cm}{\centering [\textsuperscript{3}H] methyl incorporation} &
\parbox[t][2.85cm]{2cm}{\centering Serum folate (nmol/L) in intervention group 21.5 (3.8); placebo group 17.2 (3.6)} &
\parbox[t][2.85cm]{2cm}{\centering Normal-appearing rectal mucosa from controls 109000 dpm} &
\parbox[t][2.85cm]{1.8cm}{\centering Increased global DNA methylation}\\

\parbox[t][2cm]{1.3cm}{\raggedright Cravo \textit{et al}. 1998 \cite{c437}} &
\parbox[t][2cm]{1.5cm}{\centering 20 colorectal adenoma patients} &
\parbox[t][2cm]{1.5cm}{\centering 5 mg folic acid} &
\parbox[t][2cm]{1.2cm}{\centering 6 months} &
\parbox[t][2cm]{1.4cm}{\centering Global methylation in colorectal tissues} &
\parbox[t][2cm]{1.5cm}{\centering [\textsuperscript{3}H] methyl incorporation} &
\parbox[t][2cm]{2cm}{\centering Serum folate (nmol/L) 43.5 (8.3)} &
\parbox[t][2cm]{2.0cm}{\centering 237039 dpm} &
\parbox[t][2cm]{1.8cm}{\centering No effect}\\

\parbox[t][2.75cm]{1.3cm}{\raggedright Kim \textit{et al}. 2001 \cite{c440}} &
\parbox[t][2.75cm]{1.5cm}{\centering 20 colorectal adenoma patients} &
\parbox[t][2.75cm]{1.5cm}{\centering 5 mg folic acid} &
\parbox[t][2.75cm]{1.2cm}{\centering 1 year} &
\parbox[t][2.75cm]{1.4cm}{\centering Global DNA methylation in colorectal tissues} &
\parbox[t][2.75cm]{1.5cm}{\centering [\textsuperscript{3}H] methyl incorporation} &
\parbox[t][2.75cm]{2cm}{\centering Serum folate (nmol/L) intervention 67.9, placebo 33.9; RBC folate (nmol/L) intervention 793.1, placebo 600.4} &
\parbox[t][2.75cm]{2.0cm}{\centering Intervention 220000 dpm; placebo 220000 dpm} &
\parbox[t][2.75cm]{1.8cm}{\centering Increased global DNA methylation}\\
\hline
\end{tabular}
%\end{adjustbox}
\caption*{\footnotesize{\textsuperscript{a}conversion factor of 2.266 for folate from ng/mL to nmol/L.}}
\end{sidewaystable}

% TABLE 4.4 CONTINUED PART 2
\begin{sidewaystable}[hp!]
\scriptsize
\caption*{\textbf{Table 4.4} Overview of all randomised controlled trials of folic acid with global DNA methylation as an endpoint. \emph{(continued)}}
%\begin{adjustbox}{width=18cm}
%\renewcommand{\arraystretch}{1.3}
\begin{tabular}{L{1.3cm}C{1.5cm}C{1.5cm}C{1.2cm}C{1.4cm}C{1.5cm}C{2cm}C{2cm}C{1.8cm}}
\hline
\multicolumn{9}{l}{\bfseries CANCER POPULATION}\\
\bfseries Study & \bfseries Number of participants & \bfseries Dose & \bfseries Duration & \bfseries Endpoint & \bfseries DNA methylation assessment method & \bfseries Baseline concentrations of folate\textsuperscript{a} & \bfseries Baseline levels of DNA methylation & \bfseries Treatment effect/Outcome\\
\hline

\parbox[t][4.2cm]{1.3cm}{\raggedright Pufulete \textit{et al}. 2005 \cite{c441}} &
\parbox[t][4.2cm]{1.5cm}{\centering 31 colorectal adenoma patients} &
\parbox[t][4.2cm]{1.5cm}{\centering 400 $\mu$g folic acid} &
\parbox[t][4.2cm]{1.2cm}{\centering 10 weeks} &
\parbox[t][4.2cm]{1.4cm}{\centering Global DNA methylation in leukocytes and colorectal tissues} &
\parbox[t][4.2cm]{1.5cm}{\centering [\textsuperscript{3}H]methyl incorporation} &
\parbox[t][4.2cm]{2cm}{\centering Serum folate (nmol/L) intervention 16.7 (12.9-20.8), placebo 18.5 (14.9-22.2); RBC folate (nmol/L) intervention 639.0 (514.3-876.9), placebo 716.0 (591.4-840.6)} &
\parbox[t][4.2cm]{2cm}{\centering Leukocytes 748 (672-825) Bq/$\mu$g DNA; colon 602 (515-689) Bq/$\mu$g DNA} &
\parbox[t][4.2cm]{1.8cm}{\centering Increased global DNA methylation}\\

\parbox[t][2cm]{1.3cm}{\raggedright Figueiredo \textit{et al}. 2009 \cite{c438}} &
\parbox[t][2cm]{1.5cm}{\centering 388 colorectal adenoma patients} &
\parbox[t][2cm]{1.5cm}{\centering 1 mg folic acid} &
\parbox[t][2cm]{1.2cm}{\centering 3 years} &
\parbox[t][2cm]{1.4cm}{\centering Global DNA methylation in colorectal tissues} &
\parbox[t][2cm]{1.5cm}{\centering Bisulfite pyrosequencing LINE-1 analysis} &
\parbox[t][2cm]{2cm}{\centering RBC folate (nmol/L) 898.2 (16.3); plasma folate (nmol/L) 20.9 (0.8)} &
\parbox[t][2cm]{2cm}{\centering None taken} &
\parbox[t][2cm]{1.8cm}{\centering No effect}\\
\hline
\end{tabular}
%\end{adjustbox}
\caption*{\footnotesize{\textsuperscript{a}conversion factor of 2.266 for folate from ng/mL to nmol/L.}}
\end{sidewaystable}
 
\section{Discussion} % level 1 
\noindent In this double-blind randomised controlled trial of daily folic acid supplementation in a Dutch population with moderately elevated homocysteine, we did not find any evidence of changes in global DNA methylation after three years of supplementation. Those whose genotype has previously been associated with lower leukocyte DNA methylation were no exception. 
 
\noindent To our knowledge, seven randomised controlled trials have previously examined the effect of folic acid on global DNA methylation (Table \ref{table4_4}). Two of these trials were also performed in cancer-free populations, but were short-term (12 weeks duration), used supraphysiological doses of folic acid (2 mg folic acid and 20 mg vitamin B12 and 1.2 mg folic acid, respectively), and used small sample sizes (included 63 and 61 volunteers, respectively \cite{c434,c435}. No effect of folic acid on global DNA methylation in leukocytes was observed in both these studies, which used [\textsuperscript{3}H] methylation incorporation to assess DNA methylation. Additionally, five randomised controlled trials in cancer-related studies were performed, as a global loss of methylated cytosines is frequently observed in cancers \cite{c436} and continues to be an active area of research. Two of these studies reported null results \cite{c437,c438} and three studies, all of which were carried out in populations with colorectal carcinomas or adenomas, noted an increase in global DNA methylation following daily administration with 10 mg, 5 mg, and 400 mg folic acid, respectively \cite{c439,c440,c441}. In these studies, global DNA methylation was assessed only in colorectal tissues with the exception of one study that also measured global DNA methylation in leukocytes in addition to colonic mucosa \cite{c441}. In the latter study, there was a slight increase in global DNA methylation in peripheral blood leukocytes of individuals taking folic acid compared to those in the placebo group (P= 0.05), but not in colonic mucosa of colorectal adenoma patients following 400 mg daily folic acid supplementation for 10 weeks. Their study was conducted in colorectal adenoma patients, in whom folic acid metabolism may already be altered, subsequently affecting DNA methylation, which may explain for our differences in results. Our findings are aligned with those in other trials with cancer-free volunteers.

\noindent On the other hand, our findings are unexpected, as dietary folate depletion/repletion studies in women have previously reported global DNA hypomethylation reversal following folate repletion \cite{c48,c49,c410}. In the United States, Shelnutt \emph{et al}. conducted a separate folate feeding trial of young women aged 20 to 30, who were either \emph{MTHFR} 677CC or \emph{MTHFR} 677TT, participants were put on a folate depletion diet of 115 mg dietary folate equivalents (DFE)/day for 7 weeks followed by a 7-week folate repletion diet of 400 mg DFE/day. Overall, global DNA methylation (measured using the liquid chromatography tandem mass spectrometry assay identical to the one we used in our study) decreased after folate depletion diet, but following folate repletion with folic acid supplementation, global DNA methylation significantly increased only in those with the TT genotype \cite{c442,c443}. In contrast with these findings, we did not find an effect of folic acid treatment according to genotype. As expected, baseline global DNA methylation in this American population was higher than baseline levels in our study, which was conducted in the Netherlands, where unlike the United States, folic acid fortification is not mandatory. Furthermore, baseline global DNA methylation in the Shelnutt \emph{et al}. study was still higher than global DNA methylation in our population even after 3 years folic acid supplementation. This may be explained in part by the fact that our study population is substantially older than theirs and that their study only included women. Given that DNA hypomethylation seems to increase with age, reversion of DNA hypomethylation may be more difficult and may take longer in older individuals. Furthermore, there may be differences between the effects of folic acid supplementation (the synthetic form of folate) and natural dietary folate supplementation on DNA methylation, as folic acid and natural dietary folates enter the folate-mediated one-carbon metabolism cycle as different folate vitamers and may impact DNA methylation by uniquely altering metabolic flux of reactions in folate-mediated one-carbon metabolism.

\noindent In a study by Ingrosso \emph{et al}. \cite{c443} that was partly designed to investigate the effects of folic acid administration on DNA methylation in men with hyperhomocysteinemia and uremia, DNA hypomethylation was higher in those with hyperhomocysteinemia compared to controls. Following supplementation with 15 mg of 5-methyltetrahydrofolate daily for 8 weeks, this DNA hypomethylation was reversed, which does not corroborate with our own findings. Although not a randomised controlled trial, the study conducted by Ingrosso \emph{et al}. is insightful in presenting evidence of a lowering effect of the metabolically active form of folate on DNA methylation in men with hyperhomocysteinemia.

\noindent Our study is not without limitations. Firstly, we restricted the study population to those likely to benefit from folic acid supplementation; only those with elevated homocysteine were selected to participate in the FACIT trial since high homocysteine is a very sensitive indicator of low folate status. Intuitively, effects of folic acid supplementation are expected to be most pronounced in these individuals. We were not able to test whether effects are different in those with lower homocysteine concentrations. Genetic variability and other environmental factors and interactions between any of these also play a large role in folate-mediated one-carbon metabolism \cite{c444} and may subsequently influence DNA methylation \cite{c445}. Furthermore, while our results suggest that folic acid supplementation does not increase global DNA methylation, this does not preclude the possibility that folic acid supplementation has an effect on the methylation of specific individual genes, possibly in certain tissues. Folic acid supplementation may also have an effect on DNA methylation in subgroups of the population, such as those where polymorphisms from other folate-metabolising genes have been observed to influence folate \cite{c446} and homocysteine \cite{c446,c447} concentrations.

\noindent Despite these limitations, our study has some noteworthy strengths. We were able to include two hundred sixteen participants in our analyses, making this the largest study to date investigating the effects of folic acid supplementation on global DNA methylation in healthy volunteers. Furthermore, we were able to carry out our study prospectively over three years, and this relatively long duration of folic acid supplementation can give us a better idea of the longer-term effects of folic acid.

\noindent In conclusion, our study shows that daily supplementation with 800 mg folic acid, while increasing folate status in blood, did not simultaneously alter long-term global DNA methylation in leukocytes of individuals with elevated homocysteine. The role of natural and synthetic folates in DNA methylation remains an area for further research. The relationship between leukocyte DNA methylation and DNA methylation in specific tissues should also be explored as well as genetic variability of folate-mediated one-carbon metabolism genes and other environmental factors that may have an impact on DNA methylation.
 
\begin{thebibliography}{12} 
	\bibitem{c41}	Goelz SE, Vogelstein B, Hamilton SR, Feinberg AP. Hypomethylation of DNA from benign and malignant human colon neoplasms. Science (New York, NY. 1985 Apr 12;228(4696):187-90. 
	\bibitem{c42}	Swain JL, Stewart TA, Leder P. Parental legacy determines methylation and expression of an autosomal transgene: a molecular mechanism for parental imprinting. Cell. 1987 Aug 28;50(5):719-27. 
	\bibitem{c43}	Wajed SA, Laird PW, DeMeester TR. DNA methylation: an alternative pathway to cancer. Annals of surgery. 2001 Jul;234(1):10-20. 
	\bibitem{c44}	Wood AJ, Oakey RJ. Genomic imprinting in mammals: emerging themes and established theories. PLoS genetics. 2006 Nov 24;2(11):e147. 
	\bibitem{c45}	Chen RZ, Pettersson U, Beard C, Jackson-Grusby L, Jaenisch R. DNA hypomethylation leads to elevated mutation rates. Nature. 1998 Sep 3;395(6697):89-93. 
	\bibitem{c46}	Keshet I, Yisraeli J, Cedar H. Effect of regional DNA methylation on gene expression. Proceedings of the National Academy of Sciences of the United States of America. 1985 May;82(9):2560-4. 
	\bibitem{c47}	Kim YI, Pogribny IP, Basnakian AG, Miller JW, Selhub J, James SJ, et al. Folate deficiency in rats induces DNA strand breaks and hypomethylation within the p53 tumor suppressor gene. The American journal of clinical nutrition. 1997 Jan;65(1):46-52. 
	\bibitem{c48}	Jacob RA, Gretz DM, Taylor PC, James SJ, Pogribny IP, Miller BJ, et al. Moderate folate depletion increases plasma homocysteine and decreases lymphocyte DNA methylation in postmenopausal women. The Journal of nutrition. 1998 Jul;128(7):1204-12. 
	\bibitem{c49}	Rampersaud GC, Kauwell GP, Hutson AD, Cerda JJ, Bailey LB. Genomic DNA methylation decreases in response to moderate folate depletion in elderly women. The American journal of clinical nutrition. 2000 Oct;72(4):998-1003. 
	\bibitem{c410}	Axume J, Smith SS, Pogribny IP, Moriarty DJ, Caudill MA. The MTHFR 677TT genotype and folate intake interact to lower global leukocyte DNA methylation in young Mexican American women. Nutrition research (New York, NY. 2007 Jan;27(1):1365-17. 
	\bibitem{c411}	Kim YI, Christman JK, Fleet JC, Cravo ML, Salomon RN, Smith D, et al. Moderate folate deficiency does not cause global hypomethylation of hepatic and colonic DNA or c-myc-specific hypomethylation of colonic DNA in rats. The American journal of clinical nutrition. 1995 May;61(5):1083-90. 
	\bibitem{c412}	Bagley PJ, Selhub J. A common mutation in the methylenetetrahydrofolate reductase gene is associated with an accumulation of formylated tetrahydrofolates in red blood cells. Proceedings of the National Academy of Sciences of the United States of America. 1998 Oct 27;95(22):13217-20. 
	\bibitem{c413}	Kluijtmans LA, Young IS, Boreham CA, Murray L, McMaster D, McNulty H, et al. Genetic and nutritional factors contributing to hyperhomocysteinemia in young adults. Blood. 2003 Apr 1;101(7):2483-8. 
	\bibitem{c414}	Ozturk H, Durga J, van de Rest O, Verhoef P. The MTHFR 677C>T genotype modifies the relation of folate intake and status with plasma homocysteine in middle-aged and elderly people. Nederlands Tijdschrift voor Klinische Chemie en Laboratoriumgeneeskunde. 2005;30:208-17. 
	\bibitem{c415}	Friso S, Choi SW, Girelli D, Mason JB, Dolnikowski GG, Bagley PJ, et al. A common mutation in the 5,10-methylenetetrahydrofolate reductase gene affects genomic DNA methylation through an interaction with folate status. Proceedings of the National Academy of Sciences of the United States of America. 2002 Apr 16;99(8):5606-11. 
	\bibitem{c416}	Durga J, Bots ML, Schouten EG, Kok FJ, Verhoef P. Low concentrations of folate, not hyperhomocysteinemia, are associated with carotid intima-media thickness. Atherosclerosis. 2005 Apr;179(2):285-92. 
	\bibitem{c417}	Durga J, van Boxtel MP, Schouten EG, Kok FJ, Jolles J, Katan MB, et al. Effect of 3-year folic acid supplementation on cognitive function in older adults in the FACIT trial: a randomised, double blind, controlled trial. Lancet. 2007 Jan 20;369(9557):208-16. 
	\bibitem{c418}	Durga J, Verhoef P, Anteunis LJ, Schouten E, Kok FJ. Effects of folic acid supplementation on hearing in older adults: a randomised, controlled trial. Annals of internal medicine. 2007 Jan 2;146(1):1-9. 
	\bibitem{c419}	Durga J, van Boxtel MP, Schouten EG, Bots ML, Kok FJ, Verhoef P. Folate and the methylenetetrahydrofolate reductase 677C $\rightarrow$ T mutation correlate with cognitive performance. Neurobiology of aging. 2006 Feb;27(2):334-43. 
	\bibitem{c420}	Fraga MF, Ballestar E, Paz MF, Ropero S, Setien F, Ballestar ML, et al. Epigenetic differences arise during the lifetime of monozygotic twins. Proceedings of the National Academy of Sciences of the United States of America. 2005 Jul 26;102(30):10604-9. 
	\bibitem{c421}	Wilson VL, Smith RA, Ma S, Cutler RG. Genomic 5-methyldeoxycytidine decreases with age. The Journal of biological chemistry. 1987 Jul 25;262(21):9948-51. 
	\bibitem{c422}	Issa JP, Ottaviano YL, Celano P, Hamilton SR, Davidson NE, Baylin SB. Methylation of the oestrogen receptor CpG island links ageing and neoplasia in human colon. Nature genetics. 1994 Aug;7(4):536-40. 
	\bibitem{c423}	Richardson BC. Role of DNA methylation in the regulation of cell function: autoimmunity, aging and cancer. The Journal of nutrition. 2002 Aug;132(8 Suppl):2401S-5S. 
	\bibitem{c424}	Soma T, Kaganoi J, Kawabe A, Kondo K, Imamura M, Shimada Y. Nicotine induces the fragile histidine triad methylation in human esophageal squamous epithelial cells. International journal of cancer. 2006 Sep 1;119(5):1023-7. 
	\bibitem{c425}	Marsit CJ, McClean MD, Furniss CS, Kelsey KT. Epigenetic inactivation of the SFRP genes is associated with drinking, smoking and HPV in head and neck squamous cell carcinoma. International journal of cancer. 2006 Oct 15;119(8):1761-6. 
	\bibitem{c426}	Ubbink JB, Hayward Vermaak WJ, Bissbort S. Rapid high-performance liquid chromatographic assay for total homocysteine levels in human serum. Journal of chromatography. 1991 Apr 19;565(1-2):441-6. 
	\bibitem{c427}	van de Rest O, Durga J, Verhoef P, Melse-Boonstra A, Brants HA. Validation of a food frequency questionnaire to assess folate intake of Dutch elderly people. The British journal of nutrition. 2007 Nov;98(5):1014-20. 
	\bibitem{c428}	Kok RM, Smith DE, Barto R, Spijkerman AM, Teerlink T, Gellekink HJ, et al. Global DNA methylation measured by liquid chromatography-tandem mass spectrometry: analytical technique, reference values and determinants in healthy subjects. Clin Chem Lab Med. 2007;45(7):903-11. 
	\bibitem{c429}	van Driel LM, Eijkemans MJ, de Jonge R, de Vries JH, van Meurs JB, Steegers EA, et al. Body mass index is an important determinant of methylation biomarkers in women of reproductive ages. The Journal of nutrition. 2009 Dec;139(12):2315-21. 
	\bibitem{c430}	Kim M, Long TI, Arakawa K, Wang R, Yu MC, Laird PW. DNA methylation as a biomarker for cardiovascular disease risk. PloS one. 2010;5(3):e9692. 
	\bibitem{c431}	Sarter B, Long TI, Tsong WH, Koh WP, Yu MC, Laird PW. Sex differential in methylation patterns of selected genes in Singapore Chinese. Human genetics. 2005 Aug;117(4):402-3. 
	\bibitem{c432}	El-Maarri O, Becker T, Junen J, Manzoor SS, Diaz-Lacava A, Schwaab R, et al. Gender specific differences in levels of DNA methylation at selected loci from human total blood: a tendency toward higher methylation levels in males. Human genetics. 2007 Dec;122(5):505-14. 
	\bibitem{c433}	El-Maarri O, Walier M, Behne F, van Uum J, Singer H, Diaz-Lacava A, et al. Methylation at global LINE-1 repeats in human blood are affected by gender but not by age or natural hormone cycles. PloS one. 2011;6(1):e16252. 
	\bibitem{c434}	Fenech M, Aitken C, Rinaldi J. Folate, vitamin B12, homocysteine status and DNA damage in young Australian adults. Carcinogenesis. 1998 Jul;19(7):1163-71. 
	\bibitem{c435}	Basten GP, Duthie SJ, Pirie L, Vaughan N, Hill MH, Powers HJ. Sensitivity of markers of DNA stability and DNA repair activity to folate supplementation in healthy volunteers. British journal of cancer. 2006 Jun 19;94(12):1942-7. 
	\bibitem{c436}	Feinberg AP, Vogelstein B. Hypomethylation distinguishes genes of some human cancers from their normal counterparts. Nature. 1983 Jan 6;301(5895):89-92. 
	\bibitem{c437}	Cravo ML, Pinto AG, Chaves P, Cruz JA, Lage P, Nobre Leitao C, et al. Effect of folate supplementation on DNA methylation of rectal mucosa in patients with colonic adenomas: correlation with nutrient intake. Clinical nutrition (Edinburgh, Scotland). 1998 Apr;17(2):45-9. 
	\bibitem{c438}	Figueiredo JC, Grau MV, Wallace K, Levine AJ, Shen L, Hamdan R, et al. Global DNA hypomethylation (LINE-1) in the normal colon and lifestyle characteristics and dietary and genetic factors. Cancer Epidemiol Biomarkers Prev. 2009 Apr;18(4):1041-9. 
	\bibitem{c439}	Cravo M, Fidalgo P, Pereira AD, Gouveia-Oliveira A, Chaves P, Selhub J, et al. DNA methylation as an intermediate biomarker in colorectal cancer: modulation by folic acid supplementation. Eur J Cancer Prev. 1994 Nov;3(6):473-9. 
	\bibitem{c440}	Kim YI, Baik HW, Fawaz K, Knox T, Lee YM, Norton R, et al. Effects of folate supplementation on two provisional molecular markers of colon cancer: a prospective, randomised trial. The American journal of gastroenterology. 2001 Jan;96(1):184-95. 
	\bibitem{c441}	Pufulete M, Al-Ghnaniem R, Khushal A, Appleby P, Harris N, Gout S, et al. Effect of folic acid supplementation on genomic DNA methylation in patients with colorectal adenoma. Gut. 2005 May;54(5):648-53. 
	\bibitem{c442}	Shelnutt KP, Kauwell GP, Gregory JF, 3rd, Maneval DR, Quinlivan EP, Theriaque DW, et al. Methylenetetrahydrofolate reductase 677C $\rightarrow$ T polymorphism affects DNA methylation in response to controlled folate intake in young women. The Journal of nutritional biochemistry. 2004 Sep;15(9):554-60. 
	\bibitem{c443}	Ingrosso D, Cimmino A, Perna AF, Masella L, De Santo NG, De Bonis ML, et al. Folate treatment and unbalanced methylation and changes of allelic expression induced by hyperhomocysteinaemia in patients with uraemia. Lancet. 2003 May 17;361(9370):1693-9. 
	\bibitem{c444}	Guerreiro CS, Carmona B, Goncalves S, Carolino E, Fidalgo P, Brito M, et al. Risk of colorectal cancer associated with the C677T polymorphism in 5,10-methylenetetrahydrofolate reductase in Portuguese patients depends on the intake of methyl-donor nutrients. The American journal of clinical nutrition. 2008 Nov;88(5):1413-8. 
	\bibitem{c445}	van Engeland M, Weijenberg MP, Roemen GM, Brink M, de Bruine AP, Goldbohm RA, et al. Effects of dietary folate and alcohol intake on promoter methylation in sporadic colorectal cancer: the Netherlands cohort study on diet and cancer. Cancer research. 2003 Jun 15;63(12):3133-7. 
	\bibitem{c446}	Trinh BN, Ong CN, Coetzee GA, Yu MC, Laird PW. Thymidylate synthase: a novel genetic determinant of plasma homocysteine and folate levels. Human genetics. 2002 Sep;111(3):299-302. 
	\bibitem{c447}	Ho V, Massey TE, King WD. Influence of thymidylate synthase gene polymorphisms on total plasma homocysteine concentrations. Molecular genetics and metabolism. 2010 Sep;101(1):18-24. 
	\bibitem{c448}	Basten GP, Hill MH, Duthie SJ, Powers HJ. Effect of folic Acid supplementation on the folate status of buccal mucosa and lymphocytes. Cancer Epidemiol Biomarkers Prev. 2004 Jul;13(7):1244-9. 
\end{thebibliography} 
